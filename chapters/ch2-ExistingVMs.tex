\ifx\wholebook\relax\else

% --------------------------------------------
% Lulu:

    \documentclass[a4paper,12pt,twoside]{../includes/ThesisStyle}

	\usepackage[T1]{fontenc} %%%key to get copy and paste for the code!
%\usepackage[utf8]{inputenc} %%% to support copy and paste with accents for frnehc stuff
\usepackage{times}
\usepackage{ifthen}
\usepackage{xspace}
\usepackage{alltt}
\usepackage{latexsym}
\usepackage{url}            
\usepackage{amssymb}
\usepackage{amsfonts}
\usepackage{amsmath}
\usepackage{stmaryrd}
\usepackage{enumerate}
\usepackage{cite}
%\usepackage[pdftex,colorlinks=true,pdfstartview=FitV,linkcolor=blue,citecolor=blue,urlcolor=blue]{hyperref}
\usepackage{xspace}
%\usepackage{graphicx}
\usepackage{subfigure}
\usepackage[scaled=0.85]{helvet}
        
        
\newcommand{\sepe}{\mbox{>>}}
\newcommand{\pack}[1]{\emph{#1}}
\newcommand{\ozo}{\textsc{oZone}\xspace}
\newcommand\currentissues{\par\smallskip\textbf{Current Issues -- }}

\newboolean{showcomments}
\setboolean{showcomments}{true}
\ifthenelse{\boolean{showcomments}}
  {\newcommand{\bnote}[2]{
	\fbox{\bfseries\sffamily\scriptsize#1}
    {\sf\small$\blacktriangleright$\textit{#2}$\blacktriangleleft$}
    % \marginpar{\fbox{\bfseries\sffamily#1}}
   }
   \newcommand{\cvsversion}{\emph{\scriptsize$-$Id: macros.tex,v 1.1.1.1 2007/02/28 13:43:36 bergel Exp $-$}}
  }
  {\newcommand{\bnote}[2]{}
   \newcommand{\cvsversion}{}
  } 


\newcommand{\here}{\bnote{***}{CONTINUE HERE}}
\newcommand{\nb}[1]{\bnote{NB}{#1}}
\newcommand{\fix}[1]{\bnote{FIX}{#1}}
%%%% add your own macros 

\newcommand{\sd}[1]{\bnote{Stef}{#1}}
\newcommand{\ja}[1]{\bnote{Jannik}{#1}}
\newcommand{\na}[1]{\bnote{Nico}{#1}}
%%% 


\newcommand{\figref}[1]{Figure~\ref{fig:#1}}
\newcommand{\figlabel}[1]{\label{fig:#1}}
\newcommand{\tabref}[1]{Table~\ref{tab:#1}}
\newcommand{\layout}[1]{#1}
\newcommand{\commented}[1]{}
\newcommand{\secref}[1]{Section \ref{sec:#1}}
\newcommand{\seclabel}[1]{\label{sec:#1}}

%\newcommand{\ct}[1]{\textsf{#1}}
\newcommand{\stCode}[1]{\textsf{#1}}
\newcommand{\stMethod}[1]{\textsf{#1}}
\newcommand{\sep}{\texttt{>>}\xspace}
\newcommand{\stAssoc}{\texttt{->}\xspace}

\newcommand{\stBar}{$\mid$}
\newcommand{\stSelector}{$\gg$}
\newcommand{\ret}{\^{}}
\newcommand{\msup}{$>$}
%\newcommand{\ret}{$\uparrow$\xspace}

\newcommand{\myparagraph}[1]{\noindent\textbf{#1.}}
\newcommand{\eg}{\emph{e.g.,}\xspace}
\newcommand{\ie}{\emph{i.e.,}\xspace}
\newcommand{\ct}[1]{{\textsf{#1}}\xspace}


\newenvironment{code}
    {\begin{alltt}\sffamily}
    {\end{alltt}\normalsize}

\newcommand{\defaultScale}{0.55}
\newcommand{\pic}[3]{
   \begin{figure}[h]
   \begin{center}
   \includegraphics[scale=\defaultScale]{#1}
   \caption{#2}
   \label{#3}
   \end{center}
   \end{figure}
}

\newcommand{\twocolumnpic}[3]{
   \begin{figure*}[!ht]
   \begin{center}
   \includegraphics[scale=\defaultScale]{#1}
   \caption{#2}
   \label{#3}
   \end{center}
   \end{figure*}}

\newcommand{\infe}{$<$}
\newcommand{\supe}{$\rightarrow$\xspace}
\newcommand{\di}{$\gg$\xspace}
\newcommand{\adhoc}{\textit{ad-hoc}\xspace}

\usepackage{url}            
\makeatletter
\def\url@leostyle{%
  \@ifundefined{selectfont}{\def\UrlFont{\sf}}{\def\UrlFont{\small\sffamily}}}
\makeatother
% Now actually use the newly defined style.
\urlstyle{leo}



	\input{../includes/formatAndDefs}

	\graphicspath{{.}{../figures/}}
	\begin{document}
\fi

\chapter{State of the art}
\label{chap:stateOfTheArt}
\minitoc

The main focus of the thesis is the design and implementation of an optimising JIT architecture for Pharo called Sista. The goal of this architecture is to write the optimising JIT in Pharo itself and to have it running in the same runtime than the optimised application on top of the existing VM. Standard object-oriented languages feature dynamic dispatch. This feature is typically present in the form of virtual calls: the function to activate for each virtual call depends on information available at runtime but not at compile-time. Because of dynamic dispatch, it is difficult for an ahead-of-time compiler to optimise efficiently the code to execute. This problem is especially important for languages where virtual calls are very common. 

To efficiently optimise code in a language featuring dynamic dispatch, one solution is to use an optimising JIT. When a VM featuring an optimising JIT executes a code snippet multiple times, the execution goes through different phasis. The first runs of a code snippet are done through a slow execution path, such as an interpreter, which collects information about the running program while executing it. Once the code snippet has been run a significant number of times, the optimising JIT recompiles the code snippet at runtime to optimised native code. The compiler optimisations are directed by the runtime information collected during the first runs. Further uses of the same code snippet can be executed using the optimised version. The same code snippet can therefore be executed differently depending on how frequently it is run and how many times it has been executed since the last start-up. We call each different way the VM can execute the same code snippet a different \emph{tier}.

\paragraph{Multiple tiers.} As an optimising JIT requires runtime information to direct the compiler optimisations, high-performance VMs are implemented with at least two tiers. One tier, slow to execute code, is used for the first runs of a code snippet to collect runtime information. The other tier requires both runtime information and compilation time to generate optimised native code, but the resulting execution is faster. Conceptually, a high-performance VM can be implemented with many tiers: each tier requires more compilation time than the previous tier but the resulting generated native code is faster.

The tier concept is summarized in figure \ref{fig:GeneralTieredArchitecture} with a theoretical VM using two tiers. The first tier is an interpreter executing instructions present in v-functions. The interpreter requires no compilation time and takes 0.5ms to execute the given code snippet. Once the code snippet has been executed a hundred time since the last start-up, the optimising JIT kicks in and generates optimised native instructions using runtime information collected during the interpreter runs. The run hundred and one requires 5 ms of compilation time to generate the optimised version. However, once the optimised version is generated, subsequent runs of the same code snippet are much faster, taking 0.1 ms instead of 0.5 ms.

\begin{figure}[h!]
    \begin{center}
        \includegraphics[width=0.5\linewidth]{GeneralTieredArchitecture}
        \caption{Execution of a frequently used code snippet}
        \label{fig:GeneralTieredArchitecture}
    \end{center}
\end{figure}

\paragraph{Optimising JIT architectures.} Two main architectures are widely used to design an optimising JIT. The first optimising JIT architecture, historically the first one invented, attempts to boost performance by optimising frequently used functions. The native code generated by such optimising JITs are optimised version of functions. We call this architecture the \emph{Function based architecture} and we describe it in section \ref{sec:architecture}. The second architecture focus on the optimisation of linear sequences of frequently used instructions. We call this architecture the \emph{Meta-tracing architecture}. Typically, meta-tracing JITs optimise the common execution path of one iteration of each frequently used loop. This second architecture is detailled in section \ref{sec:metaArchitecture}. As the Sista architecture is more similar to the function based architecture, section \ref{sec:architecture} is more detailled than the other one.

\paragraph{Research problems.} In the context of the design and implementation of the optimising JIT for Pharo, the thesis focuses on 
two
%three 
aspects:
\begin{itemize}
	\item \emph{Metacircular optimising JITs: } Optimising JITs can be written in different programming languages, including the language they optimise. In the latter case, it may be possible for the JIT to optimise its own code. Such aspects are discussed in section \ref{sec:implLang}.
	\item \emph{Runtime state persistance:} Most modern VMs always start-up an application with only non optimised code. The application then needs a certain amount of time, called \emph{warm-up time}, to reach peak performance. Warm-up time is a problem if the application needs high-performance immediately. Existing solutions for this problem are detailled in section \ref{sec:persistance}.
	%\item \emph{Virtual function instruction set:} Most VMs are either only able to execute code from a specific programming language or from a specific virtual function representation. Several modern VMs are able to execute code in other form, for example
\end{itemize}

%what do we discuss
\paragraph{Closed-source VMs.} The chapter tries to discuss the main production and research open source VMs. Specific closed-source VMs are described as they are relevant in the context the thesis. However, most closed-source VMs are ignored as it is too difficult to get reliable information about them. 

%Smalltalk.
\paragraph{Smalltalk VMs.} Commercial Smalltalk VMs in production today are closed-source and do not feature optimising JITs so we do not discuss them. In the 90s, the Self VM (CITE) and the Strongtalk VM (CITE) were able to execute Smalltalk code using an optimising JIT. Those VMs are briefly discussed but as far as we know there are not used in production today and no further development on those VMs is in progress.

%%%%%%%%%%%%%%%%%%%%%%%%%%%%%%%%%%%%%%%%%%%%%%%%%%%%%%%%%%%%%%%%%%%%%%%%%%%%%%%%%%%%%%%%%%%%%%%%%%%%%%%%%%%%%%%%%%%%%%%%%%%%%%%%%%%%%%%%%%%%%%%%%%%%%%%%%%%%%%%%%%%%%%%

\section{Function based architecture}
\label{sec:architecture}

The first optimising JIT architecture invented was designed to generate optimised functions. From a given v-function, the optimising JIT performs a set of optimisations which includes inlining of other functions, and generates an optimised n-function.

\subsection{Architecture overview}

In most VMs following this architecture, three tiers are present. The following three paragraphs detail each tier, including how virtual calls is executed efficiently in each case.

\paragraph{Tier 1: v-function interpreter: } The first tier is a virtual function interpreter. In most VMs, no compilation time is required at all to interpret a v-function~\footnote{Some VMs require compilation time for interpretation because the v-functions are not provided in a format the interpreter can execute (for example source code is provided).} but the execution by the interpreter is not very fast. Virtual calls are usually implemented using a global look-up cache to avoid re-computing the function to activate at each call. The interpreter tier does not necessarily collect runtime information. 
\paragraph{Tier 2: baseline JIT: } The second tier is the baseline JIT, which generates from a single v-function a n-function with a very limited number of optimisations. Once compiled, the n-function is used to execute the function instead of interpreting the v-function. A small amount of time is wasted to generate the n-function but the execution of the n-function is faster than the v-function interpretation. The n-function generated by the baseline JIT will be instrospected to collect runtime information if the function is executed enough time to be optimised by the next tier. The goal of the baseline JIT code generation is therefore to provide reliable runtime information with limited performance overhead over generating efficent n-functions. Virtual calls are usually generated in machine code using inline caches (CITE): each virtual call has a local cache with the functions it has activated, both speeding-up the execution and collecting runtime information for the next tier.
\paragraph{Tier 3: optimising JIT: } The last tier is the optimising JIT, which generates an optimised n-function. The optimising JIT uses runtime information such as the inline cache data to speculate on what function is called at each virtual call, allowing to perform inlining and to generate the optimised n-function from multiple v-functions. Such optimisations greatly speed-up execution but are invalid if one of the compile-time speculation is not valid at runtime. In this case, the VM deoptimises the code and re-optimise it differently. The optimising JIT require more time than the baseline JIT to generate n-functions, but the generated code is much faster. The execution of virtual calls is not really relevant in this tier as most virtual calls are removed through inlining and most of the remaining ones are transformed to direct calls.

\begin{figure}[h!]
    \begin{center}
        \includegraphics[width=0.95\linewidth]{TieredArchitecture}
        \caption{Execution of a frequently used v-function}
        \label{fig:TieredArchitecture}
    \end{center}
\end{figure}

Figure \ref{fig:TieredArchitecture} shows the execution of a frequently used v-functions over the three tiers. The first few runs are interpreted, each run taking 0.5 ms. The following run requires some compilation time for the baseline JIT to kick in, but the function is then executed 2.5 times faster and runtime information is collected. Lastly, after 10,000 runs, the optimising JIT takes a significant amount of time to generate an optimised n-function, but the optimised version is three times faster than the baseline JIT version.

\paragraph{Terminology.} We note that in this architecture, which is one of the most common architecture for high-performance VMs today, two JIT tiers are present but are very different from each other. To disambiguate the two, in this thesis, we use the term \emph{baseline JIT} to discuss the JIT tier translating quickly v-functions to n-functions with a limited number of optimisations. We call the other tier the \emph{optimising JIT}. This tier translates multiple v-functions to highly optimised n-functions with speculative optimisations, based on the runtime information collected on n-functions generated by the baseline JIT. 

\subsection{Existing virtual machines}

The first VM featuring this architecture was the Self VM (CITE). The Self VM had only two tiers, the baseline JIT and the optimising JIT. The Self programming language has never really become popular so the VM is not really used anymore. 

The second VM built with this design was the animorphic VM for the Strongtalk programming language~\cite{Sun06}, a Smalltalk dialect. This VM is the first to feature three tiers. The first tier is a threaded code interpreter hence interpretation requires a small amount of compilation time to generate threaded code from the v-function. The two other tiers are the same as in the Self-VM. The animorphic VM has never reached production.

The hotspot VM (CITE) was implemented from the animorphic VM code base and has been the default Java VM provided by Oracle for more than a decade. In the first versions of the hotspot VM, two executables were distributed. One was called the client VM, which included only the baseline JIT and was distributed for applications where start-up performance matters. The other one was called the server VM, which included both JIT tiers, and was distributed for application where peak performance matters. Later, the optimising JIT was introduced in the client VM with different optimisation policies than the server version to improve the client VM performance without decreasing too much start-up performance. In Java 6 and onwards, the server VM became the default VM as new strategies allowed the optimising JIT to improve performance with little impact on start-up performance. Lastly, since the 64 bits release, only one binary is distributed, the server VM. 

More recently, the V8 Javascript engine (CITE), used for Google Chrome and Node JS was built with a similar design. Other VMs, less popular than the Java and Javascript VMs are also using similar architectures, such as the Dart VM (CITE). One research project, the Graal compiler (CITE), is an optimising JIT for Java that can be used, among multiple use-cases, as an alternative optimising JIT in the hotspot VM.

\subsection{Just-in-time compiler tiers}

Most VMs featuring this function based architecture have three tiers. The number of tiers may however vary from two to as many as the development team feels like. The following paragraphs discuss the reasons why the VM implementors may choose to two tiers, three tiers or more.

\paragraph{Engineering cost.} Each new tier needs to be maintained and evolved accordingly to the other tiers. Hence, a VM having more tiers requires more engineering time for maintainance and evolutions. Any bug can come from any tier and bugs coming from only a single tier can be difficult to track down. Evolutions need to be implemented on each tier. To lower the VM maintenance and evolution cost, a VM needs to have the least number of tiers possible.

\paragraph{Minimum number of tiers.} By design, the optimising JIT is the key component for high-performance and it needs runtime information from previous runs to generate optimised code. Hence, a VM with this architecture requires at least two tiers. One tier, the non optimising tier is used for the first runs to collect statistical information and is typically implemented as an interpreter tier or a baseline JIT tier. The second tier, the optimising tier, generates optimised n-functions and is implemented as an optimising JIT. To perform well, the optimising tier has to kick in only if the function is used frequently (else the compilation time would not be worth the execution time saved) and the previous tier(s) must have executed the v-function enough time to have collected reliable runtime information. For this reason, the optimising tier usually kicks in after several thousands executions of the v-function by the previous tier(s).

\paragraph{Two non optimising tiers.}Most VMs features two non-optimising tiers and one optimising tier. The non optimising tiers are composed of an interpreter tier and a baseline JIT tier. These two tiers have different pros and cons and featuring both allows the VM to have the best of both worlds. The main differences between the two tiers are the following:

\begin{enumerate}
	\item \emph{Speed: } The interpreter tier is faster is the function is executed a very small number of times. In the example in figure \ref{fig:TieredArchitecture}, interpreting the v-function twice takes 1 ms (2 * 0.5 ms) while compiling it with the baseline JIT and executing it twice takes 1.4 ms (2 * 0.2 + 1 ms). However, the baseline JIT is faster if the function is executed some more times. As the optimising JIT kicks in after thousands of execution, the performance overhead of the interpreter can be significant, especially during application start-up.
	\item \emph{Runtime information collection: }One of the most relevant runtime information to collect is the function called at each virtual call. It is currently not possible to collect this information without overhead in an interpreter tier if the interpreter is written in a machine independent language. However, the inline cache technique (CITE) allows to collect such information in a baseline JIT tier while speeding-up code execution.
	\item \emph{Memory footprint: }The interpreter does not require extra memory for the function it executes as only the v-function representation is needed. On the other hand, the baseline JIT requires memory to store the n-function for each v-function it executes. If many functions are executed once, not having the interpreter tier can lead to significant memory overhead.
	%\item \emph{Compatibility with optimising JIT: }The code executed by the processor in the interpreter tier is the interpreter code while in the baseline JIT tier is native code. Having both an interpreter and JIT tiers require the runtime to be able to switch efficiently between both runtime, including the management of stack frame representation in both cases.
\end{enumerate}
Having both tiers allows the VM to have a lower memory footprint thanks to the interpreter tier. If the interpreter tier does not collect runtime information and is used only for the first few executions, the start-up performance is much better when both tier are present than when one or the other is present alone and runtime information can be collected with less overhead. For these reasons, many VMs feature these two non optimising tiers. 

A good example of the pros and cons of multiple tiers is the V8 javascript engine. In 2008, the first version was released featuring only the baseline JIT tier. The following year, the optimising JIT tier was added to improve performance. In 2016, the interpreter tier was added both to lower the memory footprint and to improve start-up performance.

In general, the two non optimising tiers are kept as simple as possible to ease maintainance and evolutions. Only the third tier, the optimising JIT, is complex to be able to generate efficient n-functions.

\paragraph{More than three tiers.} Adding more than three tiers is usually not worth it as it would mean additional maintenance and evolution cost. However, in specific languages such as Javascript where the start-up performance is critical, it can be worth it to have two optimising JIT tiers to increase start-up performance. The Javascript Webkit VM has four tiers since 2015 (CITE). In this case, the VM team introduced two optimising JIT tiers after the interpreter and baseline JIT. One optimising JIT tier has smaller compilation time than the other one but produce less efficient n-function.

\paragraph{Independant compiler tiers.}
In most VMs, the baseline JIT and the optimising JIT are completely independant entities.

The baseline JIT has multiple constraints that incompatible with the optimising JIT. The baseline JIT has the following constraints:
\begin{itemize}
	\item It has to be as simple as possible to limit the maintenance cost, simplicity is more important than generated code quality as most of the VM performance comes from the optimising JIT.
	\item The n-function it generates needs to be easily introspected to collect runtime information about the previous runs for the optimising JIt to direct compiler optimisations.
	\item The compilation time has to be very small.
\end{itemize}
Based on this constraints, the baseline JIT is typically implemeted as a template-based engine, generating a predefined sequence of native instructions for each virtual instruction. Template-based generation engines are relatively simple to implement and maintain. Templates are very convenient for native code introspection because the JIT knows the exact sequence of instructions generated for each virtual instruction so it can know the exact bytes to read to extract runtime information. Lastly, template-based compilation is usually very efficient, providing low compilation time.

The optimising JIT is very different. It requires runtime information in addition from the v-functions. The main constraint of the optimising JIT is to generate n-function as efficient as possible with a compilation reasonnable, but much potentially higher than the baseline JIT. The n-function generated by the optimising JIT are in most VMs not introspected, allowing the optimising JIT to generate the most efficient instructions. As any software project, complexity has to be controlled but it is usually worth it to add complexity in the optimising JIT to allow it to generate more efficient code because it leads to significantly better VM performance. The optimising JIT typically translates the v-function to a high-level intermediate representation to perform language-specific optimisations. It then lowers it to another intermediate representation, closer to native instruction, where machine-specific optimisations are performed. Lastly it generates native code.

\paragraph{Sharing code between compiler tiers.} Because of the fondamental differences, most optimising JITs use a completely different code base than the baseline JIt they work with. However, there are some rare cases where part of the JIT compilers are shared between multiple tiers. 

The first case is the Javascript Webkit VM. As four tiers are present (CITE), it is possible to share portion of the compilers because some features are required in multiple tiers. For example, both the baseline JIT and the first-level optimising JIT requires the VM to be able to instrospect the generated machine code. In addition, both optimising JITs perform similar optimisations (CITE) allowing to share part of the optimisation pipeline. In this case, they share the high-level optimisations while the low-level optimisations are done in different back-ends.

The second case is related to VM extensions. The Javascript VMs are now attempting to support, in addition of Javascript, an abstract assembly language called WebAssembly. WebAssembly allows the programmer to compile ahead-of-time specific framework or libraries for use-cases difficult to optimise efficiently at runtime, such as real-time video games. WebAssembly provides both abstract assembly code instructions and convenient instructions to interface the assembly code with Javascript and the web page. In the V8 Javascript engine, the low-level intermediate representation of TurboFan, the optimising JIT of V8, is shared between the WebAssembly back-end and the optimisation path for Javascript code.

\subsection{Concurrent compilation.}

The first optimising JIT, designed and implemented in Self, was done in single-threaded environment. In this case, the optimising JIT had a limited time period to produce optimised code, and if the time period was not enough, the function was not optimised. Since the early 2000s, multi-threaded environment has become more common and many optimising JITs now perform optimisations concurrently to the application native thread(s) (CITE Adaptive Optimization in the Jalapeno JVM: The Controller's Analytical Model AND maybe something with Graal).

The part of the optimisation pipeline that is done concurrently needs however to be precised. The baseline JIT is typically executed in the same native thread than the application. As it has very small compilation time, the compilation time overhead is usually not significant. When a frequently used portion of code is detected, the optimising JIT has to choose a function to optimise based on the current stack. This cannot be done concurrently as the stack needs to be introspected. Once the function to optimise is chosen, the optimisation of the function can be done concurrently. The optimising JIT has access to a pool of native threads which take functions to optimise in a compilation queue, optimise them and installs them. Further calls on such functions can use the optimised version installed. However, the optimising JIT may insert guards to ensure assumption taken (such as the type of a specific variable) is valid at runtime. If one of the guard fails, the stack needs to be deoptimised to resume with non optimised code. The deoptimisation process cannot be done concurrently (the application needs the deoptimisation to be finished to resume execution). 

%%%%%%%%%%%%%%%%%%%%%%%%%%%%%%%%%%%%%%%%%%%%%%%%%%%%%%%%%%%%%%%%%%%%%%%%%%%%%%%%%%%%%%%%%%%%%%%%%%%%%%%%%%%%%%%%%%%%%%%%%%%%%%%%%%%%%%%%%%%%%%%%%%%%%%%%%%%%%%%%%%%%%%%

\section{Meta-tracing architecture}
\label{sec:metaArchitecture}

The main alternative to the function based architecture is the meta-tracing design. Meta-tracing JITs do not optimise entire functions but instead focus on optimising linear sequence of instructions. As most meta-tracing JITs focus on the optimisation of loop iterations, we detail this case in this section.

\subsection{Architecture overview}

VMs with meta-tracing JITs generate optimised native code only for the frequently used paths of commonly executed loops and interpret virtual instructions for the rest of the program. Tracing JITs are built on the following basic assumptions:
\begin{itemize}
	\item Programs spend most of their runtime in loops.
	\item Several iterations of the same loop are likely to take similar code paths.
\end{itemize}

Typically, in VMs with tracing JITs, the first executions of a loop are done using a v-function interpreter. The interpreter profiles the code executed to detect frequently used loop, usually by having a counter on each backward jump instruction that counts how often this particular backward jump is executed. When a hot loop is identified, the interpreter enters a special mode, called tracing mode. During tracing, the interpreter records a history of all the operations it executes during a single execution of the hot loop. The history recorded by the tracer is called a trace: it is a list of operations, together with their actual operands and results. Such a trace can be used to generate efficient native code. This generated machine code is immediately executable and can be used in the next iteration of the loop.

Being sequential, the trace represents only one of the many possible paths through the code. To ensure correctness, the trace contains a guard at every possible point where the path could have followed another direction, for example at conditionnal branches or virtual calls. When generating the native code, every guard is turned into a quick check to guarantee that the path we are executing is still valid. If a guard fails, the execution immediately quits the native code and resumes the execution by falling back to the interpreter.

\subsection{Existing VMs}

Tracing optimisations were initially explored by the Dynamo project~\cite{Bala00a} to dynamically optimise native code at runtime. Its techniques were then successfully used to implement a JIT compiler for a Java VM~\cite{Gal06a}. The technique was used in Mozilla's JavaScript VM from Firefox 3 to Firefox 11~\cite{Gal09a} until Mozilla removed it to replace it by a function-based JIT. 

The most famous meta-tracing JITs in production are certainly the ones generated from the RPython toolchain~\cite{Rigo06a}. The RPython toolchain allows the generation of a meta-tracing JIT for free if one writes a virtual function interpreter in RPython. The most popular example is Pypy (CITE), a Python VM using a meta-tracing JIT through the RPython toolchain framework.

%%%%%%%%%%%%%%%%%%%%%%%%%%%%%%%%%%%%%%%%%%%%%%%%%%%%%%%%%%%%%%%%%%%%%%%%%%%%%%%%%%%%%%%%%%%%%%%%%%%%%%%%%%%%%%%%%%%%%%%%%%%%%%%%%%%%%%%%%%%%%%%%%%%%%%%%%%%%%%%%%%%%%%%

\section{Metacircular optimising Just-in-time compiler}
\label{sec:implLang}

In this section, the languages used to implement VMs, including optimising JITs, are discussed. In some cases, the whole VM, or only the optimising JIT, is written in the language it optimises. In this case, the optimising JIT may be able to optimise its on code.

\subsection{Virtual machine implementation language}

\paragraph{Low-level language.}
Historically, VMs have been implemented in low-level languages such as C or C++. Because of this, the first people having skills in high-performance VM development were also very good in low-level languages. Hence many VMs were implemented afterwards using this kind of language. Low-level languages are very convenient for multiple VM development tasks, such as direct memory access or optimisation of specific portion of code for VM performance. However, they lack abstractions, making it difficult for non low-level experts to contribute to the VM.

Many optimising JITs are written in low-level languages, especially C++. The main examples would be the Self, Java hotspot or V8 optimising JITs. As far as we know none of them is able to compile C++ code, so they are not able to optimise their own code.

\paragraph{High-level language compiled ahead-of-time.}
Another approach is to use a high-level language compiled ahead-of-time to assembly code. This approach is used by the RPython toolchain~\cite{Rigo06a}, where RPythong is a restricted Python that can be compiled to native code through C. This approach is also used in the existing Pharo VM, which is written in a restricted Smalltalk called Slang (CITE) that can also be compiled to native code through C, though this second case has no optimising JIT.

In RPython, the optimising JIT could potentially optimise its own code as RPython is a subset of python. However, in the current execution model where the optimising JIT code is compiled ahead-of-time to native code, the optimising JIT cannot optimise its own code at runtime as only native code is present.

\paragraph{Metacircular VMs.}

Multiple research work has shown that it is possible to implement a VM in the language the VM runs. It has been done firstly in the Jalape\~no project, now called Jikes RVM (CITE), a Java VM written in Java. Another Java VM written in Java was implemented in the late 2000s at Oracle, called Maxine VM. There were other attempts to implement meta-circular VMs for other languages than Java. For instance, the Klein VM~\cite{Unga05b} is a Self VM written in Self, the Tachyon VM is a Javascript VM written in Javascript and there is even an attempt to write a Smalltalk VM in Smalltalk (CITE), called Bee Smalltalk, which is unfortunately, as of today, closed-source. In this case, the optimising JIT is able to optimise its own code.

Another case it the Truffle framework (CITE). Truffle is a framework allowing to write efficiently VMs for different programming languages running on top of the Java runtime using the Graal optimising JIT. There are VM implementations, using Truffle, of multiple programming languages. In each case, the Graal compiler in the Truffle runtime can optimise both Java code and the programming language run. As Graal is written in Java, it can optimise its own code.
 
\subsection{Optimising Just-in-time compiler optimising itself}

To be able to optimise its own code, the optimising JIT compiler needs to be implemented in a programming language it can compile. In addition, it needs to run in the same runtime than the optimised application. Two main category of optimising JIT compilers fulfill these constraints: the optimising JITs present in the metacircular VMS and the Graal optimising JIT in the context where Graal is used as an alternative optimising compiler for Java hotspot.

The first VM where it was possible for the optimising JIT to optimise itself was the Jalape\~no VM, written in Java for Java. When the optimising JIT was released, it used concurrent native threads to optimise code (CITE Adaptive Optimization in the Jalapeno JVM: The Controller's Analytical Model).

TODO
They optimise themselves in concurrent threads. Still Graal multi threaded. Explain stop for stack search and deopt.

%%%%%%%%%%%%%%%%%%%%%%%%%%%%%%%%%%%%%%%%%%%%%%%%%%%%%%%%%%%%%%%%%%%%%%%%%%%%%%%%%%%%%%%%%%%%%%%%%%%%%%%%%%%%%%%%%%%%%%%%%%%%%%%%%%%%%%%%%%%%%%%%%%%%%%%%%%%%%%%%%%%%%%%

\section{Runtime state persistance}
\label{sec:persistance}

In our work, we will attempt to persist the runtime state across multiple start-ups, including the optimised code but also the running green threads using optimised code in their stack. It does not seem that the persistance of running green thread with optimised code has been done before. Most likely, we needed to persist green threads as the normal Smalltalk developer workflow requires it, but the problem does not exist in most other programming languages. For this reason, we focus in this section on the persistance of optimised code between multiple start-ups.

One of the main problem with optimising JITs, compared to ahead-of-time compiler, is the start-up performance. As the optimising JIT needs runtime information to optimise code, multiple (usually thousands of) non optimised runs of a code snippet are required before reaching peak performance. This warm-up time can cause significant problems in specific applications. In addition, the warm-up time implies additional energy consumption (the processor needs to optimise code, consuming energy) at each application start-up, which is a problem in specific use-cases such as mobiles where battery consumption is critical.

Because of these constrains, some object-oriented languages are compiled with an ahead-of-time compiler. Static analysis are performed over the code to guess what function is called at each virtual call. Applications for the iPhone are a good example where static analysis is used to pre-optimize the Objective-C application. The peak performance is lower than with a JIT compiler if the program uses a lot of virtual calls, as static analysis are not as precised as runtime information on highly dynamic language. However, if the program uses few dynamic features (for example most of the calls are not virtual) and is running on top of a high-performance language kernel like the Objective-C kernel, the result can be satisfying.

Most object-languages still choose to run on top of a VM with an optimising JIT. The section describes four existing techniques to improvestart-up performance, including techniques related to optimised code persistance across start-ups.

\paragraph{Many tiers architecture}
One solution to decrease warm-up time is to have many tiers in the function based architecture. The idea is that code would be executed slowly the few first iterations, a bit faster the next iterations, and very quickly after an certain number of optimizations, and so on. This way, a very good trade off between compilation time, runtime information quality and code performance can be achieved.

The best example is the Javascript Webkit VM~\cite{Webkit15}. A code snippet is:
\begin{itemize}
\item interpreted by a bytecode interpreter the first 6 executions.
\item compiled to machine code at 7th execution, with a non optimizing compiler, and executed as machine code up to 66 executions.
\item recompiled to more optimized machine code at 67th execution, with an optimizing compiler doing some but not all optimisations, up to 666 executions.
\item recompiled to heavily optimized machine code at 667th execution, with all the optimisations.
\end{itemize}

At each step, the compilation time is greater but the execution time decreases. The many tiers approach (four tiers in the case of Webkit), allows to have good performance from start-up, while reaching high performance for long running code. However, this technique has severe drawbacks: the VM team needs to maintain and evolve four different tiers.

\paragraph{Persisting metadata.}

To reach quickly peak performance, one way is to save the runtime information, especially inlining decisions made by the optimising JIT and if they were valid on the long run or not. In Strongtalk \cite{Sun06}, it is possible to save the inlining decision of the optimising compiler in a separate file. The optimizing compiler can then reuse this file to take the right inlining decision in subsequent start-ups.

\paragraph{Persisting machine code.}

In the Azul VM Zing \cite{Azul}, available for Java, the official web site claims that "operations teams can save accumulated optimizations from one day or set of market conditions for later reuse" thanks to the technology called \emph{Ready Now!}. In addition, the website precises that the Azul VM provides an API for the developer to help the JIT to make the right optimization decisions. As Azul is closed source, implementation details are not entirely known. However, word has been that the Azul VM reduces the warm-up time by saving machine code across multiple start-ups. If the application is started on another processor, then the saved machine code is simply discarded. It is very difficult to persist optimised native code across multiple start-ups due to position dependent code and low-level details, but with the example of Azul, we know it is possible.

\paragraph{Preheating through snapshots in Dart}

The Dart programming languages features snapshots for fast application start-up. In Dart, the programmer can generate different kind of snapshots \cite{Anna13a}. Since that publication, the Dart team have added two new kind of snapshots, specialized for iOS and Android application deployment, which are the most similar to our snapshots.

\subparagraph{Android.} A Dart snapshot for an Android application is a complete representation of the application code and the heap once the application code has been loaded but before the execution of the application. The Android snapshots are taken after a warm-up phase to be able to record call site caches in the snapshot. The call site cache is a regular heap object accessed from machine code, and its presence in the snapshot allows to persist type feedback and call site frequency.

In this case, the code is loaded pre-optimized with inline caches prefilled values. However, optimized functions are not loaded as our architecture allows to do. Only unoptimized code with precomputed runtime information is loaded.

\subparagraph{iOS.} For iOS, the Dart snapshot is slightly different as iOS does not allow JIT compilers. All reachable functions from the iOS application are compiled ahead of time, using only the features of the Dart optimizing compiler that don't require dynamic deoptimization. A shared library is generated, including all the instructions, and a snapshot that includes all the classes, functions, literal pools, call site caches, etc.

This second case is difficult to compare to our architecture: iOS forbids machine code generation, which is currently required by our architecture. A good application of our architecture to iOS is future work.

%%%%%%%%%%%%%%%%%%%%%%%%%%%%%%%%%%%%%%%%%%%%%%%%%%%%%%%%%%%%%%%%%%%%%%%%%%%%%%%%%%%%%%%%%%%%%%%%%%%%%%%%%%%%%%%%%%%%%%%%%%%%%%%%%%%%%%%%%%%%%%%%%%%%%%%%%%%%%%%%%%%%%%%

%\section{Virtual machine interface}

%\subsection{The Graal-Hotspot architecture}

%+ Graal and interface

%\subsection{WebAssembly}

%WebAssembly 

%%%%%%%%%%%%%%%%%%%%%%%%%%%%%%%%%%%%%%%%%%%%%%%%%%%%%%%%%%%%%%%%%%%%%%%%%%%%%%%%%%%%%%%%%%%%%%%%%%%%%%%%%%%%%%%%%%%%%%%%%%%%%%%%%%%%%%%%%%%%%%%%%%%%%%%%%%%%%%%%%%%%%%%

\paragraph{Conclusion.} This chapter detailled existing solution for our research problems, including the existing optimising JIT architectures, their implementation languages and how some VMs persist optimised code across multiple start-ups. The following chapter describe the existing Pharo runtime which was used as a starting point for our implementation.

%%%%%%%%%%%%%%%%%%%%%%%%%%%%%%%%%%%%%%%%%%%%%%%%%%%%%%%%%%%%%%%%%%%%%%%%%%%%%%%%%%%%%%%%%%%%%%%%%%%%%%%%%%%%%%%%%%%%%%%%%%%%%%%%%%%%%%%%%%%%%%%%%%%%%%%%%%%%%%%%%%%%%%%

%FOLLOWING IS OLD VERSION FOR HISTORY.

%In this chapter, the most popular production VMs and relevant research VMs are discussed. In further chapter, the thesis' proposed architecture will be compared against those VMs. 

%Most popular production VMs, such as Java hotspot (Cite) or Javascript's V8 (Cite) VMs are written in C++. Using C++ as a performance oriented low-level programming language proved to be very effective as it is possible to write code in a performance oriented fashion. A clear separation is made between the VM and the programming language run so there are no metacircular problems.

%Most of those VMs start-up from the language kernel, a set of core librairies and either source files or files containing bytecodes. Reaching peak performance takes a certain amount of time as the VM needs to detect and optimise correctly frequently used patterns of code. Reportedly, this warm-up time can be from several milliseconds up to multiple days. To solve partially the warm-up problem, such VMs are built with a tiered-architecture: the first few executions are run slowly but without any compilation time, subsequent hundreds of executing are run a bit faster with limited compilation time while further execution are run at peak performance after a certain amount of compilation time.

%An interesting point ot note is that several mainstream VMs were led by the same person (Lars Bak), who became very good at implementing very efficient and easy-to-maintain VMs in C++ as he implemented multiple of those in his life. His work therefore pushed the direction of VM implementation in the C++ direction.

%Among the C++ virtual machines, we will detail two specific VMs have uncommon features that are relevant in the context of the thesis. 

%\subsection{Azul}
%The Azul VM \cite{Azul} is a closed-source VM and expensive VM for Java. As for all closed-source projects, no one external to the project can be certain of what the code is doing. However, word has been that the Azul VM is able to persist optimised machine code across multiple start-ups. If the application is started on another processor, then the saved machine code is simply discarded. 

%\subsection{Dart}
%The Dart VM is an open-source VM for the Dart programming language. Dart features snapshots for fast application start-up. In Dart, the programmer can generate different kind of snapshots \cite{Anna13a}. Since that publication, the Dart team have added two new kind of snapshots, specialized for iOS and Android application deployment, which are quite similar to our snapshots.


%As the sista architecture is implemented in the context of Smalltalk, it could be relevant to discuss existing Smalltalk virtual machines. These VMs are interesting for multiple reasons, but unfortunately many of the Smalltalk virtual machines in production today are closed-source, making the discussion around them not that relevant as information is missing and unaccessible. In addition, as far as we know, there are no production Smalltalk VM today with an optimising JIT compiler, so the comparison with such VMs is even less relevant. 

%However, speculative optimisations in VMs started with the Self VM (CITE), with Self being a Smalltalk-like language, and was followed up with the strongtalk VM (CITE). Both VMs are open-source and available today but none of them are used in production. Self had never really broken through mainstream programming while strongtalk had never reached production state.

\ifx\wholebook\relax\else
    \end{document}
\fi