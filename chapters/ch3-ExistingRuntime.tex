\ifx\wholebook\relax\else

% --------------------------------------------
% Lulu:

    \documentclass[a4paper,12pt,twoside]{../includes/ThesisStyle}

	\usepackage[T1]{fontenc} %%%key to get copy and paste for the code!
%\usepackage[utf8]{inputenc} %%% to support copy and paste with accents for frnehc stuff
\usepackage{times}
\usepackage{ifthen}
\usepackage{xspace}
\usepackage{alltt}
\usepackage{latexsym}
\usepackage{url}            
\usepackage{amssymb}
\usepackage{amsfonts}
\usepackage{amsmath}
\usepackage{stmaryrd}
\usepackage{enumerate}
\usepackage{cite}
%\usepackage[pdftex,colorlinks=true,pdfstartview=FitV,linkcolor=blue,citecolor=blue,urlcolor=blue]{hyperref}
\usepackage{xspace}
%\usepackage{graphicx}
\usepackage{subfigure}
\usepackage[scaled=0.85]{helvet}
        
        
\newcommand{\sepe}{\mbox{>>}}
\newcommand{\pack}[1]{\emph{#1}}
\newcommand{\ozo}{\textsc{oZone}\xspace}
\newcommand\currentissues{\par\smallskip\textbf{Current Issues -- }}

\newboolean{showcomments}
\setboolean{showcomments}{true}
\ifthenelse{\boolean{showcomments}}
  {\newcommand{\bnote}[2]{
	\fbox{\bfseries\sffamily\scriptsize#1}
    {\sf\small$\blacktriangleright$\textit{#2}$\blacktriangleleft$}
    % \marginpar{\fbox{\bfseries\sffamily#1}}
   }
   \newcommand{\cvsversion}{\emph{\scriptsize$-$Id: macros.tex,v 1.1.1.1 2007/02/28 13:43:36 bergel Exp $-$}}
  }
  {\newcommand{\bnote}[2]{}
   \newcommand{\cvsversion}{}
  } 


\newcommand{\here}{\bnote{***}{CONTINUE HERE}}
\newcommand{\nb}[1]{\bnote{NB}{#1}}
\newcommand{\fix}[1]{\bnote{FIX}{#1}}
%%%% add your own macros 

\newcommand{\sd}[1]{\bnote{Stef}{#1}}
\newcommand{\ja}[1]{\bnote{Jannik}{#1}}
\newcommand{\na}[1]{\bnote{Nico}{#1}}
%%% 


\newcommand{\figref}[1]{Figure~\ref{fig:#1}}
\newcommand{\figlabel}[1]{\label{fig:#1}}
\newcommand{\tabref}[1]{Table~\ref{tab:#1}}
\newcommand{\layout}[1]{#1}
\newcommand{\commented}[1]{}
\newcommand{\secref}[1]{Section \ref{sec:#1}}
\newcommand{\seclabel}[1]{\label{sec:#1}}

%\newcommand{\ct}[1]{\textsf{#1}}
\newcommand{\stCode}[1]{\textsf{#1}}
\newcommand{\stMethod}[1]{\textsf{#1}}
\newcommand{\sep}{\texttt{>>}\xspace}
\newcommand{\stAssoc}{\texttt{->}\xspace}

\newcommand{\stBar}{$\mid$}
\newcommand{\stSelector}{$\gg$}
\newcommand{\ret}{\^{}}
\newcommand{\msup}{$>$}
%\newcommand{\ret}{$\uparrow$\xspace}

\newcommand{\myparagraph}[1]{\noindent\textbf{#1.}}
\newcommand{\eg}{\emph{e.g.,}\xspace}
\newcommand{\ie}{\emph{i.e.,}\xspace}
\newcommand{\ct}[1]{{\textsf{#1}}\xspace}


\newenvironment{code}
    {\begin{alltt}\sffamily}
    {\end{alltt}\normalsize}

\newcommand{\defaultScale}{0.55}
\newcommand{\pic}[3]{
   \begin{figure}[h]
   \begin{center}
   \includegraphics[scale=\defaultScale]{#1}
   \caption{#2}
   \label{#3}
   \end{center}
   \end{figure}
}

\newcommand{\twocolumnpic}[3]{
   \begin{figure*}[!ht]
   \begin{center}
   \includegraphics[scale=\defaultScale]{#1}
   \caption{#2}
   \label{#3}
   \end{center}
   \end{figure*}}

\newcommand{\infe}{$<$}
\newcommand{\supe}{$\rightarrow$\xspace}
\newcommand{\di}{$\gg$\xspace}
\newcommand{\adhoc}{\textit{ad-hoc}\xspace}

\usepackage{url}            
\makeatletter
\def\url@leostyle{%
  \@ifundefined{selectfont}{\def\UrlFont{\sf}}{\def\UrlFont{\small\sffamily}}}
\makeatother
% Now actually use the newly defined style.
\urlstyle{leo}



	\input{../includes/formatAndDefs}

	\graphicspath{{.}{../figures/}}
	\begin{document}
\fi

\chapter{Existing Pharo runtime}
\label{chap:existing}
\minitoc

This chapter describes Pharo and part of its implementation. Pharo is the high-level object-oriented programming language used to implement the thesis' architecture. As the architecture is built by reusing most of the existing implementation, this chapter details multiple features and implementation details that are not necessarily common in other programming langues to help the reader understanding the design decisions explained in further chapters. The chapter is not meant to explain the whole existing implementation, but only the relevant points for the thesis.

\section{Snapshots}

Snapshots are available in multiple object-oriented languages such as Smalltalk (CITE Goldberg (and) Robson 1983) and later Dart (CITE Annamalai 2013). Snapshots allow the program to save the heap~\footnote{The memory zone including the memory of all the objects in the system.} in a given state, and the virtual machine can resume execution from this snapshot later. 

During code execution, compiled code is available in different versions. On the one hand, a bytecoded version, which is on the heap if the bytecoded version of functions is reified as an object (as in Dart and Smalltalk). On the other hand one or several machine code versions are available in the machine code zone. Machine code versions are usually not part of the heap directly but of a separated part of memory which is marked as executable. 

Snapshots cannot save easily machine code versions of functions for two main reasons:
\begin{itemize}
	\item a snapshot needs to be platform-independent.
	\item machine code versions of functions are not usually represented in memory as regular objects.
\end{itemize}

In the Smalltalk normal development workflow, one persists the code one writes by taking snapshot of the current runtime, the snapshot including compiled code objects. A Smalltalk VM always uses a snapshot to start-up. Although one can argue that this workflow is not a good idea, the architecture was designed for Smalltalk developers and hence this usage of snapshot does not require any change in the programmer's workflow.

\section{Executable code and bytecode representation}

As everything is an object, executable code are objects too. From the language, reflective APIs are available to access the bytecode version of the compiled methods and closure (we call them compiled functions). A compiled function includes a function header, to describe specific things such as the required frame size, the number of temporary variables or the number of literals of the function, a list of literals, the bytecodes and a reference to the source code. 

(FIGURE CM)

The bytecode set is stack-based. This means that most operations are pushing and popping values of the stack. All the operations are untyped and work with any object. One of the main instruction is the virtual call instructions, popping the receiver and arguments from the stack and pushing back the result. The bytecode set also includes conditional and unconditional branches to encode condition and loops.

\section{Virtual machine executable generation}


\cite{Inga97a}

In this section is detailled the current process to generate the existing production virtual machine. 

Most of the execution engine (the memory manager, the interpreter and the baseline JIT) are written in Slang, a subset of Smalltalk. The Slang code can be translated to C, and then the C code can be compiled to machine code by standard C compilers.

The slang code has two main advantages over plain C:
\begin{itemize}
	\item \emph{Specifying inlining and code duplication:} To keep the interpreter code efficient, one has to be very careful on what code is inlined in the main interpreter loop and what code is not. In addition, for performance, specific code may need to be duplicated. For example, the interpreter code to push a temporary variable on stack is duplicated 17 times, the 16 first versions are dedicated versions for temporary numbers 0 to 15, the most common cases, more efficient because of constant usage, and the 17th version is the generic version. Slang allows to specify for each function if it needs to be inlined or duplicated udring Slang to C compilation.
	\item \emph{Simulation:} As Slang is a subset of Smalltalk, it can be executed as normal Smalltalk code. This is used to simulate the interpreter and garbage collector behavior. The JIT runtime is simulated using both Slang execution and external processor simulators. Simulation is very convenient to debug the VM, as all the Smalltalk debugging tools are available, the simulator can be saved and it is deterministic
\end{itemize}

\section{Stack frame reification}

The call stack is reified on demand as a linked list of context object. Each context object correspond to a stack frame. For the Smalltalk programmer, a context object is a normal object where all its fields (caller, program counter, temporaries, etc.) can be edited as all objects. The VM uses a complex proxy architecture to remap correctly the programmer's modification to the stack. Most transformation, such as editing the temporary variables, are easy to implement as there is a direct mapping to the stack frame. Other transformations are more complex, for instance, the mutation of a context object caller. If that operation happens, the VM split the stack in two at the mutation and transforms it so that each part of the stack is on a different stack pages. Returns across stack pages are handled specifically to keep the stack size of each process reasonnable.

mutation of pc.

The reification of the stack frames in used in three main places in Smalltalk:
\begin{enumerate}
	\item \emph{Debugging: } The debugging tools manipulates the stack and simulate program execution using this feature.
	\item \emph{Exceptions: } The exceptions are implemented entirely in Smalltalk, with no VM support, using this feature.
	\item \emph{Continuations: } Continuations are implemented entirely in Smalltalk, with no VM support, using this feature.
\end{enumerate}

\section{Baseline JIT}

The baseline JIT, called \emph{Cogit}, is used to compile bytecode
- baseline JIT behavior + reg alloc + templates + abstractions over back-ends, literal management, Cogits, memory managers
Doc: my blog posts. need to explain template and co

\section{Existing Memory Manager}
- the crap in V3 mem manager. Why it needed to change.
take the example of inline cache and class tags and memory representation.
Doc: my blog post. need to explain


+ become because used both in Mem manager paper and deopt solution 1.

\ifx\wholebook\relax\else
    \end{document}
\fi