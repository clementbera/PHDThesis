\ifx\wholebook\relax\else

% --------------------------------------------
% Lulu:

    \documentclass[a4paper,12pt,twoside]{../includes/ThesisStyle}

	\usepackage[T1]{fontenc} %%%key to get copy and paste for the code!
%\usepackage[utf8]{inputenc} %%% to support copy and paste with accents for frnehc stuff
\usepackage{times}
\usepackage{ifthen}
\usepackage{xspace}
\usepackage{alltt}
\usepackage{latexsym}
\usepackage{url}            
\usepackage{amssymb}
\usepackage{amsfonts}
\usepackage{amsmath}
\usepackage{stmaryrd}
\usepackage{enumerate}
\usepackage{cite}
%\usepackage[pdftex,colorlinks=true,pdfstartview=FitV,linkcolor=blue,citecolor=blue,urlcolor=blue]{hyperref}
\usepackage{xspace}
%\usepackage{graphicx}
\usepackage{subfigure}
\usepackage[scaled=0.85]{helvet}
        
        
\newcommand{\sepe}{\mbox{>>}}
\newcommand{\pack}[1]{\emph{#1}}
\newcommand{\ozo}{\textsc{oZone}\xspace}
\newcommand\currentissues{\par\smallskip\textbf{Current Issues -- }}

\newboolean{showcomments}
\setboolean{showcomments}{true}
\ifthenelse{\boolean{showcomments}}
  {\newcommand{\bnote}[2]{
	\fbox{\bfseries\sffamily\scriptsize#1}
    {\sf\small$\blacktriangleright$\textit{#2}$\blacktriangleleft$}
    % \marginpar{\fbox{\bfseries\sffamily#1}}
   }
   \newcommand{\cvsversion}{\emph{\scriptsize$-$Id: macros.tex,v 1.1.1.1 2007/02/28 13:43:36 bergel Exp $-$}}
  }
  {\newcommand{\bnote}[2]{}
   \newcommand{\cvsversion}{}
  } 


\newcommand{\here}{\bnote{***}{CONTINUE HERE}}
\newcommand{\nb}[1]{\bnote{NB}{#1}}
\newcommand{\fix}[1]{\bnote{FIX}{#1}}
%%%% add your own macros 

\newcommand{\sd}[1]{\bnote{Stef}{#1}}
\newcommand{\ja}[1]{\bnote{Jannik}{#1}}
\newcommand{\na}[1]{\bnote{Nico}{#1}}
%%% 


\newcommand{\figref}[1]{Figure~\ref{fig:#1}}
\newcommand{\figlabel}[1]{\label{fig:#1}}
\newcommand{\tabref}[1]{Table~\ref{tab:#1}}
\newcommand{\layout}[1]{#1}
\newcommand{\commented}[1]{}
\newcommand{\secref}[1]{Section \ref{sec:#1}}
\newcommand{\seclabel}[1]{\label{sec:#1}}

%\newcommand{\ct}[1]{\textsf{#1}}
\newcommand{\stCode}[1]{\textsf{#1}}
\newcommand{\stMethod}[1]{\textsf{#1}}
\newcommand{\sep}{\texttt{>>}\xspace}
\newcommand{\stAssoc}{\texttt{->}\xspace}

\newcommand{\stBar}{$\mid$}
\newcommand{\stSelector}{$\gg$}
\newcommand{\ret}{\^{}}
\newcommand{\msup}{$>$}
%\newcommand{\ret}{$\uparrow$\xspace}

\newcommand{\myparagraph}[1]{\noindent\textbf{#1.}}
\newcommand{\eg}{\emph{e.g.,}\xspace}
\newcommand{\ie}{\emph{i.e.,}\xspace}
\newcommand{\ct}[1]{{\textsf{#1}}\xspace}


\newenvironment{code}
    {\begin{alltt}\sffamily}
    {\end{alltt}\normalsize}

\newcommand{\defaultScale}{0.55}
\newcommand{\pic}[3]{
   \begin{figure}[h]
   \begin{center}
   \includegraphics[scale=\defaultScale]{#1}
   \caption{#2}
   \label{#3}
   \end{center}
   \end{figure}
}

\newcommand{\twocolumnpic}[3]{
   \begin{figure*}[!ht]
   \begin{center}
   \includegraphics[scale=\defaultScale]{#1}
   \caption{#2}
   \label{#3}
   \end{center}
   \end{figure*}}

\newcommand{\infe}{$<$}
\newcommand{\supe}{$\rightarrow$\xspace}
\newcommand{\di}{$\gg$\xspace}
\newcommand{\adhoc}{\textit{ad-hoc}\xspace}

\usepackage{url}            
\makeatletter
\def\url@leostyle{%
  \@ifundefined{selectfont}{\def\UrlFont{\sf}}{\def\UrlFont{\small\sffamily}}}
\makeatother
% Now actually use the newly defined style.
\urlstyle{leo}



	\usepackage{amsmath,amssymb}             % AMS Math
% \usepackage[french]{babel}
\usepackage[latin1]{inputenc}
\usepackage[T1]{fontenc}
\usepackage[left=1.5in,right=1.3in,top=1.1in,bottom=1.1in,includefoot,includehead,headheight=13.6pt]{geometry}
\renewcommand{\baselinestretch}{1.05}

\usepackage{multicol}

% Table of contents for each chapter

\usepackage[nottoc, notlof, notlot]{tocbibind}
\usepackage{minitoc}
\setcounter{minitocdepth}{1}
\mtcindent=15pt
% Use \minitoc where to put a table of contents

\usepackage{enumitem}

\usepackage{aecompl}

% Glossary / list of abbreviations

%\usepackage[intoc]{nomencl}
%\renewcommand{\nomname}{List of Abbreviations}
%
%\makenomenclature

% My pdf code

\usepackage[pdftex]{graphicx}
\usepackage[a4paper,pagebackref,hyperindex=true]{hyperref}

\usepackage{pgfplotstable,booktabs,colortbl}
\pgfplotsset{compat=1.8}

% Links in pdf
\usepackage{color}
\definecolor{linkcol}{rgb}{0,0,0.4} 
\definecolor{citecol}{rgb}{0.5,0,0} 

% Change this to change the informations included in the pdf file

% See hyperref documentation for information on those parameters

\hypersetup
{
bookmarksopen=true,
pdftitle="Sista: a Metacircular Architecture for Runtime Optimisation Persistence",
pdfauthor="Clement BERA", 
pdfsubject="Thesis", %subject of the document
%pdftoolbar=false, % toolbar hidden
pdfmenubar=true, %menubar shown
pdfhighlight=/O, %effect of clicking on a link
colorlinks=true, %couleurs sur les liens hypertextes
pdfpagemode=None, %aucun mode de page
pdfpagelayout=SinglePage, %ouverture en simple page
pdffitwindow=true, %pages ouvertes entierement dans toute la fenetre
linkcolor=linkcol, %couleur des liens hypertextes internes
citecolor=citecol, %couleur des liens pour les citations
urlcolor=linkcol %couleur des liens pour les url
}

% definitions.
% -------------------

\setcounter{secnumdepth}{3}
\setcounter{tocdepth}{1}

% Some useful commands and shortcut for maths:  partial derivative and stuff

\newcommand{\pd}[2]{\frac{\partial #1}{\partial #2}}
\def\abs{\operatorname{abs}}
\def\argmax{\operatornamewithlimits{arg\,max}}
\def\argmin{\operatornamewithlimits{arg\,min}}
\def\diag{\operatorname{Diag}}
\newcommand{\eqRef}[1]{(\ref{#1})}

\usepackage{rotating}                    % Sideways of figures & tables
%\usepackage{bibunits}
%\usepackage[sectionbib]{chapterbib}          % Cross-reference package (Natural BiB)
%\usepackage{natbib}                  % Put References at the end of each chapter
                                         % Do not put 'sectionbib' option here.
                                         % Sectionbib option in 'natbib' will do.
\usepackage{fancyhdr}                    % Fancy Header and Footer

% \usepackage{txfonts}                     % Public Times New Roman text & math font
  
%%% Fancy Header %%%%%%%%%%%%%%%%%%%%%%%%%%%%%%%%%%%%%%%%%%%%%%%%%%%%%%%%%%%%%%%%%%
% Fancy Header Style Options

\pagestyle{fancy}                       % Sets fancy header and footer
\fancyfoot{}                            % Delete current footer settings

%\renewcommand{\chaptermark}[1]{         % Lower Case Chapter marker style
%  \markboth{\chaptername\ \thechapter.\ #1}}{}} %

%\renewcommand{\sectionmark}[1]{         % Lower case Section marker style
%  \markright{\thesection.\ #1}}         %

\fancyhead[LE,RO]{\bfseries\thepage}    % Page number (boldface) in left on even
% pages and right on odd pages
\fancyhead[RE]{\bfseries\nouppercase{\leftmark}}      % Chapter in the right on even pages
\fancyhead[LO]{\bfseries\nouppercase{\rightmark}}     % Section in the left on odd pages

\let\headruleORIG\headrule
\renewcommand{\headrule}{\color{black} \headruleORIG}
\renewcommand{\headrulewidth}{1.0pt}
\usepackage{colortbl}
\arrayrulecolor{black}

\fancypagestyle{plain}{
  \fancyhead{}
  \fancyfoot{}
  \renewcommand{\headrulewidth}{0pt}
}

\usepackage{algorithm}
\usepackage[noend]{algorithmic}

%%% Clear Header %%%%%%%%%%%%%%%%%%%%%%%%%%%%%%%%%%%%%%%%%%%%%%%%%%%%%%%%%%%%%%%%%%
% Clear Header Style on the Last Empty Odd pages
\makeatletter

\def\cleardoublepage{\clearpage\if@twoside \ifodd\c@page\else%
  \hbox{}%
  \thispagestyle{empty}%              % Empty header styles
  \newpage%
  \if@twocolumn\hbox{}\newpage\fi\fi\fi}

\makeatother
 
%%%%%%%%%%%%%%%%%%%%%%%%%%%%%%%%%%%%%%%%%%%%%%%%%%%%%%%%%%%%%%%%%%%%%%%%%%%%%%% 
% Prints your review date and 'Draft Version' (From Josullvn, CS, CMU)
\newcommand{\reviewtimetoday}[2]{\special{!userdict begin
    /bop-hook{gsave 20 710 translate 45 rotate 0.8 setgray
      /Times-Roman findfont 12 scalefont setfont 0 0   moveto (#1) show
      0 -12 moveto (#2) show grestore}def end}}
% You can turn on or off this option.
% \reviewtimetoday{\today}{Draft Version}
%%%%%%%%%%%%%%%%%%%%%%%%%%%%%%%%%%%%%%%%%%%%%%%%%%%%%%%%%%%%%%%%%%%%%%%%%%%%%%% 

\newenvironment{maxime}[1]
{
\vspace*{0cm}
\hfill
\begin{minipage}{0.5\textwidth}%
%\rule[0.5ex]{\textwidth}{0.1mm}\\%
\hrulefill $\:$ {\bf #1}\\
%\vspace*{-0.25cm}
\it 
}%
{%

\hrulefill
\vspace*{0.5cm}%
\end{minipage}
}

\let\minitocORIG\minitoc
\renewcommand{\minitoc}{\minitocORIG \vspace{1.5em}}

\usepackage{multirow}
\usepackage{slashbox}

\newenvironment{bulletList}%
{ \begin{list}%
	{$\bullet$}%
	{\setlength{\labelwidth}{25pt}%
	 \setlength{\leftmargin}{30pt}%
	 \setlength{\itemsep}{\parsep}}}%
{ \end{list} }

\newtheorem{definition}{D�finition}
\renewcommand{\epsilon}{\varepsilon}

% centered page environment

\newenvironment{vcenterpage}
{\newpage\vspace*{\fill}\thispagestyle{empty}\renewcommand{\headrulewidth}{0pt}}
{\vspace*{\fill}}



	\graphicspath{{.}{../figures/}}
	\begin{document}
\fi

\chapter{Existing Pharo runtime}
\label{chap:existing}
\minitoc

%global intro
This chapter describes the Smalltalk dialect Pharo and part of its implementation. The Sista architecture was originally designed to improve the performance of the Pharo VM by adding speculative optimisations. Some existing features and implementation details already present in Pharo impacted several design decisions. They are detailled in this chapter to help the reader understanding the design decisions explained in further chapters. The chapter is not meant to explain the whole existing implementation, but only the most relevant points for the thesis. 

Pharo is an object-oriented language. Everything is an object, including classes or bytecoded versions of methods. It is dynamically-typed and every call is a virtual call. The virtual machine relies on a bytecode interpreter and a JIT to gain performance, similarly to Java virtual machines \cite{JavaVM8}. Modern Smalltalks directly inherit from Smalltalk-80~\cite{Gold83a} but have evolved during the past 35 years. For example, real closures and exceptions were added.

The chapter successively describes the Pharo programming language, the interface with its VM and some aspects of the VM implementation.

\section{Pharo programming language}

This section details three main aspects of the programming language: native thread management, stack frame reification and snapshots.

\paragraph{Native threads management}

Pharo features a global interpreter lock, similarly to python. Only calls to external libraries through the foreign function interface and specific virtual machine extensions have access to the other native threads. All the different virtual machine tasks, such as bytecode interpretation, machine code execution, just-in-time compilation or garbage collection are not done concurrently. This has a impact on design decisions because several other VMs implement the optimising JIT in concurrent native threads to the application (CITE). As Scorch is written in Pharo, the optimisation of v-functions and the application optimised are running in the same native thread.

\paragraph{Stack frame reification.}

The current VM evolved from the VM specified in the blue book~\cite{Gold83a}. The original specification relied on a spaghetti stack: the execution stack was represented as a linked list of v-function activations. Each v-function activation was represented as an object that could be read or written by the program. Over the years, Deutsch and Schiffman~\cite{Deut84a} changed the representation of the stack in the VM to use an execution stack similar to the one used in other programming languages, where stack frames are next to each other on a single stack. However, the VM still provides the ability to the programmer to read and write reified stack frames as if they where objects. To do so, each stack frame is reified as an object on demand. 

The reification of a stack frame abstracts away from the low-level details, a reified stack frame accessed from the language always looks like it is a v-function activation. A reified frame is exactly the same if the VM is started with the interpreter only or with the hybrid interpreter plus baseline JIT runtime. Conceptually, for the Smalltalk developer, Smalltalk code is interpreted and the reified frame always look like so. In fact, it is possible that some developers manipulate the stack without even knowing that there are n-frame or v-frame in practice in the VM.

The reification of the stack frames is used in three main places: the debugger, exceptions and continuations. For the two latter, they are implemented in Smalltalk on top of the stack frame reification, without any special VM support. In the Sista architecture, as Scorch is written in Pharo, it can use this feature to instrospect or modify the stack.

\paragraph{Snapshots.}

A snapshot~\footnote{Smalltalk developers use the term \emph{image} instead of snapshot.} is a sequence of bytes that represents a serialized form of all the objects present at a precise moment in the runtime. As everything is an object in Smalltalk, including green threads, the virtual machine can, at start-up, load all the objects from a snapshot and resume the execution based on the active green thread precised by the snapshot. In fact, this is the normal way of launching a Smalltalk runtime. 

One interesting problem in snapshots is how to save the execution stack, \ie the green threads. It possible in the existing VM to convert reify each stack frame into an object. To perform a snapshot, each stack frame is reified and only objects are saved in the snapshot. When the snapshot is restarted, the VM recreates a stack frame for each reified frame lazily. As discussed in the previous paragraph, all reified frames are in virtual state. In any case, snapshots cannot save n-frames because they are platform-independent. In the Pharo VM for example, a snapshot can be taken on a laptop using a x86 processor and restarted on a raspberry pie using a ARMv6 processor.

%%%%%%%%%%%%%%%%%%%%%%%%%%%%%%%%%%%%%%%%%%%%%%%%%%%%%%%%%%%%%%%%%%%%%%%%%%%%%%%%%%%%%%%%%%%%%%%%%%%%%%%%%%%%%%%%%%%%%%%%%%%%%%%%

%restart here.

\section{Language-VM interface}

small intro

\subsection{Executable code and bytecode representation}

As everything is an object, executable code are objects too. From the language, reflective APIs are available to access the bytecode version of the compiled methods and closure (we call them compiled functions). A compiled function includes a function header, to describe specific things such as the required frame size, the number of temporary variables or the number of literals of the function, a list of literals, the bytecodes and a reference to the source code. 

(FIGURE CM)

The bytecode set is stack-based. This means that most operations are pushing and popping values of the stack. All the operations are untyped and work with any object. One of the main instruction is the virtual call instructions, popping the receiver and arguments from the stack and pushing back the result. The bytecode set also includes conditional and unconditional branches to encode condition and loops.

\subsection{Registered objects}

discuss call-backs.

\subsection{Primitive methods}

I have written somewhere where sendAndBranchData is described what is a primitive.

+ become because discuss, explain both main use for instance migration and proxy

\section{Virtual machine}

overview

\subsection{Virtual machine executable generation}


\cite{Inga97a}

In this section is detailled the current process to generate the existing production virtual machine. 

Most of the execution engine (the memory manager, the interpreter and the baseline JIT) are written in Slang, a subset of Smalltalk. The Slang code can be translated to C, and then the C code can be compiled to machine code by standard C compilers.

The slang code has two main advantages over plain C:
\begin{itemize}
	\item \emph{Specifying inlining and code duplication:} To keep the interpreter code efficient, one has to be very careful on what code is inlined in the main interpreter loop and what code is not. In addition, for performance, specific code may need to be duplicated. For example, the interpreter code to push a temporary variable on stack is duplicated 17 times, the 16 first versions are dedicated versions for temporary numbers 0 to 15, the most common cases, more efficient because of constant usage, and the 17th version is the generic version. Slang allows to specify for each function if it needs to be inlined or duplicated udring Slang to C compilation.
	\item \emph{Simulation:} As Slang is a subset of Smalltalk, it can be executed as normal Smalltalk code. This is used to simulate the interpreter and garbage collector behavior. The JIT runtime is simulated using both Slang execution and external processor simulators. Simulation is very convenient to debug the VM, as all the Smalltalk debugging tools are available, the simulator can be saved and it is deterministic
\end{itemize}

\subsection{Baseline JIT}

The baseline JIT, called \emph{Cogit}, is used to compile bytecode
- baseline JIT behavior + simple reg alloc + templates + abstractions over and back-ends, memory managers
Doc: my blog posts. need to explain template and co

\subsection{Stack frame reification}

%Flag: FRAMEREIFICATION.

The VM is responsible for intercepting all accesses to a n-frame and map correctly the value to the v-frame value, and in some case, such as virtual instruction pointer modification, the VM converts the n-frame to a v-frame.


The reified object acts as a proxy to the stack frame for reads and simple write operations. Advanced operations, such as setting the caller of a stack frame, are done by abusing returns across stack pages. In rare cases, the context objects can be a full object and not just proxies to a stack frame, for example 
%when the programmer creates the object from the language using the \ct{thisContext} special variable or 
when the VM has no more stack pages available, it creates full context objects for all the stack frames used on the least recently used stack pages. Returning to such context objects can be done only with a return across stack pages, and the VM recreates a stack frame for the context object to be able to resume execution.

The call stack is reified on demand as a linked list of context object. Each context object correspond to a stack frame. For the Smalltalk programmer, a context object is a normal object where all its fields (caller, program counter, temporaries, etc.) can be edited as all objects. The VM uses a complex proxy architecture to remap correctly the programmer's modification to the stack. Most transformation, such as editing the temporary variables, are easy to implement as there is a direct mapping to the stack frame. Other transformations are more complex, for instance, the mutation of a context object caller. If that operation happens, the VM split the stack in two at the mutation and transforms it so that each part of the stack is on a different stack pages. Returns across stack pages are handled specifically to keep the stack size of each process reasonnable.

mutation of pc.

can crash (responsibility of the user)



Discuss become in addition
+ become because used both in Mem manager paper and deopt solution 1.

\ifx\wholebook\relax\else
    \end{document}
\fi