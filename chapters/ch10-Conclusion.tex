\ifx\wholebook\relax\else

% --------------------------------------------
% Lulu:

    \documentclass[a4paper,12pt,twoside]{../includes/ThesisStyle}

	\usepackage[T1]{fontenc} %%%key to get copy and paste for the code!
%\usepackage[utf8]{inputenc} %%% to support copy and paste with accents for frnehc stuff
\usepackage{times}
\usepackage{ifthen}
\usepackage{xspace}
\usepackage{alltt}
\usepackage{latexsym}
\usepackage{url}            
\usepackage{amssymb}
\usepackage{amsfonts}
\usepackage{amsmath}
\usepackage{stmaryrd}
\usepackage{enumerate}
\usepackage{cite}
%\usepackage[pdftex,colorlinks=true,pdfstartview=FitV,linkcolor=blue,citecolor=blue,urlcolor=blue]{hyperref}
\usepackage{xspace}
%\usepackage{graphicx}
\usepackage{subfigure}
\usepackage[scaled=0.85]{helvet}
        
        
\newcommand{\sepe}{\mbox{>>}}
\newcommand{\pack}[1]{\emph{#1}}
\newcommand{\ozo}{\textsc{oZone}\xspace}
\newcommand\currentissues{\par\smallskip\textbf{Current Issues -- }}

\newboolean{showcomments}
\setboolean{showcomments}{true}
\ifthenelse{\boolean{showcomments}}
  {\newcommand{\bnote}[2]{
	\fbox{\bfseries\sffamily\scriptsize#1}
    {\sf\small$\blacktriangleright$\textit{#2}$\blacktriangleleft$}
    % \marginpar{\fbox{\bfseries\sffamily#1}}
   }
   \newcommand{\cvsversion}{\emph{\scriptsize$-$Id: macros.tex,v 1.1.1.1 2007/02/28 13:43:36 bergel Exp $-$}}
  }
  {\newcommand{\bnote}[2]{}
   \newcommand{\cvsversion}{}
  } 


\newcommand{\here}{\bnote{***}{CONTINUE HERE}}
\newcommand{\nb}[1]{\bnote{NB}{#1}}
\newcommand{\fix}[1]{\bnote{FIX}{#1}}
%%%% add your own macros 

\newcommand{\sd}[1]{\bnote{Stef}{#1}}
\newcommand{\ja}[1]{\bnote{Jannik}{#1}}
\newcommand{\na}[1]{\bnote{Nico}{#1}}
%%% 


\newcommand{\figref}[1]{Figure~\ref{fig:#1}}
\newcommand{\figlabel}[1]{\label{fig:#1}}
\newcommand{\tabref}[1]{Table~\ref{tab:#1}}
\newcommand{\layout}[1]{#1}
\newcommand{\commented}[1]{}
\newcommand{\secref}[1]{Section \ref{sec:#1}}
\newcommand{\seclabel}[1]{\label{sec:#1}}

%\newcommand{\ct}[1]{\textsf{#1}}
\newcommand{\stCode}[1]{\textsf{#1}}
\newcommand{\stMethod}[1]{\textsf{#1}}
\newcommand{\sep}{\texttt{>>}\xspace}
\newcommand{\stAssoc}{\texttt{->}\xspace}

\newcommand{\stBar}{$\mid$}
\newcommand{\stSelector}{$\gg$}
\newcommand{\ret}{\^{}}
\newcommand{\msup}{$>$}
%\newcommand{\ret}{$\uparrow$\xspace}

\newcommand{\myparagraph}[1]{\noindent\textbf{#1.}}
\newcommand{\eg}{\emph{e.g.,}\xspace}
\newcommand{\ie}{\emph{i.e.,}\xspace}
\newcommand{\ct}[1]{{\textsf{#1}}\xspace}


\newenvironment{code}
    {\begin{alltt}\sffamily}
    {\end{alltt}\normalsize}

\newcommand{\defaultScale}{0.55}
\newcommand{\pic}[3]{
   \begin{figure}[h]
   \begin{center}
   \includegraphics[scale=\defaultScale]{#1}
   \caption{#2}
   \label{#3}
   \end{center}
   \end{figure}
}

\newcommand{\twocolumnpic}[3]{
   \begin{figure*}[!ht]
   \begin{center}
   \includegraphics[scale=\defaultScale]{#1}
   \caption{#2}
   \label{#3}
   \end{center}
   \end{figure*}}

\newcommand{\infe}{$<$}
\newcommand{\supe}{$\rightarrow$\xspace}
\newcommand{\di}{$\gg$\xspace}
\newcommand{\adhoc}{\textit{ad-hoc}\xspace}

\usepackage{url}            
\makeatletter
\def\url@leostyle{%
  \@ifundefined{selectfont}{\def\UrlFont{\sf}}{\def\UrlFont{\small\sffamily}}}
\makeatother
% Now actually use the newly defined style.
\urlstyle{leo}



	\usepackage{amsmath,amssymb}             % AMS Math
% \usepackage[french]{babel}
\usepackage[latin1]{inputenc}
\usepackage[T1]{fontenc}
\usepackage[left=1.5in,right=1.3in,top=1.1in,bottom=1.1in,includefoot,includehead,headheight=13.6pt]{geometry}
\renewcommand{\baselinestretch}{1.05}

\usepackage{multicol}

% Table of contents for each chapter

\usepackage[nottoc, notlof, notlot]{tocbibind}
\usepackage{minitoc}
\setcounter{minitocdepth}{1}
\mtcindent=15pt
% Use \minitoc where to put a table of contents

\usepackage{enumitem}

\usepackage{aecompl}

% Glossary / list of abbreviations

%\usepackage[intoc]{nomencl}
%\renewcommand{\nomname}{List of Abbreviations}
%
%\makenomenclature

% My pdf code

\usepackage[pdftex]{graphicx}
\usepackage[a4paper,pagebackref,hyperindex=true]{hyperref}

\usepackage{pgfplotstable,booktabs,colortbl}
\pgfplotsset{compat=1.8}

% Links in pdf
\usepackage{color}
\definecolor{linkcol}{rgb}{0,0,0.4} 
\definecolor{citecol}{rgb}{0.5,0,0} 

% Change this to change the informations included in the pdf file

% See hyperref documentation for information on those parameters

\hypersetup
{
bookmarksopen=true,
pdftitle="Sista: a Metacircular Architecture for Runtime Optimisation Persistence",
pdfauthor="Clement BERA", 
pdfsubject="Thesis", %subject of the document
%pdftoolbar=false, % toolbar hidden
pdfmenubar=true, %menubar shown
pdfhighlight=/O, %effect of clicking on a link
colorlinks=true, %couleurs sur les liens hypertextes
pdfpagemode=None, %aucun mode de page
pdfpagelayout=SinglePage, %ouverture en simple page
pdffitwindow=true, %pages ouvertes entierement dans toute la fenetre
linkcolor=linkcol, %couleur des liens hypertextes internes
citecolor=citecol, %couleur des liens pour les citations
urlcolor=linkcol %couleur des liens pour les url
}

% definitions.
% -------------------

\setcounter{secnumdepth}{3}
\setcounter{tocdepth}{1}

% Some useful commands and shortcut for maths:  partial derivative and stuff

\newcommand{\pd}[2]{\frac{\partial #1}{\partial #2}}
\def\abs{\operatorname{abs}}
\def\argmax{\operatornamewithlimits{arg\,max}}
\def\argmin{\operatornamewithlimits{arg\,min}}
\def\diag{\operatorname{Diag}}
\newcommand{\eqRef}[1]{(\ref{#1})}

\usepackage{rotating}                    % Sideways of figures & tables
%\usepackage{bibunits}
%\usepackage[sectionbib]{chapterbib}          % Cross-reference package (Natural BiB)
%\usepackage{natbib}                  % Put References at the end of each chapter
                                         % Do not put 'sectionbib' option here.
                                         % Sectionbib option in 'natbib' will do.
\usepackage{fancyhdr}                    % Fancy Header and Footer

% \usepackage{txfonts}                     % Public Times New Roman text & math font
  
%%% Fancy Header %%%%%%%%%%%%%%%%%%%%%%%%%%%%%%%%%%%%%%%%%%%%%%%%%%%%%%%%%%%%%%%%%%
% Fancy Header Style Options

\pagestyle{fancy}                       % Sets fancy header and footer
\fancyfoot{}                            % Delete current footer settings

%\renewcommand{\chaptermark}[1]{         % Lower Case Chapter marker style
%  \markboth{\chaptername\ \thechapter.\ #1}}{}} %

%\renewcommand{\sectionmark}[1]{         % Lower case Section marker style
%  \markright{\thesection.\ #1}}         %

\fancyhead[LE,RO]{\bfseries\thepage}    % Page number (boldface) in left on even
% pages and right on odd pages
\fancyhead[RE]{\bfseries\nouppercase{\leftmark}}      % Chapter in the right on even pages
\fancyhead[LO]{\bfseries\nouppercase{\rightmark}}     % Section in the left on odd pages

\let\headruleORIG\headrule
\renewcommand{\headrule}{\color{black} \headruleORIG}
\renewcommand{\headrulewidth}{1.0pt}
\usepackage{colortbl}
\arrayrulecolor{black}

\fancypagestyle{plain}{
  \fancyhead{}
  \fancyfoot{}
  \renewcommand{\headrulewidth}{0pt}
}

\usepackage{algorithm}
\usepackage[noend]{algorithmic}

%%% Clear Header %%%%%%%%%%%%%%%%%%%%%%%%%%%%%%%%%%%%%%%%%%%%%%%%%%%%%%%%%%%%%%%%%%
% Clear Header Style on the Last Empty Odd pages
\makeatletter

\def\cleardoublepage{\clearpage\if@twoside \ifodd\c@page\else%
  \hbox{}%
  \thispagestyle{empty}%              % Empty header styles
  \newpage%
  \if@twocolumn\hbox{}\newpage\fi\fi\fi}

\makeatother
 
%%%%%%%%%%%%%%%%%%%%%%%%%%%%%%%%%%%%%%%%%%%%%%%%%%%%%%%%%%%%%%%%%%%%%%%%%%%%%%% 
% Prints your review date and 'Draft Version' (From Josullvn, CS, CMU)
\newcommand{\reviewtimetoday}[2]{\special{!userdict begin
    /bop-hook{gsave 20 710 translate 45 rotate 0.8 setgray
      /Times-Roman findfont 12 scalefont setfont 0 0   moveto (#1) show
      0 -12 moveto (#2) show grestore}def end}}
% You can turn on or off this option.
% \reviewtimetoday{\today}{Draft Version}
%%%%%%%%%%%%%%%%%%%%%%%%%%%%%%%%%%%%%%%%%%%%%%%%%%%%%%%%%%%%%%%%%%%%%%%%%%%%%%% 

\newenvironment{maxime}[1]
{
\vspace*{0cm}
\hfill
\begin{minipage}{0.5\textwidth}%
%\rule[0.5ex]{\textwidth}{0.1mm}\\%
\hrulefill $\:$ {\bf #1}\\
%\vspace*{-0.25cm}
\it 
}%
{%

\hrulefill
\vspace*{0.5cm}%
\end{minipage}
}

\let\minitocORIG\minitoc
\renewcommand{\minitoc}{\minitocORIG \vspace{1.5em}}

\usepackage{multirow}
\usepackage{slashbox}

\newenvironment{bulletList}%
{ \begin{list}%
	{$\bullet$}%
	{\setlength{\labelwidth}{25pt}%
	 \setlength{\leftmargin}{30pt}%
	 \setlength{\itemsep}{\parsep}}}%
{ \end{list} }

\newtheorem{definition}{D�finition}
\renewcommand{\epsilon}{\varepsilon}

% centered page environment

\newenvironment{vcenterpage}
{\newpage\vspace*{\fill}\thispagestyle{empty}\renewcommand{\headrulewidth}{0pt}}
{\vspace*{\fill}}



	\graphicspath{{.}{../figures/}}
	\begin{document}
\fi

\chapter{Conclusion}
\label{chap:conclusion}
\minitoc

In this chapter we summarise the dissertation, then list the contributions and lastly discuss the impact of the thesis.

\section{Summary}

In the thesis we focused on three problems. First, we investigated how to build an optimising JIT architecture where the optimising JIT is running in the same runtime than the running application on top of an existing VM composed of an interpreter and a baseline JIT. The result is Sista, our optimising JIT architecture. Second, we studied metacircular problems related to an optimising JIT that can optimise its own code. As a result, Scorch is able to optimise its own code with metacircular problems under specific circumstances. Third, we analysed the persistence of the runtime state, including runtime optimisations, across multiple VM start-ups in the context of the warm-up time problem. As a result, Sista is able to avoid most of the warm-up time by persisting optimised v-functions across VM start-ups.

%hspace to simulate non working indent.
\textbf{Chapter \ref{chap:stateOfTheArt}} defined the terminology used in the thesis. The chapter then described the two most common architectures for optimising JITs, the function-based and the meta-tracing architecture, with examples of VMs using one architecture or the other. Then, work related to Sista in the context of metacircular VMs and the persistence of the runtime state across start-ups were listed.

\textbf{Chapter \ref{chap:existing}} detailed the existing Pharo runtime. The focus was on the features not present in most other VMs or relevant in the context of the dissertation. Sista is built on top of the existing runtime described in this chapter.

\textbf{Chapter \ref{chap:architecture}} explained Sista, including the inner details of the architecture. Especially, the split between Scorch and Cogit in the optimising JIT architecture was described. The chapter compared Sista against existing architectures for optimising JITs.

\textbf{Chapter \ref{chap:runtimeEvolution}} discussed the evolutions required to the existing Pharo runtime to implement Sista on top of the existing runtime. This included all the core aspects of the system which needed to be changed to support Sista, such as the implementation of closures or the ability to mark an object as read-only.

\textbf{Chapter \ref{chap:metacircular}} was articulated around a problem happening in metacircular optimising JIT such as Scorch: under specific circumstances, Scorch ends up in an infinite recursion where it calls itself repeatedly. The problem can happen both when Scorch attempts to optimise one of its own function or when it attempts to deoptimise one of its own frame. The problem is solved by disabling temporarily the optimiser when relevant and by making the deoptimisater code completely independent from the rest of Pharo.

\textbf{Chapter \ref{chap:persistence}} detailed how the runtime state is persisted across multiple start-ups in the context of Sista, including optimised v-functions, allowing one to decrease warm-up time. 

\textbf{Chapter \ref{chap:validation}} showed benchmarks run on the existing Pharo runtime and the Sista runtime. The Sista VM is up to 5x faster at peak performance than the current production runtime and peak performance is reached almost immediately after start-up if optimised v-functions are persisted as part of the snapshot.

\textbf{Chapter \ref{chap:futureWork}} discussed future work that are worth investigating based on the contents of this thesis.

\section{Contributions}

The main contributions of this thesis are:
\begin{itemize}
	\item The implementation of Sista, which yields 1.5x to 5x speed-up in execution time compared to the existing Pharo runtime and allow one to reach peak performance very quickly if optimised v-functions are persisted from previous runs of the VM.
	\item An alternative bytecode set solving multiple existing encoding limitations.
	\item A language extension: each object can now be marked as read-only.
	\item An alternative implementation of closures, both allowing simplifications in existing code and enabling new optimisation possibilities.
\end{itemize}

\section{Impact of the thesis}

Multiple contributions of the thesis (the alternative bytecode set, the read-only objects and the alternative implementation of closures) are integrated in Pharo and got some attention from the user-base. The main impact is Sista, which is currently being integrated in Pharo and will hopefully enable new research and industrial work.

\ifx\wholebook\relax\else
    \end{document}
\fi