\documentclass[a4paper,12pt,twoside]{includes/ThesisStyle}

\usepackage[T1]{fontenc} %%%key to get copy and paste for the code!
%\usepackage[utf8]{inputenc} %%% to support copy and paste with accents for frnehc stuff
\usepackage{times}
\usepackage{ifthen}
\usepackage{xspace}
\usepackage{alltt}
\usepackage{latexsym}
\usepackage{url}            
\usepackage{amssymb}
\usepackage{amsfonts}
\usepackage{amsmath}
\usepackage{stmaryrd}
\usepackage{enumerate}
\usepackage{cite}
%\usepackage[pdftex,colorlinks=true,pdfstartview=FitV,linkcolor=blue,citecolor=blue,urlcolor=blue]{hyperref}
\usepackage{xspace}
%\usepackage{graphicx}
\usepackage{subfigure}
\usepackage[scaled=0.85]{helvet}
        
        
\newcommand{\sepe}{\mbox{>>}}
\newcommand{\pack}[1]{\emph{#1}}
\newcommand{\ozo}{\textsc{oZone}\xspace}
\newcommand\currentissues{\par\smallskip\textbf{Current Issues -- }}

\newboolean{showcomments}
\setboolean{showcomments}{true}
\ifthenelse{\boolean{showcomments}}
  {\newcommand{\bnote}[2]{
	\fbox{\bfseries\sffamily\scriptsize#1}
    {\sf\small$\blacktriangleright$\textit{#2}$\blacktriangleleft$}
    % \marginpar{\fbox{\bfseries\sffamily#1}}
   }
   \newcommand{\cvsversion}{\emph{\scriptsize$-$Id: macros.tex,v 1.1.1.1 2007/02/28 13:43:36 bergel Exp $-$}}
  }
  {\newcommand{\bnote}[2]{}
   \newcommand{\cvsversion}{}
  } 


\newcommand{\here}{\bnote{***}{CONTINUE HERE}}
\newcommand{\nb}[1]{\bnote{NB}{#1}}
\newcommand{\fix}[1]{\bnote{FIX}{#1}}
%%%% add your own macros 

\newcommand{\sd}[1]{\bnote{Stef}{#1}}
\newcommand{\ja}[1]{\bnote{Jannik}{#1}}
\newcommand{\na}[1]{\bnote{Nico}{#1}}
%%% 


\newcommand{\figref}[1]{Figure~\ref{fig:#1}}
\newcommand{\figlabel}[1]{\label{fig:#1}}
\newcommand{\tabref}[1]{Table~\ref{tab:#1}}
\newcommand{\layout}[1]{#1}
\newcommand{\commented}[1]{}
\newcommand{\secref}[1]{Section \ref{sec:#1}}
\newcommand{\seclabel}[1]{\label{sec:#1}}

%\newcommand{\ct}[1]{\textsf{#1}}
\newcommand{\stCode}[1]{\textsf{#1}}
\newcommand{\stMethod}[1]{\textsf{#1}}
\newcommand{\sep}{\texttt{>>}\xspace}
\newcommand{\stAssoc}{\texttt{->}\xspace}

\newcommand{\stBar}{$\mid$}
\newcommand{\stSelector}{$\gg$}
\newcommand{\ret}{\^{}}
\newcommand{\msup}{$>$}
%\newcommand{\ret}{$\uparrow$\xspace}

\newcommand{\myparagraph}[1]{\noindent\textbf{#1.}}
\newcommand{\eg}{\emph{e.g.,}\xspace}
\newcommand{\ie}{\emph{i.e.,}\xspace}
\newcommand{\ct}[1]{{\textsf{#1}}\xspace}


\newenvironment{code}
    {\begin{alltt}\sffamily}
    {\end{alltt}\normalsize}

\newcommand{\defaultScale}{0.55}
\newcommand{\pic}[3]{
   \begin{figure}[h]
   \begin{center}
   \includegraphics[scale=\defaultScale]{#1}
   \caption{#2}
   \label{#3}
   \end{center}
   \end{figure}
}

\newcommand{\twocolumnpic}[3]{
   \begin{figure*}[!ht]
   \begin{center}
   \includegraphics[scale=\defaultScale]{#1}
   \caption{#2}
   \label{#3}
   \end{center}
   \end{figure*}}

\newcommand{\infe}{$<$}
\newcommand{\supe}{$\rightarrow$\xspace}
\newcommand{\di}{$\gg$\xspace}
\newcommand{\adhoc}{\textit{ad-hoc}\xspace}

\usepackage{url}            
\makeatletter
\def\url@leostyle{%
  \@ifundefined{selectfont}{\def\UrlFont{\sf}}{\def\UrlFont{\small\sffamily}}}
\makeatother
% Now actually use the newly defined style.
\urlstyle{leo}



\include{includes/formatAndDefs}

\graphicspath{{.}{figures/}}
\newcommand \logoInria{./includes/logos/Inria}
\newcommand \logoRegion{./includes/logos/Region}
\newcommand \logoLifl{./includes/logos/lifl}
\newcommand \logoUSTL{./includes/logos/Lille1}

\let\wholebook=\relax

\begin{document}
\include{chapters/titlePage}
\dominitoc

\pagenumbering{roman}

\cleardoublepage

\section*{Acknowledgments}

I would like to thank my thesis supervisors St\'ephane Ducasse and Marcus Denker for allowing me to do a Ph.D at the RMoD group, as well as helping and supporting me during the three years of my Ph.D.

I thank the thesis reviewers and jury members Ga\"el Thomas, Laurence Tratt and Laurence Duchien for kindly reviewing my thesis and providing me valuable feedback.

I would like to express my gratitude to Eliot Miranda for his first design of the Sista architecture and his support during the three years of my Ph.D. 

I would like to thank Tim Felgentreff for his evaluation of Sista architecture using the Squeak speed center.

For remarks on earlier versions of this thesis I thank (To confirm if they actually do it) Guillermo Polito and Damien Cassou.

\cleardoublepage

\section*{Abstract}

Most high-level programming languages run on top of a virtual machine (VM) to abstract away from the underlying hardware. To reach high-performance, the VM typically relies on an optimising just-in-time compiler (JIT), which speculates on the program behavior based on its first runs to generate at runtime efficient machine code and speed-up the program execution. As multiple runs are required to speculate correctly on the program behavior, such a VM requires a certain amount of time at start-up to reach peak performance. The optimising JIT itself is usually compiled ahead-of-time to executable code as part of the VM.

The dissertation proposes an architecture for an optimising JIT, in which the optimised state of the VM can be persisted across multiple start-ups and the optimising JIT is running in the same runtime than the program executed. To do so, the optimising JIT is split in two parts. One part is high-level: it performs optimisations specific to the programming language run by the VM and is written in a metacircular style. Staying away from low-level details, this part can be read, edited and debugged while the program is running using the standard tool set of the programming language executed by the VM. The second part is low-level: it performs machine specific optimisations and is compiled ahead-of-time to executable code as part of the virtual machine. The two parts of the JIT use a well-defined intermediate representation to share the code to optimise. This representation is machine-independent and can be persisted across multiple start-ups, allowing the virtual machine to reach peak performance very quickly.

To validate the architecture, the dissertation includes the description of an implementation on top of Pharo Smalltalk and its virtual machine. The implementation is able to run a large set of benchmarks, from large application benchmarks provided by industrial users to micro-benchmarks used to measure the performance of specific code patterns. The optimising JIT is implemented according to the architecture proposed and shows significant speed-up (1.5x to 5x) over the current production virtual machine. In addition, large benchmarks show that peak performance can be reached almost immediately after start-up if the VM can reuse the optimised state persisted from another run.

\cleardoublepage

\section*{R\'esum\'e}

\tableofcontents
\listoffigures
%\listoftables

\mainmatter

\ifx\wholebook\relax\else

% --------------------------------------------
% Lulu:

    \documentclass[a4paper,12pt,twoside]{../includes/ThesisStyle}

	\usepackage[T1]{fontenc} %%%key to get copy and paste for the code!
%\usepackage[utf8]{inputenc} %%% to support copy and paste with accents for frnehc stuff
\usepackage{times}
\usepackage{ifthen}
\usepackage{xspace}
\usepackage{alltt}
\usepackage{latexsym}
\usepackage{url}            
\usepackage{amssymb}
\usepackage{amsfonts}
\usepackage{amsmath}
\usepackage{stmaryrd}
\usepackage{enumerate}
\usepackage{cite}
%\usepackage[pdftex,colorlinks=true,pdfstartview=FitV,linkcolor=blue,citecolor=blue,urlcolor=blue]{hyperref}
\usepackage{xspace}
%\usepackage{graphicx}
\usepackage{subfigure}
\usepackage[scaled=0.85]{helvet}
        
        
\newcommand{\sepe}{\mbox{>>}}
\newcommand{\pack}[1]{\emph{#1}}
\newcommand{\ozo}{\textsc{oZone}\xspace}
\newcommand\currentissues{\par\smallskip\textbf{Current Issues -- }}

\newboolean{showcomments}
\setboolean{showcomments}{true}
\ifthenelse{\boolean{showcomments}}
  {\newcommand{\bnote}[2]{
	\fbox{\bfseries\sffamily\scriptsize#1}
    {\sf\small$\blacktriangleright$\textit{#2}$\blacktriangleleft$}
    % \marginpar{\fbox{\bfseries\sffamily#1}}
   }
   \newcommand{\cvsversion}{\emph{\scriptsize$-$Id: macros.tex,v 1.1.1.1 2007/02/28 13:43:36 bergel Exp $-$}}
  }
  {\newcommand{\bnote}[2]{}
   \newcommand{\cvsversion}{}
  } 


\newcommand{\here}{\bnote{***}{CONTINUE HERE}}
\newcommand{\nb}[1]{\bnote{NB}{#1}}
\newcommand{\fix}[1]{\bnote{FIX}{#1}}
%%%% add your own macros 

\newcommand{\sd}[1]{\bnote{Stef}{#1}}
\newcommand{\ja}[1]{\bnote{Jannik}{#1}}
\newcommand{\na}[1]{\bnote{Nico}{#1}}
%%% 


\newcommand{\figref}[1]{Figure~\ref{fig:#1}}
\newcommand{\figlabel}[1]{\label{fig:#1}}
\newcommand{\tabref}[1]{Table~\ref{tab:#1}}
\newcommand{\layout}[1]{#1}
\newcommand{\commented}[1]{}
\newcommand{\secref}[1]{Section \ref{sec:#1}}
\newcommand{\seclabel}[1]{\label{sec:#1}}

%\newcommand{\ct}[1]{\textsf{#1}}
\newcommand{\stCode}[1]{\textsf{#1}}
\newcommand{\stMethod}[1]{\textsf{#1}}
\newcommand{\sep}{\texttt{>>}\xspace}
\newcommand{\stAssoc}{\texttt{->}\xspace}

\newcommand{\stBar}{$\mid$}
\newcommand{\stSelector}{$\gg$}
\newcommand{\ret}{\^{}}
\newcommand{\msup}{$>$}
%\newcommand{\ret}{$\uparrow$\xspace}

\newcommand{\myparagraph}[1]{\noindent\textbf{#1.}}
\newcommand{\eg}{\emph{e.g.,}\xspace}
\newcommand{\ie}{\emph{i.e.,}\xspace}
\newcommand{\ct}[1]{{\textsf{#1}}\xspace}


\newenvironment{code}
    {\begin{alltt}\sffamily}
    {\end{alltt}\normalsize}

\newcommand{\defaultScale}{0.55}
\newcommand{\pic}[3]{
   \begin{figure}[h]
   \begin{center}
   \includegraphics[scale=\defaultScale]{#1}
   \caption{#2}
   \label{#3}
   \end{center}
   \end{figure}
}

\newcommand{\twocolumnpic}[3]{
   \begin{figure*}[!ht]
   \begin{center}
   \includegraphics[scale=\defaultScale]{#1}
   \caption{#2}
   \label{#3}
   \end{center}
   \end{figure*}}

\newcommand{\infe}{$<$}
\newcommand{\supe}{$\rightarrow$\xspace}
\newcommand{\di}{$\gg$\xspace}
\newcommand{\adhoc}{\textit{ad-hoc}\xspace}

\usepackage{url}            
\makeatletter
\def\url@leostyle{%
  \@ifundefined{selectfont}{\def\UrlFont{\sf}}{\def\UrlFont{\small\sffamily}}}
\makeatother
% Now actually use the newly defined style.
\urlstyle{leo}



	\input{../includes/formatAndDefs}

	\graphicspath{{.}{../figures/}}
	\begin{document}
\fi

\chapter{Introduction}
\label{chap:intro}
\minitoc

\section{Context}

NOTE we need to disambiguate with OS VMs.

\subsection{Virtual machines for high-level programming languages}

Many high-level object-oriented programming languages run on top of a virtual machine (VM) which provides certain advantages from running directly on the underlying hardware. Many of these programming languages pursue a strict separation between language-side and VM-side. VMs for instance provide automatic memory management or use platform agnostic instructions such as bytecodes. These properties allow a programming language to develop independently from the underlying hardware.

High performance VMs, such as Java HotSpot or current Javascript VMs achieve high performance through just-in-time compilation techniques: once the VM has detected that a portion of code is frequently used, it recompiles it on-the-fly with speculative optimizations based on previous runs of the code. If usage patterns change and the code is not executed as previously speculated anymore, the VM dynamically deoptimizes the execution stack and resumes execution with the unoptimized code.

Originally VMs were built in performance oriented low-level programming languages such as C. However, as the VMs were reaching higher and higher performance, the complexity of their code base increased and some VMs started to get written in higher-level languages as an attempt to ease develpment. Such VMs got written either in the language run by the VM itself or in specific DSLs.

\subsection{Pharo programming language}

In this thesis the focus is on a specific high-level object-oriented programming language, the Smalltalk dialect named Pharo. In Pharo, everything is an object, including classes, bytecoded versions of methods or processes. It is dynamically-typed and every call is a virtual call. The VM relies on a bytecode interpreter and a baseline JIT to gain performance. Modern Smalltalk dialects directly inherit from Smalltalk-80 specified in CITE (Goldberg  Robson 1983) but have evolved during the past 35 years. For example, real closures and exceptions were added.

As Pharo is evolving, the community is looking for better VM performance. Compared to many high performance VMs, the Pharo VM is lacking an optimising JIT with speculative optimisations. However, in high performance VMs, the optimising JIT is one of the most complex part, if not the most complex and the Pharo community has just enough ressources to fund the maintenance and evolution of the existing VM. Hence, the optimising JIT design has to be done under two main constraints:
\begin{itemize}
\item The maintenance of the resulting VM has to remain affordable
\item The resulting VM should be built on top of the existing runtime
\end{itemize}

A first design emerged in the early 2000s according to those constraints. The main ideas were:
\begin{itemize}
	\item To build the optimising JIT in the Pharo runtime itself. The existing VM is written in a DSL compiling to native code through C. As the DSL semantics are very close to C and that most people in the Smalltalk community are either not familiar with C or more productive with Smalltalk than C, this design was seen as a way to reduce the maintenance cost.
	\item To build the optimising JIT as a bytecode to bytecode optimiser and reuse the baseline JIT as a back-end of the optimising JIT to produce efficent machine code. This design would avoid implementing and maintaining two native code back-ends while reusing and extending the existing language-VM interface with the bytecodes, overall lowering the development cost.
\end{itemize}

The design looked interesting but was very abstract so it was unclear how multiple part of the system would be implemented. The thesis started from this proposal and explore multiple aspects of the design that are different from existing designs, what their advantages and issues are and how they are implemented as an attempt to implement the proposal was made.

\section{Problem}

Scheme:

The design is the closest to Truffle. 
we tried to keep minimal interface (no n code installation and bytecode as they do)
in the context of st snapshot, persist processes and because bc to bc, we had everything to persist optimization to avoid deopt for snapshots.
While first bench working and trying to reach production state, we tested opt and deopt separatedly. 

ToREMOVE:
\begin{itemize} 
	\item The interface between the VM including the baseline JIT and the bytecode compiler and the language including the optimising bytecode to bytecode JIT needs to be expressive so optimised code can be installed and run while it needs to be small to keep the design easy to work with. 
	\item Multiple properties can be conserved across multiple start-ups in this design, including the persistance of the running processes state and runtime optimisations.
	\item Testing and validating that an optimising JIT compiler is a difficult task. As the design splits this part of the VM in two independent parts, with the optimising bytecode to bytecode JIT in the language and the back-end as the baseline JIT in the VM, it is possible to consider new ways of testing and validating the optimising JIT.
\end{itemize}

END ToREMOVE.

The thesis focuses on these three problems:

\begin{itemize}
	\item \emph{Problem 1:} How to design a minimal interface between the VM and the language in the context of a language-side optimising JIT ?
	\item \emph{Problem 2:} How to persist the runtime state across multiple VM start-up, including the running processes state and the optimised code state ?
	\item \emph{Problem 3:} Is it possible to validate the language-side optimising JIT independently from the back-end ?
\end{itemize}

\section{Contributions}

we implemented to validate blabla. PArt of it is integrated, other part is available as separate MIT project.

\section{Artifacts}

\section{Outline}

\ifx\wholebook\relax\else
    \end{document}
\fi
\ifx\wholebook\relax\else

% --------------------------------------------
% Lulu:

    \documentclass[a4paper,12pt,twoside]{../includes/ThesisStyle}

	\usepackage[T1]{fontenc} %%%key to get copy and paste for the code!
%\usepackage[utf8]{inputenc} %%% to support copy and paste with accents for frnehc stuff
\usepackage{times}
\usepackage{ifthen}
\usepackage{xspace}
\usepackage{alltt}
\usepackage{latexsym}
\usepackage{url}            
\usepackage{amssymb}
\usepackage{amsfonts}
\usepackage{amsmath}
\usepackage{stmaryrd}
\usepackage{enumerate}
\usepackage{cite}
%\usepackage[pdftex,colorlinks=true,pdfstartview=FitV,linkcolor=blue,citecolor=blue,urlcolor=blue]{hyperref}
\usepackage{xspace}
%\usepackage{graphicx}
\usepackage{subfigure}
\usepackage[scaled=0.85]{helvet}
        
        
\newcommand{\sepe}{\mbox{>>}}
\newcommand{\pack}[1]{\emph{#1}}
\newcommand{\ozo}{\textsc{oZone}\xspace}
\newcommand\currentissues{\par\smallskip\textbf{Current Issues -- }}

\newboolean{showcomments}
\setboolean{showcomments}{true}
\ifthenelse{\boolean{showcomments}}
  {\newcommand{\bnote}[2]{
	\fbox{\bfseries\sffamily\scriptsize#1}
    {\sf\small$\blacktriangleright$\textit{#2}$\blacktriangleleft$}
    % \marginpar{\fbox{\bfseries\sffamily#1}}
   }
   \newcommand{\cvsversion}{\emph{\scriptsize$-$Id: macros.tex,v 1.1.1.1 2007/02/28 13:43:36 bergel Exp $-$}}
  }
  {\newcommand{\bnote}[2]{}
   \newcommand{\cvsversion}{}
  } 


\newcommand{\here}{\bnote{***}{CONTINUE HERE}}
\newcommand{\nb}[1]{\bnote{NB}{#1}}
\newcommand{\fix}[1]{\bnote{FIX}{#1}}
%%%% add your own macros 

\newcommand{\sd}[1]{\bnote{Stef}{#1}}
\newcommand{\ja}[1]{\bnote{Jannik}{#1}}
\newcommand{\na}[1]{\bnote{Nico}{#1}}
%%% 


\newcommand{\figref}[1]{Figure~\ref{fig:#1}}
\newcommand{\figlabel}[1]{\label{fig:#1}}
\newcommand{\tabref}[1]{Table~\ref{tab:#1}}
\newcommand{\layout}[1]{#1}
\newcommand{\commented}[1]{}
\newcommand{\secref}[1]{Section \ref{sec:#1}}
\newcommand{\seclabel}[1]{\label{sec:#1}}

%\newcommand{\ct}[1]{\textsf{#1}}
\newcommand{\stCode}[1]{\textsf{#1}}
\newcommand{\stMethod}[1]{\textsf{#1}}
\newcommand{\sep}{\texttt{>>}\xspace}
\newcommand{\stAssoc}{\texttt{->}\xspace}

\newcommand{\stBar}{$\mid$}
\newcommand{\stSelector}{$\gg$}
\newcommand{\ret}{\^{}}
\newcommand{\msup}{$>$}
%\newcommand{\ret}{$\uparrow$\xspace}

\newcommand{\myparagraph}[1]{\noindent\textbf{#1.}}
\newcommand{\eg}{\emph{e.g.,}\xspace}
\newcommand{\ie}{\emph{i.e.,}\xspace}
\newcommand{\ct}[1]{{\textsf{#1}}\xspace}


\newenvironment{code}
    {\begin{alltt}\sffamily}
    {\end{alltt}\normalsize}

\newcommand{\defaultScale}{0.55}
\newcommand{\pic}[3]{
   \begin{figure}[h]
   \begin{center}
   \includegraphics[scale=\defaultScale]{#1}
   \caption{#2}
   \label{#3}
   \end{center}
   \end{figure}
}

\newcommand{\twocolumnpic}[3]{
   \begin{figure*}[!ht]
   \begin{center}
   \includegraphics[scale=\defaultScale]{#1}
   \caption{#2}
   \label{#3}
   \end{center}
   \end{figure*}}

\newcommand{\infe}{$<$}
\newcommand{\supe}{$\rightarrow$\xspace}
\newcommand{\di}{$\gg$\xspace}
\newcommand{\adhoc}{\textit{ad-hoc}\xspace}

\usepackage{url}            
\makeatletter
\def\url@leostyle{%
  \@ifundefined{selectfont}{\def\UrlFont{\sf}}{\def\UrlFont{\small\sffamily}}}
\makeatother
% Now actually use the newly defined style.
\urlstyle{leo}



	\input{../includes/formatAndDefs}

	\graphicspath{{.}{../figures/}}
	\begin{document}
\fi

\chapter{State of the art}
\label{chap:stateOfTheArt}
\minitoc

The thesis is an attempt to build an optimising JIT for Pharo. We discuss in the first section about the two main different architectures possible to build an optimising JIT. The traditional architecture focuses on the optimisation of frequently used functions based on its previous runs. The modern architecture, the meta-tracing JITs, attempts to optimise linear sequences of frequently used instructions. The chapter follows by analysing existing work on the three reseach problems.

Firstly, the JITs written in a high-level language are discussed. Multiple high-level languages were used with different execution models. In the case where the optimising JIT is running in the same runtime than the optimised application, some JITs are able to optimise their own code. However, certains constraints exist to avoid metacircular issues.

Secondly, most modern VMs always start-up an application with only non optimised code. This can be a problem in specific cases. For example, the application needs a certain amount of time, called \emph{warm-up time}, to reach peak performance, which is a problem if the application needs high-performance immediately. We detail how some VMs attempt to persist part of the optimisations across multiple start-ups.

Thirdly, the Sista architecture is designed so that the optimising JIT re-use the baseline JIT as a back-end while being written in a different programming language. We detail one of the rare cases where the JIT back-end is shared between the baseline JIT and the optimising JIT. We follow up by discussing the interface provided from the VM to generate and execute efficient machine code.

%%%%%%%%%%%%%%%%%%%%%%%%%%%%%%%%%%%%%%%%%%%%%%%%%%%%%%%%%%%%%%%%%%%%%%%%%%%%%%%%%%%%%%%%%%%%%%%%%%%%%%%%%%%%%%%%%%%%%%%%%%%%%%%%%%%%%%%%%%%%%%%%%%%%%%%%%%%%%%%%%%%%%%%

\section{Optimising Just-in-Time compiler architecture}



intro on the need of optimising JIT quickly. 

discuss quickly AOT, advantages and cons.

Explain more compile time vs number of time it is executed

modern st VM no opt JIT in prod

do not really discuss closed-source.

two main tendencies.

\subsection{Classical tiered architecture}

explain overall architecture.

cost of opt compared to runtime perf improvement.
first run required to direct in dynamic language.

quote Self (never really become mainstream), into strongtalk (ST but never reach prod) and hotspot, V8 and Dart. 

Designed to be plugged in other VMS ? Extracted from Maxine and put on hotspot. Can be used as a specific compiler.
Explain quickly multilanguage
Discuss Graal, Truffle + Graal.

describe the runtime model

explain JIT generation from annotation on AST interpreter

\paragraph{Number of tiers}

explain tier between 1 and 4

Example of V8 (2 -> 3 tiers)
Example of webkit 4 tiers + shared back-end.

\paragraph{Sharing back-ends}

Shared back-end in webkit 

Shared backend in TurboFan and WebAssembly


\subsection{Meta-tracing architecture}

+ Mozilla monkeys
LuaJIT

Alternative for multi language
Rpython + Pypy.
RPython tool chain \cite{Rigo06a}

explain quickly metatracing from interpreter

present them here so they can be compared to later.

%%%%%%%%%%%%%%%%%%%%%%%%%%%%%%%%%%%%%%%%%%%%%%%%%%%%%%%%%%%%%%%%%%%%%%%%%%%%%%%%%%%%%%%%%%%%%%%%%%%%%%%%%%%%%%%%%%%%%%%%%%%%%%%%%%%%%%%%%%%%%%%%%%%%%%%%%%%%%%%%%%%%%%%

\section{Virtual machine implementation language}


Can the JIT optimise itself ?

1) Low level languages

typically C / C++ or other

why advantages and cons

2) DSl / HIGH level language compiling AOT to assembly code.


3) Metacircular VM (check metacircular related work)

%%%%%%%%%%%%%%%%%%%%%%%%%%%%%%%%%%%%%%%%%%%%%%%%%%%%%%%%%%%%%%%%%%%%%%%%%%%%%%%%%%%%%%%%%%%%%%%%%%%%%%%%%%%%%%%%%%%%%%%%%%%%%%%%%%%%%%%%%%%%%%%%%%%%%%%%%%%%%%%%%%%%%%%

\section{Runtime state persistance}

Check paper Sista persistance

1) Fast warm-up

Discuss tiered arch and lower warm-up.

saving metadata + strongtalk and maybe Dart

save machine code and Azul

AOT

2) Preheating through snapshot in Dart

Snapshot in general for Kernel ?

%%%%%%%%%%%%%%%%%%%%%%%%%%%%%%%%%%%%%%%%%%%%%%%%%%%%%%%%%%%%%%%%%%%%%%%%%%%%%%%%%%%%%%%%%%%%%%%%%%%%%%%%%%%%%%%%%%%%%%%%%%%%%%%%%%%%%%%%%%%%%%%%%%%%%%%%%%%%%%%%%%%%%%%

\section{Interface}

+ Graal and interface

WebAssembly 

%%%%%%%%%%%%%%%%%%%%%%%%%%%%%%%%%%%%%%%%%%%%%%%%%%%%%%%%%%%%%%%%%%%%%%%%%%%%%%%%%%%%%%%%%%%%%%%%%%%%%%%%%%%%%%%%%%%%%%%%%%%%%%%%%%%%%%%%%%%%%%%%%%%%%%%%%%%%%%%%%%%%%%%

%MAybe conclusion, 1 sentence what we did and what comes next

%%%%%%%%%%%%%%%%%%%%%%%%%%%%%%%%%%%%%%%%%%%%%%%%%%%%%%%%%%%%%%%%%%%%%%%%%%%%%%%%%%%%%%%%%%%%%%%%%%%%%%%%%%%%%%%%%%%%%%%%%%%%%%%%%%%%%%%%%%%%%%%%%%%%%%%%%%%%%%%%%%%%%%%

%FOLLOWING IS OLD VERSION FOR HISTORY.

%In this chapter, the most popular production VMs and relevant research VMs are discussed. In further chapter, the thesis' proposed architecture will be compared against those VMs. 

%Most popular production VMs, such as Java hotspot (Cite) or Javascript's V8 (Cite) VMs are written in C++. Using C++ as a performance oriented low-level programming language proved to be very effective as it is possible to write code in a performance oriented fashion. A clear separation is made between the VM and the programming language run so there are no metacircular problems.

%Most of those VMs start-up from the language kernel, a set of core librairies and either source files or files containing bytecodes. Reaching peak performance takes a certain amount of time as the VM needs to detect and optimise correctly frequently used patterns of code. Reportedly, this warm-up time can be from several milliseconds up to multiple days. To solve partially the warm-up problem, such VMs are built with a tiered-architecture: the first few executions are run slowly but without any compilation time, subsequent hundreds of executing are run a bit faster with limited compilation time while further execution are run at peak performance after a certain amount of compilation time.

%An interesting point ot note is that several mainstream VMs were led by the same person (Lars Bak), who became very good at implementing very efficient and easy-to-maintain VMs in C++ as he implemented multiple of those in his life. His work therefore pushed the direction of VM implementation in the C++ direction.

%Among the C++ virtual machines, we will detail two specific VMs have uncommon features that are relevant in the context of the thesis. 

%\subsection{Azul}
%The Azul VM \cite{Azul} is a closed-source VM and expensive VM for Java. As for all closed-source projects, no one external to the project can be certain of what the code is doing. However, word has been that the Azul VM is able to persist optimised machine code across multiple start-ups. If the application is started on another processor, then the saved machine code is simply discarded. 

%\subsection{Dart}
%The Dart VM is an open-source VM for the Dart programming language. Dart features snapshots for fast application start-up. In Dart, the programmer can generate different kind of snapshots \cite{Anna13a}. Since that publication, the Dart team have added two new kind of snapshots, specialized for iOS and Android application deployment, which are quite similar to our snapshots.


%As the sista architecture is implemented in the context of Smalltalk, it could be relevant to discuss existing Smalltalk virtual machines. These VMs are interesting for multiple reasons, but unfortunately many of the Smalltalk virtual machines in production today are closed-source, making the discussion around them not that relevant as information is missing and unaccessible. In addition, as far as we know, there are no production Smalltalk VM today with an optimising JIT compiler, so the comparison with such VMs is even less relevant. 

%However, speculative optimisations in VMs started with the Self VM (CITE), with Self being a Smalltalk-like language, and was followed up with the strongtalk VM (CITE). Both VMs are open-source and available today but none of them are used in production. Self had never really broken through mainstream programming while strongtalk had never reached production state.

\ifx\wholebook\relax\else
    \end{document}
\fi
\ifx\wholebook\relax\else

% --------------------------------------------
% Lulu:

    \documentclass[a4paper,12pt,twoside]{../includes/ThesisStyle}

	\usepackage[T1]{fontenc} %%%key to get copy and paste for the code!
%\usepackage[utf8]{inputenc} %%% to support copy and paste with accents for frnehc stuff
\usepackage{times}
\usepackage{ifthen}
\usepackage{xspace}
\usepackage{alltt}
\usepackage{latexsym}
\usepackage{url}            
\usepackage{amssymb}
\usepackage{amsfonts}
\usepackage{amsmath}
\usepackage{stmaryrd}
\usepackage{enumerate}
\usepackage{cite}
%\usepackage[pdftex,colorlinks=true,pdfstartview=FitV,linkcolor=blue,citecolor=blue,urlcolor=blue]{hyperref}
\usepackage{xspace}
%\usepackage{graphicx}
\usepackage{subfigure}
\usepackage[scaled=0.85]{helvet}
        
        
\newcommand{\sepe}{\mbox{>>}}
\newcommand{\pack}[1]{\emph{#1}}
\newcommand{\ozo}{\textsc{oZone}\xspace}
\newcommand\currentissues{\par\smallskip\textbf{Current Issues -- }}

\newboolean{showcomments}
\setboolean{showcomments}{true}
\ifthenelse{\boolean{showcomments}}
  {\newcommand{\bnote}[2]{
	\fbox{\bfseries\sffamily\scriptsize#1}
    {\sf\small$\blacktriangleright$\textit{#2}$\blacktriangleleft$}
    % \marginpar{\fbox{\bfseries\sffamily#1}}
   }
   \newcommand{\cvsversion}{\emph{\scriptsize$-$Id: macros.tex,v 1.1.1.1 2007/02/28 13:43:36 bergel Exp $-$}}
  }
  {\newcommand{\bnote}[2]{}
   \newcommand{\cvsversion}{}
  } 


\newcommand{\here}{\bnote{***}{CONTINUE HERE}}
\newcommand{\nb}[1]{\bnote{NB}{#1}}
\newcommand{\fix}[1]{\bnote{FIX}{#1}}
%%%% add your own macros 

\newcommand{\sd}[1]{\bnote{Stef}{#1}}
\newcommand{\ja}[1]{\bnote{Jannik}{#1}}
\newcommand{\na}[1]{\bnote{Nico}{#1}}
%%% 


\newcommand{\figref}[1]{Figure~\ref{fig:#1}}
\newcommand{\figlabel}[1]{\label{fig:#1}}
\newcommand{\tabref}[1]{Table~\ref{tab:#1}}
\newcommand{\layout}[1]{#1}
\newcommand{\commented}[1]{}
\newcommand{\secref}[1]{Section \ref{sec:#1}}
\newcommand{\seclabel}[1]{\label{sec:#1}}

%\newcommand{\ct}[1]{\textsf{#1}}
\newcommand{\stCode}[1]{\textsf{#1}}
\newcommand{\stMethod}[1]{\textsf{#1}}
\newcommand{\sep}{\texttt{>>}\xspace}
\newcommand{\stAssoc}{\texttt{->}\xspace}

\newcommand{\stBar}{$\mid$}
\newcommand{\stSelector}{$\gg$}
\newcommand{\ret}{\^{}}
\newcommand{\msup}{$>$}
%\newcommand{\ret}{$\uparrow$\xspace}

\newcommand{\myparagraph}[1]{\noindent\textbf{#1.}}
\newcommand{\eg}{\emph{e.g.,}\xspace}
\newcommand{\ie}{\emph{i.e.,}\xspace}
\newcommand{\ct}[1]{{\textsf{#1}}\xspace}


\newenvironment{code}
    {\begin{alltt}\sffamily}
    {\end{alltt}\normalsize}

\newcommand{\defaultScale}{0.55}
\newcommand{\pic}[3]{
   \begin{figure}[h]
   \begin{center}
   \includegraphics[scale=\defaultScale]{#1}
   \caption{#2}
   \label{#3}
   \end{center}
   \end{figure}
}

\newcommand{\twocolumnpic}[3]{
   \begin{figure*}[!ht]
   \begin{center}
   \includegraphics[scale=\defaultScale]{#1}
   \caption{#2}
   \label{#3}
   \end{center}
   \end{figure*}}

\newcommand{\infe}{$<$}
\newcommand{\supe}{$\rightarrow$\xspace}
\newcommand{\di}{$\gg$\xspace}
\newcommand{\adhoc}{\textit{ad-hoc}\xspace}

\usepackage{url}            
\makeatletter
\def\url@leostyle{%
  \@ifundefined{selectfont}{\def\UrlFont{\sf}}{\def\UrlFont{\small\sffamily}}}
\makeatother
% Now actually use the newly defined style.
\urlstyle{leo}



	\input{../includes/formatAndDefs}

	\graphicspath{{.}{../figures/}}
	\begin{document}
\fi

\chapter{Existing Pharo runtime}
\label{chap:existing}
\minitoc

%global intro
This chapter describes the Smalltalk dialect Pharo and part of its implementation. The Sista architecture was originally designed to improve the performance of the Pharo VM by adding speculative optimisations. Some existing features and implementation details already present in Pharo impacted several design decisions. They are detailled in this chapter to help the reader understanding the design decisions explained in further chapters. The chapter is not meant to explain the whole existing implementation, but only the most relevant points for the thesis. 

Pharo is an object-oriented language. Everything is an object, including classes or bytecoded versions of methods. It is dynamically-typed and every call is a virtual call. The virtual machine relies on a bytecode interpreter and a JIT to gain performance, similarly to Java virtual machines \cite{JavaVM8}. Modern Smalltalks directly inherit from Smalltalk-80~\cite{Gold83a} but have evolved during the past 35 years. For example, real closures and exceptions were added.

The chapter successively describes the Pharo programming language, the interface with its VM and some aspects of the VM implementation.

\section{Pharo programming language}

This section details three main aspects of the programming language: native thread management, stack frame reification and snapshots.

\paragraph{Native threads management}

Pharo features a global interpreter lock, similarly to python. Only calls to external libraries through the foreign function interface and specific virtual machine extensions have access to the other native threads. All the different virtual machine tasks, such as bytecode interpretation, machine code execution, just-in-time compilation or garbage collection are not done concurrently. This has a impact on design decisions because several other VMs implement the optimising JIT in concurrent native threads to the application (CITE). As Scorch is written in Pharo, the optimisation of v-functions and the application optimised are running in the same native thread.

\paragraph{Stack frame reification.}

The current VM evolved from the VM specified in the blue book~\cite{Gold83a}. The original specification relied on a spaghetti stack: the execution stack was represented as a linked list of v-function activations. Each v-function activation was represented as an object that could be read or written by the program. Over the years, Deutsch and Schiffman~\cite{Deut84a} changed the representation of the stack in the VM to use an execution stack similar to the one used in other programming languages, where stack frames are next to each other on a single stack. However, the VM still provides the ability to the programmer to read and write reified stack frames as if they where objects. To do so, each stack frame is reified as an object on demand. 

The reification of a stack frame abstracts away from the low-level details, a reified stack frame accessed from the language always looks like it is a v-function activation. A reified frame is exactly the same if the VM is started with the interpreter only or with the hybrid interpreter plus baseline JIT runtime. Conceptually, for the Smalltalk developer, Smalltalk code is interpreted and the reified frame always look like so. In fact, it is possible that some developers manipulate the stack without even knowing that there are n-frame or v-frame in practice in the VM.

The reification of the stack frames is used in three main places: the debugger, exceptions and continuations. For the two latter, they are implemented in Smalltalk on top of the stack frame reification, without any special VM support. In the Sista architecture, as Scorch is written in Pharo, it can use this feature to instrospect or modify the stack.

\paragraph{Snapshots.}

A snapshot~\footnote{Smalltalk developers use the term \emph{image} instead of snapshot.} is a sequence of bytes that represents a serialized form of all the objects present at a precise moment in the runtime. As everything is an object in Smalltalk, including green threads, the virtual machine can, at start-up, load all the objects from a snapshot and resume the execution based on the active green thread precised by the snapshot. In fact, this is the normal way of launching a Smalltalk runtime. 

One interesting problem in snapshots is how to save the execution stack, \ie the green threads. It possible in the existing VM to convert reify each stack frame into an object. To perform a snapshot, each stack frame is reified and only objects are saved in the snapshot. When the snapshot is restarted, the VM recreates a stack frame for each reified frame lazily. As discussed in the previous paragraph, all reified frames are in virtual state. In any case, snapshots cannot save n-frames because they are platform-independent. In the Pharo VM for example, a snapshot can be taken on a laptop using a x86 processor and restarted on a raspberry pie using a ARMv6 processor.

%%%%%%%%%%%%%%%%%%%%%%%%%%%%%%%%%%%%%%%%%%%%%%%%%%%%%%%%%%%%%%%%%%%%%%%%%%%%%%%%%%%%%%%%%%%%%%%%%%%%%%%%%%%%%%%%%%%%%%%%%%%%%%%%

%restart here.

\section{Language-VM interface}

small intro

\subsection{Executable code and bytecode representation}

As everything is an object, executable code are objects too. From the language, reflective APIs are available to access the bytecode version of the compiled methods and closure (we call them compiled functions). A compiled function includes a function header, to describe specific things such as the required frame size, the number of temporary variables or the number of literals of the function, a list of literals, the bytecodes and a reference to the source code. 

(FIGURE CM)

The bytecode set is stack-based. This means that most operations are pushing and popping values of the stack. All the operations are untyped and work with any object. One of the main instruction is the virtual call instructions, popping the receiver and arguments from the stack and pushing back the result. The bytecode set also includes conditional and unconditional branches to encode condition and loops.

\subsection{Registered objects}

discuss call-backs.

\subsection{Primitive methods}

I have written somewhere where sendAndBranchData is described what is a primitive.

+ become because discuss, explain both main use for instance migration and proxy

\section{Virtual machine}

overview

\subsection{Virtual machine executable generation}


\cite{Inga97a}

In this section is detailled the current process to generate the existing production virtual machine. 

Most of the execution engine (the memory manager, the interpreter and the baseline JIT) are written in Slang, a subset of Smalltalk. The Slang code can be translated to C, and then the C code can be compiled to machine code by standard C compilers.

The slang code has two main advantages over plain C:
\begin{itemize}
	\item \emph{Specifying inlining and code duplication:} To keep the interpreter code efficient, one has to be very careful on what code is inlined in the main interpreter loop and what code is not. In addition, for performance, specific code may need to be duplicated. For example, the interpreter code to push a temporary variable on stack is duplicated 17 times, the 16 first versions are dedicated versions for temporary numbers 0 to 15, the most common cases, more efficient because of constant usage, and the 17th version is the generic version. Slang allows to specify for each function if it needs to be inlined or duplicated udring Slang to C compilation.
	\item \emph{Simulation:} As Slang is a subset of Smalltalk, it can be executed as normal Smalltalk code. This is used to simulate the interpreter and garbage collector behavior. The JIT runtime is simulated using both Slang execution and external processor simulators. Simulation is very convenient to debug the VM, as all the Smalltalk debugging tools are available, the simulator can be saved and it is deterministic
\end{itemize}

\subsection{Baseline JIT}

The baseline JIT, called \emph{Cogit}, is used to compile bytecode
- baseline JIT behavior + simple reg alloc + templates + abstractions over and back-ends, memory managers
Doc: my blog posts. need to explain template and co

\subsection{Stack frame reification}

%Flag: FRAMEREIFICATION.

The VM is responsible for intercepting all accesses to a n-frame and map correctly the value to the v-frame value, and in some case, such as virtual instruction pointer modification, the VM converts the n-frame to a v-frame.


The reified object acts as a proxy to the stack frame for reads and simple write operations. Advanced operations, such as setting the caller of a stack frame, are done by abusing returns across stack pages. In rare cases, the context objects can be a full object and not just proxies to a stack frame, for example 
%when the programmer creates the object from the language using the \ct{thisContext} special variable or 
when the VM has no more stack pages available, it creates full context objects for all the stack frames used on the least recently used stack pages. Returning to such context objects can be done only with a return across stack pages, and the VM recreates a stack frame for the context object to be able to resume execution.

The call stack is reified on demand as a linked list of context object. Each context object correspond to a stack frame. For the Smalltalk programmer, a context object is a normal object where all its fields (caller, program counter, temporaries, etc.) can be edited as all objects. The VM uses a complex proxy architecture to remap correctly the programmer's modification to the stack. Most transformation, such as editing the temporary variables, are easy to implement as there is a direct mapping to the stack frame. Other transformations are more complex, for instance, the mutation of a context object caller. If that operation happens, the VM split the stack in two at the mutation and transforms it so that each part of the stack is on a different stack pages. Returns across stack pages are handled specifically to keep the stack size of each process reasonnable.

mutation of pc.

can crash (responsibility of the user)



Discuss become in addition
+ become because used both in Mem manager paper and deopt solution 1.

\ifx\wholebook\relax\else
    \end{document}
\fi
\ifx\wholebook\relax\else

% --------------------------------------------
% Lulu:

    \documentclass[a4paper,12pt,twoside]{../includes/ThesisStyle}

	\usepackage[T1]{fontenc} %%%key to get copy and paste for the code!
%\usepackage[utf8]{inputenc} %%% to support copy and paste with accents for frnehc stuff
\usepackage{times}
\usepackage{ifthen}
\usepackage{xspace}
\usepackage{alltt}
\usepackage{latexsym}
\usepackage{url}            
\usepackage{amssymb}
\usepackage{amsfonts}
\usepackage{amsmath}
\usepackage{stmaryrd}
\usepackage{enumerate}
\usepackage{cite}
%\usepackage[pdftex,colorlinks=true,pdfstartview=FitV,linkcolor=blue,citecolor=blue,urlcolor=blue]{hyperref}
\usepackage{xspace}
%\usepackage{graphicx}
\usepackage{subfigure}
\usepackage[scaled=0.85]{helvet}
        
        
\newcommand{\sepe}{\mbox{>>}}
\newcommand{\pack}[1]{\emph{#1}}
\newcommand{\ozo}{\textsc{oZone}\xspace}
\newcommand\currentissues{\par\smallskip\textbf{Current Issues -- }}

\newboolean{showcomments}
\setboolean{showcomments}{true}
\ifthenelse{\boolean{showcomments}}
  {\newcommand{\bnote}[2]{
	\fbox{\bfseries\sffamily\scriptsize#1}
    {\sf\small$\blacktriangleright$\textit{#2}$\blacktriangleleft$}
    % \marginpar{\fbox{\bfseries\sffamily#1}}
   }
   \newcommand{\cvsversion}{\emph{\scriptsize$-$Id: macros.tex,v 1.1.1.1 2007/02/28 13:43:36 bergel Exp $-$}}
  }
  {\newcommand{\bnote}[2]{}
   \newcommand{\cvsversion}{}
  } 


\newcommand{\here}{\bnote{***}{CONTINUE HERE}}
\newcommand{\nb}[1]{\bnote{NB}{#1}}
\newcommand{\fix}[1]{\bnote{FIX}{#1}}
%%%% add your own macros 

\newcommand{\sd}[1]{\bnote{Stef}{#1}}
\newcommand{\ja}[1]{\bnote{Jannik}{#1}}
\newcommand{\na}[1]{\bnote{Nico}{#1}}
%%% 


\newcommand{\figref}[1]{Figure~\ref{fig:#1}}
\newcommand{\figlabel}[1]{\label{fig:#1}}
\newcommand{\tabref}[1]{Table~\ref{tab:#1}}
\newcommand{\layout}[1]{#1}
\newcommand{\commented}[1]{}
\newcommand{\secref}[1]{Section \ref{sec:#1}}
\newcommand{\seclabel}[1]{\label{sec:#1}}

%\newcommand{\ct}[1]{\textsf{#1}}
\newcommand{\stCode}[1]{\textsf{#1}}
\newcommand{\stMethod}[1]{\textsf{#1}}
\newcommand{\sep}{\texttt{>>}\xspace}
\newcommand{\stAssoc}{\texttt{->}\xspace}

\newcommand{\stBar}{$\mid$}
\newcommand{\stSelector}{$\gg$}
\newcommand{\ret}{\^{}}
\newcommand{\msup}{$>$}
%\newcommand{\ret}{$\uparrow$\xspace}

\newcommand{\myparagraph}[1]{\noindent\textbf{#1.}}
\newcommand{\eg}{\emph{e.g.,}\xspace}
\newcommand{\ie}{\emph{i.e.,}\xspace}
\newcommand{\ct}[1]{{\textsf{#1}}\xspace}


\newenvironment{code}
    {\begin{alltt}\sffamily}
    {\end{alltt}\normalsize}

\newcommand{\defaultScale}{0.55}
\newcommand{\pic}[3]{
   \begin{figure}[h]
   \begin{center}
   \includegraphics[scale=\defaultScale]{#1}
   \caption{#2}
   \label{#3}
   \end{center}
   \end{figure}
}

\newcommand{\twocolumnpic}[3]{
   \begin{figure*}[!ht]
   \begin{center}
   \includegraphics[scale=\defaultScale]{#1}
   \caption{#2}
   \label{#3}
   \end{center}
   \end{figure*}}

\newcommand{\infe}{$<$}
\newcommand{\supe}{$\rightarrow$\xspace}
\newcommand{\di}{$\gg$\xspace}
\newcommand{\adhoc}{\textit{ad-hoc}\xspace}

\usepackage{url}            
\makeatletter
\def\url@leostyle{%
  \@ifundefined{selectfont}{\def\UrlFont{\sf}}{\def\UrlFont{\small\sffamily}}}
\makeatother
% Now actually use the newly defined style.
\urlstyle{leo}



	\input{../includes/formatAndDefs}

	\graphicspath{{.}{../figures/}}
	\begin{document}
\fi

\chapter{Sista Architecture}
\label{chap:architecture}
\minitoc

The overall thesis focuses on the design and implementation of an optimising JIT compiler for Pharo, written in Pharo itself, running in the same runtime than the optimised application on top of the existing runtime environment. In this chapter, we detail the architecture designed and implemented to make it work.

Cogit was extended to detect hot spot through profiling counters in non optimised n-functions. When a hot spot is detected, Cogit immediately calls Scorch in Pharo. Scorch then looks for the best v-function to optimise based on the current stack, optimises it and installs the optimised version. To perform optimisation, Scorch may ask Cogit to introspect specific n-functions to extract type information and basic block usage from previous runs. Once installed, the VM can execute the optimised v-function at the next call to the function. As the VM runtime is hybrid between an interpreter and Cogit, the optimised v-function may be interpreted or compiled to an optimised n-function by Cogit. As optimised v-functions have accessed to operations not normally possible by v-functions, both the interpreter and Cogit were extended to support the new operations.

Due to speculative optimisations, optimised v-functions may contain guards to ensure optimisation-time speculations are valid at runtime or dynamically deoptimise the code. If a guard fails, the optimised code needs to be deoptimised and the stack editing to resume execution with non optimised code. Cogit is only able to map a stack frame from n-function state (values in register, machine-specific calling convetions, etc.) to a single stack frame in v-function state as if the function was interpreted (all values are on stack). Hence, when an optimised n-function needs to be deoptimised, Cogit maps the stack frame to v-function state, provides it to Scorch, which maps the stack frame holding the optimised v-function to multiple stack frames with non-optimised v-functions. The execution can then resume using non optimised code.

\section {Function's optimisation}

%overview

%first, repeat all steps.

%second, explain problem with responsiveness and past from next chapter the explanation with critical and background mode.

%Detail

%Hot spot detection ?

%VM call-back ? => explain what's in C and what is not in C

%Stack search, explain almost never bottom stack frame ?

%optimisation. Maybe a short paragraph

%Either postpone or installion (include dependencies)

%restart code, return, frame just after application frame


\begin{enumerate}
	\item \emph{Hot spot detection:} When the baseline JIT compiler generates a non optimised n-function, it inserts profiling counters. Each time the execution flow reaches a counter, it is incremented and when the counter reaches a threshold, the function is detected as a hot spot.
	\item \emph{VM call-back:} When a hot spot is detected, a specific Slang routine is called. The routine makes sure the stack frame with the tripping counter is reified or reify it, then it performs a virtual call with a selector specified in Smalltalk with the reified stack frame as receiver. This moves hot spot detection management from the VM to Smalltalk.
	\item \emph{Stack search:} In Smalltalk Scorch introspects the reified stack and looks for a function to optimise. Based on multiple heuristics, one of the bottom stack frames' functions is selected for optimisations (It selects a function where as many closures as possible can be inlined).
	\item \emph{Optimisation:} Scorch has then a limited amount of time to optimise the selected v-function. To speculate on types, Scorch request the VM to introspect specific n-function to extract the type information found in inline caches and profiling counter to determine basic block usage. The optimisations performed are standard and basic (mainly speculative inlining, array bounds check elimination, global value numbering, loop invariant code motion). Scorch finally generates an optimised v-function encoded in an extended bytecode set annotated with deoptimisation and dependency metadata.
	\item \emph{Installation:} The optimized function is installed, either in the method dictionary of a class if this is a method, or in a method if it's a closure. The dependencies are installed in the dependency manager so that if new code is loaded, code that may be dependent is discarded.
	\item \emph{Code restart:} The execution flow then returns to the tripping function to resume the execution of the code. The return is performed as a normal return, no special VM routine is called. If an optimised function is installed, it will be activated at next call.
\end{enumerate}




\section {Function's deoptimisation}

The dynamic deoptimization process, again, is very similar to other virtual machines \cite{Fin03a, Holz92a}. The main difference is that it is split in two parts: firstly the baseline JIT maps  machine state to a state as if the bytecode interpreter would execute the function, second the deoptimizer maps the interpreter state to the deoptimized interpreter frames.

During dynamic deoptimization, we deal only with the recovery of the stack from its optimized state using optimized functions to the  unoptimized state using unoptimized functions. The unoptimized code itself is always present, as the bytecode version of the  unoptimized function is quite compact. As far as we know, modern VM such as V8~\cite{V8} always keep the machine code representation of unoptimized functions, which is less compact than the bytecode version, so we believe keeping the unoptimized bytecode function is not a problem in term of memory footprint.

\begin{enumerate}
\item \emph{Deoptimization trigger:} Deoptimization can happen in two main cases. First, a guard inserted during the optimization phases of the compiler has failed. Second, the language requests the stack to be deoptimized, typically for debugging.
\item \emph{JIT map:} The first step, done by the baseline JIT compiler is to map the machine code state of the stack frame to the bytecode interpreter state, as it would do for an unoptimized method. This mapping is a one-to-one mapping: a machine code stack frame maps to a single interpreter stack frame. In this step, the baseline JIT maps the machine code program counter to the bytecode program counter, boxes unboxed values present and spills values in registers on stack.
\item \emph{Deoptimizer map:} The JIT then requests the deoptimizer to map the stack frame of the optimized bytecoded function to multiple stack frames of unoptimized functions. In this step, it can also rematerialize objects from values on stack and constants, whose allocations have been removed by the optimizer. The stack with all the unoptimized functions at the correct bytecode interpreter state is recovered.
\item \emph{Stack edition:}
The deoptimizer edits the bottom of the stack to use the deoptimized stack frames instead of the optimized ones, and resumes execution in the unoptimized stack.
\end{enumerate}

+ Debugger and code loading support: reuse discarding logic and deopt logic

%%%%%%%%%%%%%%%%%%%%%%%%%%%%%%%%%%%%%%%%%%%%%%%%%%%%%%%%%%%%%%%%%%%%%%%%%%%%%%%%%%%%%%%%%%%%%%%%%%%%%%%%%%%%%%%%%%%%%%%%%%%%%%%%%%%%%%%%
%%%%%%%%%%%%%%%%%%%%%%%%%%%%%%%%%%%%%%%%%%%%%%%%%%%%%%%%%%%%%%%%%%%%%%%%%%%%%%%%%%%%%%%%%%%%%%%%%%%%%%%%%%%%%%%%%%%%%%%%%%%%%%%%%%%%%%%%

\section{Required language evolutions}

evolutions required for support

\subsection{Profiling counters}


To detect frequently used method, counters were introduced in code compiled by the baseline JIT, triggering a call-back when the counter reaches a specific threshold. Profiling counters add some overhead. As the baseline JIT is used both to compile non optimised and optimised code, compilation became conditional to introduce counters only in non optimised code. For this purpose, each bytecoded function is marked as being optimized or not thanks to a bit in the header. If the function has not yet been optimized, the baseline JIT generates counters in the machine code that are incremented each time they are reached by the flow of execution. Once a counter reaches a threshold, a call-back requests the runtime optimizer to generate optimized code.

Based on \cite{Arn02}, we added counters by extending the way the baseline JIT generates conditional jumps to add counters just before and just after the branch. In several other VMs, the counters are added at the beginning of each function. The technique we used allowed us to reduce the counter overhead as branches are 6 times less frequent that virtual calls in the Smalltalk code we observe on production application. In addition, the counters provide information about basic block usage. Every finite loop requires a branch to stop the loop iteration and most recursive code requires a branch to stop the recursion, so the main cases for which we wanted to detect hot spots for are covered.

TODO why pinned array and not normal object like symbol literals ? => because direct access to object slot 
Using an immovable object results in 
- not having to add new method map entries
- not having to add new map/reference scanning machinery to update derived pointers (pointers into objects)
- the most direct access to the counter and hence the simplest possible instruction sequence (remember the AbstractInstrcution set is somewhat RISC like)
ENd TODO

To make the processor instruction-cache happy, counter values could not be next the machine code of each function. Indeed, doing so reuire the processor to flush the instruction-cache each time the counter is modified, slowing down massively code execution. To keep it efficient, at machine code compilation time a function is associated with a pinned array holding the counter values. The pinned property of the array allows the machine code to refer directly to the counter addresses, without having to do any garbage collection extension.


\subsection{Extended bytecode set}

To support unsafe operations, the bytecode set needed to be extended. B\'era and Miranda describes the extended bytecode set used \cite{Bera14a}. The extended bytecode set design relies on the assumption that only a small number of new bytecode instructions are needed for the baseline JIT to produce efficient machine code. Three main kind of instructions were added into the bytecode set:
\begin{itemize}
\item \textbf{Guards}: guards are used to ensure a specific object has a given type, else they trigger dynamic deoptimization.
\item \textbf{Object unchecked accesses}: normally variable-sized objects such as arrays or byte arrays require type and bounds checks to allow a program to access their fields. Unchecked access directly reads the field of an object without any checks.
\item \textbf{Unchecked arithmetics}: Arithmetic operations needs to check for the operand types to know what arithmetic operation to call (on integers, double, etc.). Unchecked operations are typed and do not need these check. In addition, unchecked operations do not do an overflow check and are converted efficiently to machine code conditional branches if followed by a conditional jump.
\end{itemize}

We are considering adding other unchecked operations in the future. For example, we believe instructions related to object creation or stores without the garbage collector write barrier could make sense.

As the optimized methods are represented as bytecodes, one could consider executing them using the bytecode interpreter. This is indeed possible and we explain later in section \ref{interpreter} that in very uncommon cases it can happen in our runtime. However, improving the performance to speed-up the bytecode interpreter or to speed-up the machine code generated using the baseline JIT as a back-end are two different tasks that may conflict with one another. We designed the bytecode set so the machine code generated by the baseline JIT as a back-end is as efficient as possible, not really considering the speed of the interpretation of those methods as they are almost never interpreted in our runtime.


\subsection{Call-backs}

As in our implementation the runtime optimizer and deoptimizer are implemented in Smalltalk and not in the virtual machine itself, we needed to introduce callbacks activated by the virtual machine to activate the optimizer and the deoptimizer. 

These callbacks use the reification of stack frames available for the debugger to inform the language which frame had its method detected as a hot spot and which frame has to be deoptimized.


\subsection{Machine code introspection}

%merge 2 part, §1 and pargraph 2-3
To extract information from the machine code version of a method, we added a new primitive operation \emph{sendAndBranchData}. This operation can be performed only on compiled methods. If the method has currently a machine code version, the primitive answers the types met at each inline cache and the values of the counters at each branch. This information can be then used by the runtime optimizer to type variables and to detect the usage of each basic block. The primitive answers the runtime information relative to the compiled method and all the closures defined in the compiled method.


A primitive operation was added in the language to extract the \emph{send and branch data} of the associated machine code of each function. It works as follows:
\begin{itemize}
\item If the function is present in the machine code zone, then it answers the \emph{send data}, which means the types met at each virtual call site based on the state of the inline caches, and the \emph{branch data}, which means the number of times each branch was taken at each branch.
\item If the function is not present in the machine code zone, then this primitive fails.
\end{itemize}

The data is answered as an array of array, with each entry being composed of:
\begin{itemize}
\item the bytecode program counter of the instruction.
\item either the types met and the function founds for virtual calls or the number of time each branch was taken for branches.
\end{itemize}

\subsection{Runtime optimiser and deoptimiser}

Explain simple behavior.

Maybe it makes sense to say Context are the same as objects.

deoptimizer 
Doc: my blog post + sista arch

%%%%%%%%%%%%%%%%%%%%%%%%%%%%%%%%%%%%%%%%%%%%%%%%%%%%%%%%%%%%%%%%%%%%%%%%%%%%%%%%%%%%%%%%%%%%%%%%%%%%%%%%%%%%%%%%%%%%%%%%%%%%%%%%%%%%%%%%
%%%%%%%%%%%%%%%%%%%%%%%%%%%%%%%%%%%%%%%%%%%%%%%%%%%%%%%%%%%%%%%%%%%%%%%%%%%%%%%%%%%%%%%%%%%%%%%%%%%%%%%%%%%%%%%%%%%%%%%%%%%%%%%%%%%%%%%%

\section{Optional language evolutions}

evolutions not required, but would be nice to get performance.

\subsection{New memory manager}

%I NEED TO CITE THE SPUR PAPER HERE.

The first problem met was related to the complexity of the existing memory manager (called V3) in the virtual machine. The memory manager is responsible for the representation of object in memory, as well as object allocation and garbage collection. The main issue lied with memory representation of objects: V3 was designed for an interpreter-based VM running heaps of several Mbs at most and it did not allow the JIT to generate efficient machine code nor scaling to large heap of objects.

\paragraph{Efficient JIT compilation.} Let's take a simple example. With V3, an object can access to its class in three different ways. Firstly, immediate objects\footnote{An immediate object is an object directly encoded in the object pointer, typically tagged integers.} access their classes through a special table, a list of 16 common classes have their instances access their class through another special table, while the rest of the objects have a pointer to their class in their header. In machine code, efficient class access becomes critical for efficient type tests (typically deoptimisation guards) and for inline caches, used in the baseline JIT to collect virtual calls receiver types. With V3, each class access in machine code requires to compile the three different paths. 

\paragraph{Heap scaling.} The second main issue was related to heap scaling. Let's take another simple example. V3 memory manager requests 1 Gb of memory to the operating system at start-up to keep all the heap contiguous, instead of segmenting its heap in smaller portions. This is very problematic in multiple OS such as Windows, where the OS may refuse to provide such a large amount of memory. 

(CITE Spur paper)
\cite{Mir15a}

Spur solutions.

pinned objects

Doc: spur paper

\subsection{New bytecode set}
New bytecode set needs to be discussed (not only extensions).

\subsection{Register allocation}

Linear scan and reg alloc, plus branches
Doc: TODO

\subsection{Read-only objects}

%I NEED TO CITE THE WRITEBARRIER PAPER HERE.

\cite{Bera16b}

One of the main problem encountered while trying to improve the performance of the Pharo runtime was literal mutability. In most programming languages, if the program executes a simple double addition between two double constants the compiler can compute at compile-time the result. In Pharo, as literals are mutable, one of the double constant may be accessed through reflective APIs and mutated into another boxed double, invalidating the compile-time result. 

To solve the problem, I introduced a feature, called read-only objects. With this feature, a program can mark any object as read-only. Such read-only objects cannot be mutated unless the program explicitly revert them to a writable state. This feature was introduced with little to no overhead on the production VM (CITE). 

Literals can now all be read-only objects, and any attempt to mutate a read-only literal is caught and, if the literal was used for compile-time computation, dependant optimised code is discarded if the mutation happens. This way, traditional compiler optimisation can be applied to the Pharo runtime.

\subsection{Closure implementation}


The second major issue encountered was related to the closure implementation. 

FullBlock
Doc: ? well... the FullBlock talk

%%%%%%%%%%%%%%%%%%%%%%%%%%%%%%%%%%%%%%%%%%%%%%%%%%%%%%%%%%%%%%%%%%%%%%%%%%%%%%%%%%%%%%%%%%%%%%%%%%%%%%%%%%%%%%%%%%%%%%%%%%%%%%%%%%%%%%%%
%%%%%%%%%%%%%%%%%%%%%%%%%%%%%%%%%%%%%%%%%%%%%%%%%%%%%%%%%%%%%%%%%%%%%%%%%%%%%%%%%%%%%%%%%%%%%%%%%%%%%%%%%%%%%%%%%%%%%%%%%%%%%%%%%%%%%%%%

\section{Conclusion}


\ifx\wholebook\relax\else
    \end{document}
\fi
\ifx\wholebook\relax\else

% --------------------------------------------
% Lulu:

    \documentclass[a4paper,12pt,twoside]{../includes/ThesisStyle}

	\usepackage[T1]{fontenc} %%%key to get copy and paste for the code!
%\usepackage[utf8]{inputenc} %%% to support copy and paste with accents for frnehc stuff
\usepackage{times}
\usepackage{ifthen}
\usepackage{xspace}
\usepackage{alltt}
\usepackage{latexsym}
\usepackage{url}            
\usepackage{amssymb}
\usepackage{amsfonts}
\usepackage{amsmath}
\usepackage{stmaryrd}
\usepackage{enumerate}
\usepackage{cite}
%\usepackage[pdftex,colorlinks=true,pdfstartview=FitV,linkcolor=blue,citecolor=blue,urlcolor=blue]{hyperref}
\usepackage{xspace}
%\usepackage{graphicx}
\usepackage{subfigure}
\usepackage[scaled=0.85]{helvet}
        
        
\newcommand{\sepe}{\mbox{>>}}
\newcommand{\pack}[1]{\emph{#1}}
\newcommand{\ozo}{\textsc{oZone}\xspace}
\newcommand\currentissues{\par\smallskip\textbf{Current Issues -- }}

\newboolean{showcomments}
\setboolean{showcomments}{true}
\ifthenelse{\boolean{showcomments}}
  {\newcommand{\bnote}[2]{
	\fbox{\bfseries\sffamily\scriptsize#1}
    {\sf\small$\blacktriangleright$\textit{#2}$\blacktriangleleft$}
    % \marginpar{\fbox{\bfseries\sffamily#1}}
   }
   \newcommand{\cvsversion}{\emph{\scriptsize$-$Id: macros.tex,v 1.1.1.1 2007/02/28 13:43:36 bergel Exp $-$}}
  }
  {\newcommand{\bnote}[2]{}
   \newcommand{\cvsversion}{}
  } 


\newcommand{\here}{\bnote{***}{CONTINUE HERE}}
\newcommand{\nb}[1]{\bnote{NB}{#1}}
\newcommand{\fix}[1]{\bnote{FIX}{#1}}
%%%% add your own macros 

\newcommand{\sd}[1]{\bnote{Stef}{#1}}
\newcommand{\ja}[1]{\bnote{Jannik}{#1}}
\newcommand{\na}[1]{\bnote{Nico}{#1}}
%%% 


\newcommand{\figref}[1]{Figure~\ref{fig:#1}}
\newcommand{\figlabel}[1]{\label{fig:#1}}
\newcommand{\tabref}[1]{Table~\ref{tab:#1}}
\newcommand{\layout}[1]{#1}
\newcommand{\commented}[1]{}
\newcommand{\secref}[1]{Section \ref{sec:#1}}
\newcommand{\seclabel}[1]{\label{sec:#1}}

%\newcommand{\ct}[1]{\textsf{#1}}
\newcommand{\stCode}[1]{\textsf{#1}}
\newcommand{\stMethod}[1]{\textsf{#1}}
\newcommand{\sep}{\texttt{>>}\xspace}
\newcommand{\stAssoc}{\texttt{->}\xspace}

\newcommand{\stBar}{$\mid$}
\newcommand{\stSelector}{$\gg$}
\newcommand{\ret}{\^{}}
\newcommand{\msup}{$>$}
%\newcommand{\ret}{$\uparrow$\xspace}

\newcommand{\myparagraph}[1]{\noindent\textbf{#1.}}
\newcommand{\eg}{\emph{e.g.,}\xspace}
\newcommand{\ie}{\emph{i.e.,}\xspace}
\newcommand{\ct}[1]{{\textsf{#1}}\xspace}


\newenvironment{code}
    {\begin{alltt}\sffamily}
    {\end{alltt}\normalsize}

\newcommand{\defaultScale}{0.55}
\newcommand{\pic}[3]{
   \begin{figure}[h]
   \begin{center}
   \includegraphics[scale=\defaultScale]{#1}
   \caption{#2}
   \label{#3}
   \end{center}
   \end{figure}
}

\newcommand{\twocolumnpic}[3]{
   \begin{figure*}[!ht]
   \begin{center}
   \includegraphics[scale=\defaultScale]{#1}
   \caption{#2}
   \label{#3}
   \end{center}
   \end{figure*}}

\newcommand{\infe}{$<$}
\newcommand{\supe}{$\rightarrow$\xspace}
\newcommand{\di}{$\gg$\xspace}
\newcommand{\adhoc}{\textit{ad-hoc}\xspace}

\usepackage{url}            
\makeatletter
\def\url@leostyle{%
  \@ifundefined{selectfont}{\def\UrlFont{\sf}}{\def\UrlFont{\small\sffamily}}}
\makeatother
% Now actually use the newly defined style.
\urlstyle{leo}



	\input{../includes/formatAndDefs}

	\graphicspath{{.}{../figures/}}
	\begin{document}
\fi

\chapter{Runtime evolutions}
\label{chap:runtimeEvolution}
\minitoc

%Generic intro
To support the architecture described in the previous chapter, the Pharo runtime had to evolve. We distinguish two kind of evolutions. Some evolutions were required to support the architecture, the Sista runtime could not work without those features. Other evolutions were not mandatory, Sista could have worked without these features, but each of them were important to improve the overall performance.

%%%%%%%%%%%%%%%%%%%%%%%%%%%%%%%%%%%%%%%%%%%%%%%%%%%%%%%%%%%%%%%%%%%%%%%%%%%%%%%%%%%%%%%%%%%%%%%%%%%%%%%%%%%%%%%%%%%%%%%%%%%%%%%%%%%%
%%%%%%%%%%%%%%%%%%%%%%%%%%%%%%%%%%%%%%%%%%%%%%%%%%%%%%%%%%%%%%%%%%%%%%%%%%%%%%%%%%%%%%%%%%%%%%%%%%%%%%%%%%%%%%%%%%%%%%%%%%%%%%%%%%%%

\section{Required language evolutions}

Five major evolutions were required to have Sista up and running:
\begin{enumerate}
	\item Cogit was extended to detect hot spots through profiling counters.
	\item The interpreter and Cogit were extended to be able to execute and compile the additional instructions of the extended bytecode set.
	\item Two VM call-backs were added to trigger Scorch when a hot spot is detected or a guard fails.
	\item A new primitive was introduced to provide runtime information in Smalltalk for v-functions having a corresponding n-function generated by Cogit.
	\item Scorch, including the optimiser, the deoptimiser and the dependency manager were introduced.
\end{enumerate}

\subsection{Profiling counters}

To detect hot spots, Cogit was extended to be able to generate profiling counters when generating a n-function. When the execution flow reaches a counter, it increases its value by one. If the counter reaches a threshold, the VM triggers a special call-back to activate Scorch. Profiling counters induce overhead, which can be significant enough to be seen in some benchmarks (the overhead is detailed on a set of benchmarks in the validation chapter, Chapter \ref{chap:validation}). To avoid the overhead in optimised code, Cogit was extended to support conditional compilation. Based on a bit in a v-function's header, Cogit generates a n-function with or without profiling counters.

Based on \cite{Arn02}, we added profiling counters on conditional branches, with one counter just before and one counter just after the conditional branch. This strategy allows the VM to provide basic block usage information in addition to the detection of hot spots. Every finite loop requires a branch to stop the loop iteration and most recursive code requires a branch to stop the recursion, so the main cases for which we wanted to detect hot spots for are covered. Each time the execution flow reaches a conditional branch in a n-function, it increases the profiling counter by one, compares the counter value to a threshold and jumps to the hot spot detection routine if the threshold is reached. If the threshold is not reached, the conditional branch is performed. If the branch is not taken, a second counter is incremented by one to provide later basic block usage information.

The main issue we had to deal with when implementing profiling counters is the location of the counters and the access to the counters. Indeed, in our first naive implementation, the counter values were directly inlined in the native code. That was a terrible idea as every write near executable code flushes part of the processor instruction cache, leading to horrible performance. In the end, we changed the logic to allocate a pinned unsigned 32-bits integer array\footnote{A pinned object is an object that can never be moved in memory. For example, the garbage collector cannot move it.} for each n-function requiring counters. The pinned array is on heap, far from executable code, and contains all the counter values. As the array is pinned, the native code can access the array and each of its fields (each counter) through a constant address. This is very nice as the native code can be efficient by using constant addresses and the n-function does not require any metadata\footnote{References to non pinned objects from n-function normally require metadata to update the reference when the object is moved in memory, typically by the garbage collector.}.

\begin{figure}[h!]
    \begin{center}
        \includegraphics[width=0.8\linewidth]{ProfilingCounters}
        \caption{Unoptimised n-function with two profiling counters}
        \label{fig:ProfilingCounters}
    \end{center}
\end{figure}

Figure \ref{fig:ProfilingCounters} shows a n-function with two profiling counters. The n-function is present in the n-function zone, which is readable, writable and executable. The n-function zone is exclusively modified by Cogit. The pinned array is allocated on heap, where all objects are present, which is a readable and writable (but not executable) zone. The n-function is composed of a header, which encodes different properties of the n-function such as its size, the n-function native code and metadata to be able to introspect the n-function. 

As the n-function from Figure \ref{fig:ProfilingCounters} requires two counters, an array with two 32 bits wide fields is allocated on heap. Each 32 bits counter field is split in two. The high 16 bits are used for the counter just before the conditional branch, precising how many times the branch has been reached. The low 16 bits are used for the counter just after the branch, precising how many times the branch was not taken. The native code has direct references to the counter addresses. The n-function header has also a reference to the pinned array, not shown on the figure to avoid confusion, which is used to reclaim the pinned array memory when the n-function is garbage collected. In the example, we can see that the first branch is always taken (the counter increased when the branch is not taken is at 0 while the branch has been reached 61285 times).

\subsection{Extended bytecode set}

Our architecture requires an extended bytecode set to support all the new operations permitted only in optimised v-functions. The new operations were described in one of our paper~\cite{Bera14a}. The extended bytecode set design relies on the assumption that only a small number of new instructions are needed for Cogit to produce efficient machine code. Four main kind of instructions were introduced:
\begin{itemize}
\item \textbf{Guards}: Guards are used to ensure an optimisation-time assumption is valid at runtime. If the guard fails at runtime, the dynamic deoptimisation routine is triggered.
\item \textbf{Object unchecked accesses}: Normally variable-sized objects such as arrays or byte arrays require bound checks to allow a program to access their fields. Unchecked access directly reads the field of an object without any checks. Instructions to access the size of a variable-sized object without any checks are included.
\item \textbf{Unchecked arithmetics}: Arithmetic operations need to check for the operand types to know what arithmetic operation to execute (integer operation, double operation, etc.). Unchecked operations are typed and do not need these check. In addition, unchecked operations do not perform any overflow check.
\item \textbf{Unchecked object allocation and stores}: Normal object allocations do many different things in addition to memory allocation, such as the initialization of all fields to \ct{nil} which is not needed if all fields are set immediately after to other values. Normal stores into objects go through a write barrier to make sure that the store does not break any garbage collector invariant. Such a write barrier can be ignored in specific cases, for example when doing a store in an object that has just been allocated, because it is guaranteed to be in the young object space.
\end{itemize}

As discussed in the previous chapter, optimised v-functions may be interpreted but we made sure that their interpretation is very uncommon. We designed the unsafe operations to be efficient when an optimised n-function is generated and executed. We did not design the unsafe operations for efficient interpretation. 

Designing the operations for efficient native code generation is quite different from designing them for efficient interpretation. For efficient native code generation, we encoded the unsafe operations in multiple bytes to be able to provide extra information to Cogit on how to produce good native code for the instruction. This strategy is not very good for the interpreter as it needs additional time to decode the multiple bytes. A good design for efficient interpretation would have been to encode performance critical instructions in a single byte.

%As the optimised methods are represented as bytecodes, they can potentially be executed by the VM interpreter. However, as discussed in the previous chapter, we made sure that it was very uncommon. Improving the performance to speed-up the bytecode interpreter or to speed-up the machine code generated using Cogit are two different tasks that may conflict with one another. We designed the extended bytecode set so the machine code generated by Cogit is as efficient as possible, not really considering the speed of the interpretation of such v-functions.

\subsection{Call-backs}

As Scorch optimiser and deoptimiser are written in Pharo while the rest of the VM is written in Slang, the VM needs a special way to activate them. 

Pharo has an array of registered objects which can be accessed both from the VM and the language, as described in Section \ref{par:regObjects}. We registered two new selectors. One is activated by the VM when a hot spot is detected in an unoptimised n-function. The other one is activated when a guard failed in an optimised n-function. A method is implemented in Smalltalk for each selector in the class used for reified stack frames. In our case, these methods activate respectively Scorch optimiser or Scorch deoptimiser.

\subsection{Machine code introspection}

%what does the primitive + intro
To extract runtime information from a n-function, we added a new primitive method called \emph{sendAndBranchData}. SendAndBranchData is activated with no arguments and fails if the receiver is not a v-function. If the v-function was compiled by Cogit and therefore has an associated n-function, the primitive answers runtime information present for the function, if the v-function was not compiled, the primitive fails. The runtime information includes types and functions met in each inline cache and the profiling counter values. This information can be then used by Scorch to speculate on types and basic block usage. 

%reading the cache/counter through 
Cogit had an introspection API used for multiple features such as inline cache relinking or debugging. The cache and counter values are read by the implementation of a small extension on top of Cogit API for introspection. In unoptimised n-functions, Cogit generates an inline cache for each virtual call~\cite{Deut84a,Holz91a}, relinked at runtime each time a new receiver type is met for the call. The caches can be read using low-level platform-specific code, similar to the code used for relinking. Cogit generates for each profiling counter a call to the hot spot detection Slang routine, used to trigger the hot spot detection routine. This call is annotated with the corresponding virtual instruction pointer, allowing Scorch to know to which conditional branch in the v-function each profiling counters correspond to.

%Result of primitive
The new primitive iterates over the n-function, collecting for each virtual call and each conditional branch the virtual instruction pointer as well as respectively type and profiling information. Because the data is collected from Slang, it's not convenient to build complex data structures using multiple different kind of objects (Each object's class internal representation would need to be specifically known by both the VM and Scorch). To keep things simple, the primitive answers an array of arrays, each inner array containing the virtual program counter of the instruction, and a list of types and v-functions targetted by the inline cache or the number of times each branch has been taken.

\subsection{Scorch}

In the implementation of our architecture, the bulk of the work is the design and the implementation of Scorch optimiser. The thesis is centered around Sista in general and there are no big innovative features in Scorch.

Scorch optimiser is implemented as a traditional optimising compiler. It translates the v-functions to optimise into a single static assignment intermediate representation, represented as a control flow graph. Conditional branch and send instructions are annotated with the runtime information provided by Cogit to speculate on what method to inline and on basic block usage.

The optimisations performed are very similar to the ones performed in other optimising JITs such as V8 Crankshaft optimising JIT~\cite{V8}. Scorch starts by speculating on types to inline other functions. Guards are inserted to ensure speculations are valid at runtime. Once inlining is done, Scorch performs multiple standard optimisations such as array-bounds check elimination with the ABCD algorithm~\cite{Bodi00a} or global value numbering. Scorch also attempts to postpone allocation of objects not escaping the optimised v-function from runtime to deoptimisation time (or completely removes the allocation if the object is never required for deoptimisation).

Scorch back-end generates an optimised v-function from the intermediate representation. The phi instructions from single static assignment are removed, using instead temporary locations assigned multiple times. For each instruction, Scorch determines if the computed value needs to be stored to be reused, and if so, if it can be stored as a spilled value on stack or as a temporary variable. The deoptimisation metadata is attached to the optimised v-function to be able to restore multiple frames.

Scorch deoptimiser is much simpler than the optimiser. It reads the deoptimisation metadata for a given program counter in the optimised v-function. The metadata consists of a list of objects to reconstruct, including closures and reified stack frames. The objects are reconstructed by reading constant values in the metadata or reading the optimised stack frame values. Once the objects are reconstructed, execution can resume in the restored bottom frame.

Scorch dependency manager is also much simpler than the optimiser. It keeps track for each optimised function of the list of selectors it depends on. If a method with one of these selectors is installed, all the optimised functions dependant are discarded. 

%%%%%%%%%%%%%%%%%%%%%%%%%%%%%%%%%%%%%%%%%%%%%%%%%%%%%%%%%%%%%%%%%%%%%%%%%%%%%%%%%%%%%%%%%%%%%%%%%%%%%%%%%%%%%%%%%%%%%%%%%%%%%%%%%%%%
%%%%%%%%%%%%%%%%%%%%%%%%%%%%%%%%%%%%%%%%%%%%%%%%%%%%%%%%%%%%%%%%%%%%%%%%%%%%%%%%%%%%%%%%%%%%%%%%%%%%%%%%%%%%%%%%%%%%%%%%%%%%%%%%%%%%

\section{Optional language evolutions}


Five major evolutions were introduced in the language in addition to the required evolutions to allow Scorch to produce more efficient optimised v-functions:
\begin{enumerate}
	\item A new memory manager for efficient n-function generation.
	\item A new bytecode set to leverage encoding limitations.
	\item A register allocation algorithm in Cogit.
	\item A write barrier feature to be able to mark objects as read-only.
	\item A new closure implementation to be able to optimise closure more efficiently.
\end{enumerate}

\subsection{New memory manager}

%pb with existing representation
The first version of Sista was built on the existing VM with a minimum number of modifications. One of the main issues met was related to the memory representation of objects. The existing memory manager was designed and implemented before the implementation of Cogit for a pure interpreter VM~\cite{Inga97a}. The representation of objects did not allow Cogit to produce efficient accesses to object fields in native code.

%existing class access
One problem for example was the encoding of the class field in an object. It could be encoded in three different ways:
\begin{itemize}
	\item \emph{Immediate classes:} A very limited subset of classes, included \ct{SmallInteger}, have their instances encoded in the pointer to the object itself. As all objects are aligned in memory for efficient access to their fields, the last few bits (the exact number depends on the alignment) of a pointer to an object are never set. By setting some of these last few bits, the memory manager can encode a class identifier. \ct{SmallInteger} for example are encoded by setting the last bit of the pointer.
	\item \emph{Compact classes:} A limited set of classes, up to 15 classes, had their instances encoding their classes as an index in a 4-bits field in the first word of the object's header. The memory manager had access to an array mapping the indexes to the actual classes.
	\item \emph{Other classes:} All the other instances encoded their class as a pointer to the class object, encoded in an extra pointer-sized field in the header of the object.
\end{itemize}

%existing class access pb in n-functions
In practice, Cogit compiles many type-checks. In unoptimised n-functions, type-checks are generated mainly in inline caches. In optimised functions, type-checks are generated for deoptimisation guards. For each type-check, the native code generated needed three paths to find out which one of the three encodings was used for the instance which was type-checked, to finally compare it against the expected type. In addition, as many instances encoded their class as a pointer to the class object while class objects can be moved by the garbage collector in memory, cogit needed to annotate the expected type to correctly update the pointer value during garbage collection. Overall both the generated n-function and the garbage collector were slowed by the memory representation.

%Spur and problem solving
To solve this problem, a new memory manager was implemented and deployed in production~\cite{Mir15a}. The new representation of objects in memory allows the generation of very efficient n-functions. For example, there is now only two ways for an instance to access its class, the class is either immediate or compact. Compact class indexes are stored in the instances in a 22-bit fields, allowing over four millions different concrete classes. Cogit does not need any more to annotate type-checks when generating the n-function as an indirection index is referred instead of the class object.

%other pbs soled
In addition to the generation of efficient n-functions, other problems non directly related to the thesis were present in the existing memory manager (poor support for large heaps, slow scavenges, etc.) which were solved with the new memory manager.

% pinned objects ??? I don't know if it's worth going into details there.

\subsection{New bytecode set}

The existing Pharo bytecode set had multiple encoding limitations~\cite{Bera14a}. For example, jumps (forward, backward and conditonnal) were able to jump over 1024 bytes at most. Such limitations are very rarely a problem while compiling normal Smalltalk code due to coding convention encouraging developers to write small functions. However, the optimised function produced by Scorch includes many inlined functions and in some case the limitations were a problem. As the bytecode set already needed to be extended to support the new unsafe operations, we designed a complete new bytecode set instead of just adding the new operations to leverage encoding limitations.

\subsection{Register allocation}

To allocate registers, Cogit simulates the stack state during compilation. When reaching an instruction using values on stack, Cogit uses a dynamic template scheme to generate native instructions. The simulated stack provides information such as which values are constants or already in registers. Based on this information, Cogit picks one of the available templates for the instruction, uses a linear scan algorithm to allocate registers that do not need to be fixed into specific concrete registers and generates native instructions.

The existing linear scan algorithm was very naive and limited. It was very efficient because registers are not live across certain instructions that are very common in unoptimised code. Specifically, registers cannot live across these three instructions:
\begin{enumerate}
	\item \emph{virtual calls:} All registers are caller-saved.
	\item \emph{backjumps:} Backjumps are interrupt points.
	\item \emph{Conditional branches:} If the branch is on a non-boolean, a slow path is taken to handle the case requiring to spill the registers.
\end{enumerate}

However, these instructions are not that common in optimised code. Most virtual calls are inlined. Some backjumps are annotated not to require an interrupt check. Some conditional branches are removed because one branch has never been used and others are annotated as branching on a value which is guaranteed to be a boolean. Registers can therefore stay live across many more instructions and the register allocation algorithm has more impact on native code quality.

We wrote a new linear scan register algorithm, performing better under register pressure. The most difficult part is to correctly keep registers live across conditional branches. At each control flow merge point, the register state has to be the same in both branches or Cogit needs to generate additional instructions to spill or move registers.

\subsection{Read-only objects}

One of the main problem encountered while trying to improve the performance of the optimised v-functions generated by Scorch was literal mutability. In most programming languages, if the program executes a simple double\footnote{We use \emph{double} to discuss double-precision floating-point in this paragraph.} addition between two double constants the compiler can compute at compile-time the result. In Pharo, as literals are mutable, one of the double constant may be accessed through reflective APIs and mutated into another double, invalidating the compile-time result. 

To solve the problem, we introduced read-only objects. With this feature, a program can mark any object as read-only. Such read-only objects cannot be modified unless the program explicitly reverts them to a writable state. Any attempt to modify a read-only object triggers a specific call-back in Smalltalk, similarly to the hot spot detection and guard failure call-backs. The modification failure routine can, for example, revert the object to a writable state, perform the modification and notify a list of subscribers that the modification happened. This feature was introduced with limited overhead to the existing runtime~\cite{Bera16b}. 

In Sista, literals are read-only objects by default. Any attempt to modify a read-only literal is caught by the runtime and Scorch is notified. If the literal was used for compile-time computation, corresponding optimised v-functions are discarded. Thanks to this technique, traditional compiler optimisations can be applied to Smalltalk.

\subsection{Closure implementation}

Another important problem encountered when implementing Scorch was related to the implementation of closures. The existing closures were implemented in a way that the closure's v-functions were inlined into their enclosing v-function. This led to multiple problems as it was difficult to optimise a v-function without having to rewrite the v-functions of all the closures that could be instantiated inside the v-function. This was increasing the complexity of the optimiser and required very expensive object manipulation at deoptimisation time to correctly remap all the v-functions of the closures created inside an optimised function. The closure implementation was also complicating the code of Cogit: Cogit could not compile a virtual method without compiling all the virtual functions of the closures the virtual method could compile. This induced significant complexity also to introspect n-functions and to activate closures. 

To solve these issues, we designed a new closure implementation. In the new implementation, closures have v-functions separated from their enclosing environment v-functions. Scorch can optimise independently methods and closures. We were able to reduce the complexity of Cogit, both when it is used as a baseline JIT and as a back-end for Scorch.

\begin{figure}[h!]
    \begin{center}
        \includegraphics[width=0.85\linewidth]{CompiledBlock}
        \caption{Old and new closure representation}
        \label{fig:CompiledBlock}
    \end{center}
\end{figure}

Figure \ref{fig:CompiledBlock} shows on the left the old closure function representation and on the right the new representation. In the old representation, the closure function literals and bytecodes are present inside the enclosing function. The creation of a closure in the v-function interpreter requires to jump over the closure instructions. All the function metadata, normally present in the function header, can be computed by disassembling the closure creation instruction or the closure bytecodes. Any change on the closure function impacts the enclosing function and any change on the enclosing function impacts the closure function. When Cogit compiles the v-function to n-function, it compiles both the function and all the inner closure functions at the same time. In the new representation, the two functions are independent. Each function has its own function header. Both functions can be changed independently. Cogit compiles separatedly each v-function to n-function. The closure function has no source pointer as it can be fetched from the enclosing function (the last literal of the compiled block).

\section*{Conclusion}

This chapter described the runtime evolutions required in the Pharo runtime to support Sista. The next chapter discusses how Scorch is able to optimise its own code and under which constraints.

\ifx\wholebook\relax\else
    \end{document}
\fi
\ifx\wholebook\relax\else

% --------------------------------------------
% Lulu:

    \documentclass[a4paper,12pt,twoside]{../includes/ThesisStyle}

	\usepackage[T1]{fontenc} %%%key to get copy and paste for the code!
%\usepackage[utf8]{inputenc} %%% to support copy and paste with accents for frnehc stuff
\usepackage{times}
\usepackage{ifthen}
\usepackage{xspace}
\usepackage{alltt}
\usepackage{latexsym}
\usepackage{url}            
\usepackage{amssymb}
\usepackage{amsfonts}
\usepackage{amsmath}
\usepackage{stmaryrd}
\usepackage{enumerate}
\usepackage{cite}
%\usepackage[pdftex,colorlinks=true,pdfstartview=FitV,linkcolor=blue,citecolor=blue,urlcolor=blue]{hyperref}
\usepackage{xspace}
%\usepackage{graphicx}
\usepackage{subfigure}
\usepackage[scaled=0.85]{helvet}
        
        
\newcommand{\sepe}{\mbox{>>}}
\newcommand{\pack}[1]{\emph{#1}}
\newcommand{\ozo}{\textsc{oZone}\xspace}
\newcommand\currentissues{\par\smallskip\textbf{Current Issues -- }}

\newboolean{showcomments}
\setboolean{showcomments}{true}
\ifthenelse{\boolean{showcomments}}
  {\newcommand{\bnote}[2]{
	\fbox{\bfseries\sffamily\scriptsize#1}
    {\sf\small$\blacktriangleright$\textit{#2}$\blacktriangleleft$}
    % \marginpar{\fbox{\bfseries\sffamily#1}}
   }
   \newcommand{\cvsversion}{\emph{\scriptsize$-$Id: macros.tex,v 1.1.1.1 2007/02/28 13:43:36 bergel Exp $-$}}
  }
  {\newcommand{\bnote}[2]{}
   \newcommand{\cvsversion}{}
  } 


\newcommand{\here}{\bnote{***}{CONTINUE HERE}}
\newcommand{\nb}[1]{\bnote{NB}{#1}}
\newcommand{\fix}[1]{\bnote{FIX}{#1}}
%%%% add your own macros 

\newcommand{\sd}[1]{\bnote{Stef}{#1}}
\newcommand{\ja}[1]{\bnote{Jannik}{#1}}
\newcommand{\na}[1]{\bnote{Nico}{#1}}
%%% 


\newcommand{\figref}[1]{Figure~\ref{fig:#1}}
\newcommand{\figlabel}[1]{\label{fig:#1}}
\newcommand{\tabref}[1]{Table~\ref{tab:#1}}
\newcommand{\layout}[1]{#1}
\newcommand{\commented}[1]{}
\newcommand{\secref}[1]{Section \ref{sec:#1}}
\newcommand{\seclabel}[1]{\label{sec:#1}}

%\newcommand{\ct}[1]{\textsf{#1}}
\newcommand{\stCode}[1]{\textsf{#1}}
\newcommand{\stMethod}[1]{\textsf{#1}}
\newcommand{\sep}{\texttt{>>}\xspace}
\newcommand{\stAssoc}{\texttt{->}\xspace}

\newcommand{\stBar}{$\mid$}
\newcommand{\stSelector}{$\gg$}
\newcommand{\ret}{\^{}}
\newcommand{\msup}{$>$}
%\newcommand{\ret}{$\uparrow$\xspace}

\newcommand{\myparagraph}[1]{\noindent\textbf{#1.}}
\newcommand{\eg}{\emph{e.g.,}\xspace}
\newcommand{\ie}{\emph{i.e.,}\xspace}
\newcommand{\ct}[1]{{\textsf{#1}}\xspace}


\newenvironment{code}
    {\begin{alltt}\sffamily}
    {\end{alltt}\normalsize}

\newcommand{\defaultScale}{0.55}
\newcommand{\pic}[3]{
   \begin{figure}[h]
   \begin{center}
   \includegraphics[scale=\defaultScale]{#1}
   \caption{#2}
   \label{#3}
   \end{center}
   \end{figure}
}

\newcommand{\twocolumnpic}[3]{
   \begin{figure*}[!ht]
   \begin{center}
   \includegraphics[scale=\defaultScale]{#1}
   \caption{#2}
   \label{#3}
   \end{center}
   \end{figure*}}

\newcommand{\infe}{$<$}
\newcommand{\supe}{$\rightarrow$\xspace}
\newcommand{\di}{$\gg$\xspace}
\newcommand{\adhoc}{\textit{ad-hoc}\xspace}

\usepackage{url}            
\makeatletter
\def\url@leostyle{%
  \@ifundefined{selectfont}{\def\UrlFont{\sf}}{\def\UrlFont{\small\sffamily}}}
\makeatother
% Now actually use the newly defined style.
\urlstyle{leo}



	\input{../includes/formatAndDefs}

	\graphicspath{{.}{../figures/}}
	\begin{document}
\fi

\chapter{Metacircular optimising JIT}
\label{chap:metacircular}
\minitoc

%Intro
By design, Scorch's optimiser and deoptimiser are written in Smalltalk and are running in the same runtime than the optimised application. This design leads to recursion problems similar to the ones existing in metacircular virtual machines, detailed in this chapter. 

%single-threaded -> for comparison with Graal and co later
As Pharo is currently single-threaded, it is not possible to run Scorch in a concurrent native thread. To optimise code, Scorch requires either to temporarily interrupt the application green thread or to postpone the optimisation to a background-priority green thread as described in Section \ref{sec:optModes}. The deoptimiser cannot however postpone the deoptimisation of a frame as it would block completely the running application. The deoptimiser has necessarily to interrupt the application green thread to deoptimise the stack.

%introduction of the main issue
Hot spots can be detected in any Smalltalk code using conditional branches, including Scorch optimiser code itself (as Scorch is written in Smalltalk). When a hot spot is detected in the optimiser code, the optimiser interrupts itself and starts to optimise one of its own functions. While doing so, the same hot spot may be detected again before the optimised function is installed, leading the optimiser to interrupt itself repeatedly. This problem is discussed in Section \ref{sec:optRec}.

A similar problem exists for the deoptimiser. As the Scorch deoptimiser is written in Smalltalk, its code base may get optimised. One of the optimisation-time speculation may be incorrect at runtime, leading the deoptimiser to require the deoptimisation of one of its own frames. In this case, the deoptimiser calls itself on one of its own frames, which may require to deoptimise another frame for the same function, leading the deoptimiser to call itself repeatedly. This second problem is detailed in Section \ref{sec:deoptRec}.

We call this issue where the optimiser or the deoptimiser calls itself repeatedly as the \emph{meta-recursion} problem. The expression meta-recursion comes from other works~\cite{Chib96a,Denk08b} where similar problems are present.

%Different constraint so different solutions
The optimiser and deoptimiser have different constraints. It is possible to disable temporarily the optimiser while the application is running. In the worst case, a disabled optimiser leads some functions not to be optimised, but the application keeps running correctly. However, the deoptimiser cannot be disabled temporarily while the application is running. Indeed, an application requiring deoptimisation cannot continue to execute code until the deoptimisation is performed. As the optimiser and the deoptimiser have different constraints, they need different solutions for the meta-recursion problem.

%Outline and solution
This chapter explains the design used to avoid the meta-recursion issue in both the optimiser and the deoptimiser. Section \ref{sec:optRec} describes how the problem is solved for the optimiser by temporarily disabling it in specific circumstances. Section \ref{sec:deoptRec} shows how the deoptimiser solves the problem by using a code base completely independent from the rest of the system that cannot be optimised, to never require to be deoptimised. Section \ref{sec:recRelW} discusses similar designs in other VMs and compares our solution to other solutions when relevant.

%%%%%%%%%%%%%%%%%%%%%%%%%%%%%%%%%%%%%%%%%%%%%%%%%%%%%%%%%%%%%%%%%%%%%%%%%%%%%%%%%%%%%%%%%%%%%%%%%%%%%%%%%%%%%%%%%%%%%%%%%%%%%

\section{Scorch optimiser}
\label{sec:optRec}

%Intro, what Scorch optimiser does and critical mode.
Scorch optimiser is activated by the VM when a hot spot is detected. As Pharo is single-threaded, the optimiser is activated by interrupting one of the application green threads. The optimiser chooses, based on the current stack, a v-function to optimise.  Once the v-function to optimise is chosen, the optimiser gets started in critical mode: it attempts to generate an optimised v-function in a limited time period. If it succeeds, the optimised v-function is installed and used by further calls on the function. If the optimiser fails to generate the optimised v-function in the limited time period, it adds the v-function to a background compilation queue. In any case, the application is then resumed. When the application becomes idle, if the background compilation queue is not empty, Scorch gets activated in background mode. It produces and installs optimised v-functions for each function in the compilation queue without any time limit. 

\subsection{Meta-recursion issue}

%repeat problem for optimiser
As Scorch optimiser is written in Smalltalk, it can theoretically optimise its own code. In practice, if it happens, it may lead to a meta-recursion. Indeed, each time Scorch tries to optimise a function, before reaching the point where it can install the optimised function, it may interrupt itself to start optimising one of its own functions. If a hot spot is detected in the optimiser code each time it attempts to optimise anything, then the optimiser never reaches the point where it can install an optimised function.

Figure \ref{fig:InfiniteRecursionOptPb} shows the problem. On the left, in the normal optimisation flow, the application is interrupted when a hot spot is detected. The optimiser generates an optimised v-function, installs it and the application resumes. On the right, in the meta-recursion issue, the application is also interrupted when a hot spot is detected. While the optimiser is generating an optimised function, a hot spot is detected in the optimiser code. The optimiser then restarts to optimise one of its own functions, but another hot spot (potentially the same one) is detected in the optimiser code: the optimiser keeps restarting the optimisation of one of its own function.

\begin{figure}[h!]
    \begin{center}
        \includegraphics[width=0.65\linewidth]{InfiniteRecursionOptPb}
        \caption{Meta-recursion problem during optimisation}
        \label{fig:InfiniteRecursionOptPb}
    \end{center}
\end{figure}

This problem has different consequences depending if the optimiser is started in critical or in background mode. In practice, the meta-recursion issue leads to a massive performance loss that we detail in the following paragraphs.

%slow down - critical
\paragraph{Critical mode.} In critical mode, the optimiser has a limited time period to optimise code. If the meta-recursion issue happens, the optimiser spins until the time period ends as shown in Figure \ref{fig:InfiniteRecursionOptPbCritical}. The application is then resumed without any optimised function installed. The application gets drastically slower as it gets interrupted for the full critical mode time period without gaining any performance from those interruptions.

\begin{figure}[h!]
    \begin{center}
		\subfigure[Critical mode]{\label{fig:InfiniteRecursionOptPbCritical}\includegraphics[width=0.3\linewidth]{InfiniteRecursionOptPbCritical}}
		\hspace{1cm}
		\subfigure[Background mode]{\label{fig:InfiniteRecursionOptPbBackground}\includegraphics[width=0.3\linewidth]{InfiniteRecursionOptPbBackground}}
		
		\subfigure{\includegraphics[width=0.3\linewidth]{InfiniteRecursionOptPbLegend}}
		\caption{Meta-recursion problem in the two optimisation modes}
    \end{center}
\end{figure}

%slow down - background
%It can conceptually happen that the optimiser starts spinning while searching the stack for a function to optimise. In this case, no function can be added to the background compilation queue as the optimiser has not been able to find a function to optimise in the limited time period. In practice, in our case, the stack search code is quite simple and in practice no hot spot can be detected repeatedly in this code.

\paragraph{Background mode.} When the application becomes idle, the optimiser is started in background mode to optimise functions in the background compilation queue. In this case, the optimiser always successfully generates and installs optimised functions. However, the optimisation of a function is very slow. Indeed, while the optimiser is running in background mode, it activates itself multiple times in critical mode when detecting hot spots in its own code. Each time it happens, the optimiser spins in critical mode for the time period allowed as shown on Figure \ref{fig:InfiniteRecursionOptPbBackground}. If the optimiser is started many times on itself during background optimisation, the optimisation of the function may take a significant amount of time. 

Eventually, the optimiser optimises most of its own code correctly through the background mode. Once done, both modes can work correctly as the optimiser cannot be triggered on already optimised code.

The problem is therefore that the application executed gets really slow at start-up because of time wasted spinning in critical mode. Peak performance takes a long time to reach because the optimiser successfully installs code only in background mode or when the meta-recursion issue does not happen in critical mode. 

\subsection{Current solution}

The first solution we implemented is to disable the optimiser when it is running. This way, the optimiser cannot optimise its own code any more and no meta-recursion can happen. This first design has a significant advantage: it is quite simple both conceptually and implementation-wise, while it completely avoids the meta-recursion problem. It has however a major drawback: the optimiser code cannot be optimised at all any more. Of course, the optimiser may use core libraries that can be optimised. For example, the optimiser uses collections such as arrays. If the application optimised is also using the same collections, and it is very likely that an application would use arrays, the array code base may get optimised. Then, the optimiser ends up using an optimised version of arrays. However, the code specific to the optimiser is not optimised.

%call-back removal / addition implementation
To implement our first solution, we changed the VM call-back activating the optimiser to uninstall itself upon activation. When a hot spot is detected but the call-back is not installed, the VM resets the profiling counters which has reached the hot spot threshold and restarts immediately the execution of the application. Then, we changed the optimiser to install back the call-back when it resumes the application, after installing optimised code or adding a function to the background queue. This way, we believed the optimiser would never end up in a situation where it optimises itself, solving entirely the problem.

%can optimise itself through back process
Then, we ran our benchmarks and saw that the problem was solved but the optimiser could still optimise its own code. Our first implementation effectively disabled the optimiser, but only when it was running on critical mode. When hot spots were detected, they were optimised or postponed without any issue as the optimiser disabled itself in critical mode, and the application resumed just fine. When the optimiser was started in background mode, it was not disabled. Hence, in this case, hot spots were detected in the optimiser code and the optimiser was sometimes interrupted by itself in critical mode to optimise its own code.

\begin{figure}[h!]
    \begin{center}
        \includegraphics[width=0.4\linewidth]{Disabling}
        \caption{Hot spot detection disabled in critical mode.}
        \label{fig:Disabling}
    \end{center}
\end{figure}

Figure \ref{fig:Disabling} shows the first solution. The application code can be optimised by Scorch, but Scorch cannot optimise its own code when run in critical mode as hot spot detection is disabled. When Scorch is started in background mode, then hot spots are detected (as in any application) and the optimiser code can get optimised.

%1st sol work and good enough 
This first solution is implemented, stable and works fine. Multiple benchmarks run with significant speed-up over the normal VM (This will be detailed in Chapter \ref{chap:validation}). In general, in our production VM, simplicity is really important to keep the code base relatively easy to maintain. For each added complexity in the VM we evaluate if the complexity is worth the benefit. This first solution is very valuable to us because it is fairly simple to understand and to maintain. Hence, the optimising JIT may move to production with this design. The next section discusses alternative solutions, which are more complex but allow, at least partially, the optimiser to optimise its own code when run in critical mode.

\subsection{Discussion and advanced solutions}

%but cannot optimise itself in critical mode
The first solution is working but has one major drawback: Scorch optimiser cannot optimise itself in critical mode. Indeed, hot spots detected inside the optimiser in critical mode are completely ignored and the corresponding profiling counters are reset. If the optimiser attempts to optimise code later, it may get confused by some counter values which were reset (basic block usage is incorrectly inferred in this case, in the worst case, a branch may be speculated as unused whereas it is frequently used). 

\paragraph{Decay strategy.} Instead of resetting entirely the counters, we could implement a decay strategy, by for example dividing the current counter values by two. We did not go in this direction because the counters are currently encoded in 16 bits while the hot spot threshold is set to the maximum value. Due to the 16 bits encoding limitation, not completely resetting the counters leads to the detection of many hot spots, repeatedly, without any optimisation happening slowing down the optimiser at start-up. Further analysis in this direction are required to conclude anything.

%the postpone pb while in critical mode - Saving current stack impossible
\paragraph{Partial disabling.} \label{par:PartialDisabing} Another naive approach is to postpone the optimisation to background mode when the meta-recursion issue happens in critical mode. In our design, it is quite difficult to do so. Indeed, when the VM call-back starts the optimiser, it provides only a reification of the current stack as detailed in Section \ref{ss:stackSearch}. The optimiser then needs to search the stack to select a function to optimise, and only then it can add a function to optimise to the background compilation queue. 

The stack is modified upon execution and may reference a very large graph of objects, so it is very difficult to save it efficiently for the optimiser to search it later in the background green thread. In addition, as discussed in Section \ref{ss:stackSearch}, there is no simple and quick heuristic to figure out the best function to optimise based on the current stack. 

It is however possible, once the optimiser has found what function to optimise, to add it to the background compilation queue. Therefore, we believe that instead of disabling the entire optimiser while it is running in critical mode, we could instead disable it only during the stack searching phase in critical mode and postpone the optimisation to the background green thread instead if the function to optimise has already been found. This way, only hot spots detected during the stack search would be ignored, while the rest of the optimiser would be optimised at the next idle pause. As the stack search phase represents less than 1\% of the optimiser execution time, this approach looks very promising.

Figure \ref{fig:PartialDisabling} shows the solution proposed. When a hot spot is detected, Scorch is activated on a stack to optimise and starts by searching a function to optimise. During this phase, the optimiser is disabled to avoid the meta-recursion issue. Once the function to optimise is found, Scorch optimises it. During this phasis, if a hot spot is detected, the optimiser searches a function to optimise and directly appends it to the background compilation queue. Once the function is optimised, the optimised v-function is installed and the application can resume.

\begin{figure}[h!]
    \begin{center}
        \includegraphics[width=0.55\linewidth]{PartialDisabling}
        \caption{Partial disabling of the optimiser.}
        \label{fig:PartialDisabling}
    \end{center}
\end{figure}

\paragraph{Ahead-of-time optimisation.} 
Alternatively, we could consider optimising the Scorch optimiser code ahead-of-time.

As Sista allows one to persist optimised code (this is discussed in details in the following chapter, Chapter \ref{chap:persistence}), the optimiser code could be optimised ahead of time. The optimised optimiser code would be shipped to production, and no runtime optimisation would happen on the optimiser code in production.

To generate the optimised code ahead-of-time, one way is to preheat the optimiser through warm-up runs. For example, the optimiser can be given a list of well-chosen functions to optimise. This way, all hot spots inside the optimiser would be detected ahead of time and optimised. 

Alternatively, the optimiser's code could be optimised statically by calling itself on its own code, using types inferred from a static type inferencer instead of types inferred from the runtime. This solution has a significant cost in our case as we have to implement and maintain a library to infer types.

%\subsection{Dependencies and optimisations} 

% from 1 old part
%Rephrase -  solution does not forbig Scorch to use optimise libs

% maybe only one sentence ? -> be careful about dependencies
%The first constraint to note when programming Scorch, which may be obvious to the Kernel programmer, is that Scorch cannot depends on any framework or library but the Kernel and Core libraries. Each framework or library in the system relies on the execution engine to perform its code. Scorch is part of the execution engine. Hence, if Scorch relies on an external library and that someone modifies the library, the execution engine may not be stable any more and the runtime completely crashes. In fact, all the Kernel code and Core librairies have similar constraints, they cannot rely on anything to keep the system modular. 

%While writting Scorch, we needed a tool to compress the deoptimisation metadata generated aside from the optimised code. We wanted to use the standard Pharo serializer, Fuel (CITE), but we were not able to do it or further modification on Fuel would break the execution engine.

%In the end, we limited the dependencies of Scorch to the Pharo Kernel and the core collections (exactly: Set, OrderedCollection, Array, ByteArray and Dictionary in addition to the kernel). Any change on one of this dependency may require to change something in Scorch to keep the system running.

%\subsection{Debugging and runtime modification}

%Should I talk about that at all ? I was thinking over a restricted compiling to C but maybe we don't care.

%Maybe rewrite so that formally does not work but in practice it does.

%As any Smalltalk program, it is possible to modify the optimiser while it is running, for example in the debugger. If the modifications leads to incorrect optimiser behavior, then the runtime may crash. To avoid crashes, it may be wise to disable the optimiser while editing it. In practice, this feature is used only by the optimiser implementors. It is very useful to debug the optimiser to understand specific bugs or compiler decisions. With careful understanding of the infrastructure, it is possible in practice to debug the optimiser while it is running and modify its code. The optimiser is set by default to catch all exceptions, failing the optimisation of a specific v-function if an exception was raised. Hence, if the code modification triggers a compile-time exception, the system shall not crash. Unfortunately, in some cases, the optimiser may have silent errors, generating incorrect code without raising exceptions and completely crashing the system.

%The only part of Scorch that cannot really be edited is the deoptimisation metadata generation. Indeed, deoptimisation metadata is also used by the deoptimiser which, as detailed in the following section, has stronger constraints on its code. If one modifies the deoptimisation metadata generated, the deoptimiser may not be able to deoptimise correctly optimised code any more, leading to crashes.

%%%%%%%%%%%%%%%%%%%%%%%%%%%%%%%%%%%%%%%%%%%%%%%%%%%%%%%%%%%%%%%%%%%%%%%%%%%%%%%%%%%%%%%%%%%%%%%%%%%%%%%%%%%%%%%%%%%%%%%%%%%%%

%KEEP WRITING FROM HERE

\section{Scorch deoptimiser}
\label{sec:deoptRec}

The deoptimiser can be activated in multiple situations. If an optimisation time assumption is invalid at runtime, a deoptimisation guard fails and Cogit triggers a call-back to deoptimise the stack. In addition, multiple development tools in the language, such as the debugging tools, may call the deoptimiser to introspect the stack.

\subsection{Meta-recursion issue}

% repeat problem for deopt
As Scorch deoptimiser is written in Smalltalk, its code base may get optimised. If one of the optimisation-time speculation is incorrect at runtime, the optimised frame requires the deoptimiser to restore the non-optimised stack frames to continue the execution of the program. 

In the case where the deoptimiser functions are optimised, the deoptimiser may call itself on one of its own frames to be able to continue deoptimising code. If at each deoptimisation the deoptimiser calls itself on one of its own frames, the deoptimisation will never terminate because the deoptimiser needs to call itself again. The application then gets stuck in an infinite loop\footnote{The infinite loop is theoretical: in practice, a maximum number of stack frames can be allocated or the application runs out of memory.}.

Figure \ref{fig:InfiniteRecursionDeoptPb} shows the problem in the case where the deoptimiser is triggered by a guard failure. On the left, in the normal deoptimisation flow, the application is interrupted when a guard fails. The deoptimiser recreates the unoptimised stack frames from the optimised stack frame and edits the stack. The application can then resume with unoptimised code. On the right, in the meta-recursion issue, the application is also interrupted when a guard fails. However, while the deoptimiser is deoptimising the stack, another guard fails in the deoptimiser code. The deoptimiser then restarts to deoptimise one of its own frame, but another guard fails in its own code. The deoptimiser keeps restarting the deoptimisation of one of its own frame and the application gets stuck in an infinite loop.

\begin{figure}[h!]
    \begin{center}
        \includegraphics[width=0.65\linewidth]{InfiniteRecursionDeoptPb}
        \caption{Meta-recursion problem during deoptimisation}
        \label{fig:InfiniteRecursionDeoptPb}
    \end{center}
\end{figure}

Unlike the optimiser, the deoptimiser cannot be disabled temporarily, otherwise no application green thread requiring deoptimisation would be able to keep executing code. To solve this meta-recursion problem, we implemented two solutions. 

The first solution, described in Section \ref{sec:recovery}, restores the runtime in a "recovery mode" when recursive deoptimisation happens. In recovery mode, no optimised function can be used (the runtime relies entirely on the v-function interpreter and the baseline JIT). This solution was used successfully for a subset of the current set of benchmarks. However, this solution was not very good for applications and benchmarks using multiple green threads. 

We then designed and implemented a second solution, detailed in Section \ref{sec:independentLib}, that is still in use now. The second solution consists in keeping all the deoptimiser code in a library completely independent from the rest of the system that cannot be optimised.

\subsection{Recovery mode}
\label{sec:recovery}

As a first attempt to solve the meta-recursion issue for the deoptimiser, Scorch was modified to keep a recovery copy of each method dictionary where optimised v-functions are installed. The recovery copies include only unoptimised v-functions. We added a global flag, marking if a deoptimisation is in progress. If the deoptimiser is activated while a deoptimisation is in progress (this can be known thanks to the global flag), the deoptimiser falls back to recovery mode. To do so, the deoptimiser uses the primitive \ct{become:} (described in Section \ref{par:become}) to swap all method dictionaries with their recovery copy and disables the optimiser not to optimise anything in the recovery copies. The deoptimiser can then deoptimise the stack without calling itself repeatedly as it now uses only unoptimised functions. Once the stack is deoptimised, the deoptimiser restores the method dictionaries with the optimised v-functions and re-enables the optimiser. 

Figure \ref{fig:InfiniteRecursionDeoptPbRecovery} shows how the recovery mode solves the meta-recursion issue in the deoptimiser. The application is interrupted when a guard fails, and the deoptimiser recreates the unoptimised stack frames from the optimised stack frame. If another guard fails in the deoptimiser code, Scorch falls back to recovery mode, and deoptimises its stack frame using only unoptimised code. Once done, Scorch deactivates the recovery mode to resume the deoptimisation of the application optimised frame. In the worst case, the deoptimiser may switch multiple times to recovery mode to deoptimise the application frame. One the unoptimised stack frames are recreated, the application can resume with non-optimised code.

\begin{figure}[h!]
    \begin{center}
        \includegraphics[width=0.6\linewidth]{InfiniteRecursionDeoptPbRecovery}
        \caption{Meta-recursion problem solved with recovery mode}
        \label{fig:InfiniteRecursionDeoptPbRecovery}
    \end{center}
\end{figure}

With this solution, most of the deoptimiser code can be optimised. Indeed, if a meta-recursion happens, Scorch is able to switch to recovery mode and executes correctly the code. Although most of the deoptimiser code can be optimised, not all the code can be. The Smalltalk code executed between the guard failure call-back and the point where recovery mode is activated cannot be optimised. Such code is not protected by the recovery mode and may suffer from the meta-recursion issue. To avoid the problem, we marked a very small list of functions so they cannot be optimised.

\paragraph{Issues.} We were able to run most benchmarks with this solution. However, some benchmarks showed significant slow-down during deoptimisation. Moreover, other benchmarks (the ones using multiple green threads) were crashing. With this solution, we had two major problems. 

The first problem is that switching to recovery mode requires to edit many look-up caches in the VM to use unoptimised functions. Each look-up cache entry referencing an optimised function needs to be edited to refer back to the unoptimised function. Once the deoptimisation in recovery mode is terminated, the caches need to be updated again to reference the optimised functions. Updating all the caches can take a significant amount of time. In addition, in our implementation, the inline caches are directly written in the native code of each n-function. Each update in the native code requires the processor to partially flush its instruction cache. The recovery mode therefore slows down the application for a short while due to the time spent for the VM to update the caches but also to the cpu instruction cache flush and miss (due to the flush).

The second and main problem is that several of our benchmarks use multiple green threads. In this case, the global flag approach to mark if a deoptimisation is in progress does not work as multiple deoptimisations may happen concurrently. In a normal application, when such a problem happens, a developer uses semaphores or other green thread management features present in the language to make the code green thread safe. In the case of Pharo, most of the code related to green thread management and scheduling is written in Pharo itself. Using this code to mark if a deoptimisation is in progress in a process would require it to be marked in the list of functions that cannot be optimised, as this is used in-between the guard failure call-back and the activation of the recovery mode. We did not want to disable optimisations on code present in the semaphores and the process scheduler as such code may be performance critical in some applications. Forbidding the optimisation of such code seemed to be too restrictive. We concluded that this approach could not work for our production environment.

\subsection{Independent library}
\label{sec:independentLib}

As a second solution, we designed the whole deoptimiser as a completely independent Smalltalk library. The deoptimiser may use primitive operations but is not allowed to use any external function. All the deoptimiser classes are marked: their functions do not have profiling counters and the Scorch optimiser is aware that the optimisation of such functions is not allowed. The deoptimiser code is therefore not optimised at runtime, it is running using only the v-function interpreter and the baseline JIT. As the deoptimiser code cannot be optimised and cannot use any external function that could be optimised, the meta-recursion issue cannot happen.

\paragraph{Constraints.} This solution has three main constraints:
\begin{enumerate}
	\item \emph{The deoptimiser code cannot be optimised at runtime.} All the deoptimiser classes are marked not to be optimised at runtime, forbidding the deoptimiser code to reach high performance like the rest of the code. This constraint may be considered as minor as deoptimisation is uncommon. Not optimising the deoptimiser code may therefore not be a problem as the overall time spent in executing its code is very low. In addition, this problem can be partially solved by optimising the deoptimiser code ahead-of-time using Scorch only with optimisations that do not require deoptimisation guards and type information inferred statically. As the library is quite small (500 LoC) and has a very strong invariant (it cannot call any external function), a type inferencer can be easy to implement, very precise and efficient.
	\item \emph{The deoptimiser code needs to be completely independent.} Only primitive operations can be used directly because any external function called may be optimised, potentially leading to the meta-recursion problem. To remove the dependencies to external functions, we analysed what libraries the deoptimiser depends on. Most dependencies were very small (for instance accessors to the reification of stack frames) and they could be removed by duplicating some code. However, one dependency was a problem: the deoptimiser uses arrays and dictionaries to deoptimise the stack. Those classes are used in many applications so we cannot just forbid to optimise their code base. To solve this issue, we created a minimal array and a minimal dictionary as part of the deoptimiser library that cannot be optimised. The two collections have a separate code base from the classical collection library, which needs to be maintained in parallel to the core collections. As mentioned in the previous paragraph, the overall optimiser code base is very small (500 LoC including the duplicated collections), so we believe it is possible to maintain it without too much effort.
	\item \emph{Work on the deoptimiser is very tedious.} A simple thing such as logging a string in the deoptimiser code for debugging requires to call an external function and may lead to the meta-recursion problem. Understanding and debugging the deoptimiser code is therefore quite difficult.
\end{enumerate}

Although the constraints are important, we were able to run all our benchmarks with this design. We believe this implementation is good enough to move to production, at least on the short term.

%%%%%%%%%%%%%%%%%%%%%%%%%%%%%%%%%%%%%%%%%%%%%%%%%%%%%%%%%%%%%%%%%%%%%%%%%%%%%%%%%%%%%%%%%%%%%%%%%%%%%%%%%%%%%%%%%%%%%%%%%%%%%

\section{Related work}
\label{sec:recRelW}

To have an optimising JIT optimising its own code and encounter the meta-recursion problem we discussed in this chapter, the optimising JIT has to be written in one of the languages it can optimise and run in the same runtime than the optimised application. Such an optimising JIT is not common.

Many production VMs are entirely written in a low-level language such as C++~\cite{V8,Webkit15}. The optimising JIT cannot optimise its own code in such VMs. Other VMs such as the ones written with the RPython toolchain ~\cite{Rigo06a} are written in a language that the optimising JIT could optimise, but the production VMs are compiled ahead-of-time to native code, hence the optimising JIT does not optimise its own code at runtime in production. Metacircular VMs~\cite{Unga05b,Alp99a} are entirely written in a language they can run. H

Many metacircular VMs, such as Klein~\cite{Unga05b}, do not feature an optimising JIT\footnote{In 2009, Adam Spitz reported some work in the direction of an inlining JIT compiler in Klein on the project web page, but there has been no further news about it since then.}. There are two main projects where the optimising JIT optimises its own code at runtime. The first project is the Graal compiler~\cite{Oracle13,Dubo13c} which can be used both in the context of the Maxine VM~\cite{Wimm13a} and the Java Hotspot VM~\cite{Pale01a}. The Graal compiler effectively optimises its own code at runtime as it would optimise the application code. The other project is the Jalape\~no VM~\cite{Alp99a}, now called Jikes RVM, features a runtime compiler that can optimise its own code at runtime. 

\subsection{Graal optimising JIT}

The Graal runtime compiler~\cite{Oracle13,Dubo13c} is an optimising JIT written in Java, which is able to optimise its own code at runtime. Graal can be used in different contexts, with different solutions to the meta-recursion problem. Initially, the Graal runtime compiler was designed and implemented as part of the Maxine VM~\cite{Wimm13a}, a metacircular Java VM. Graal was then extracted from Maxine and it can now work on top of the Java Hotspot VM~\cite{Pale01a}. Graal can be used in two main ways on top of the hotspot VM. On the one hand, it can be used as an alternative optimising JIT, replacing the Java Hotspot optimising JIT written in C++. On the other hand, it can be used as a special purpose optimising JIT, optimising only specific libraries or application while the rest of the Java runtime is optimised with hotspot optimising JIT. %VMs built with Truffle~\cite{Wur13a}, a framework allowing to build efficient VMs by simply writting an AST interpreter in Java, are now running using the Java Hotspot VM and the Graal compiler as the optimising JIT.

\paragraph{Graal-hotspot architecture.} In our context, the most relevant use-case is when the Graal compiler is used as an alternative optimising JIT on top of the Java Hotspot VM. The interpreter and baseline JIT tiers are in this case present in Java Hotspot VM, written in a low level language (C++) and compiled ahead-of-time to native code. This is very similar to our design, where the Pharo interpreter and baseline JIT are also compiled ahead-of-time to native code. The optimising JIT are in both cases written in the language run by the VM (Graal in Java and Scorch in Smalltalk), they can optimise their own code and they need to interface with the existing VM to trigger runtime compilation and to install optimised code.

In the Graal-hotspot runtime, when a hotspot is detected, code in the hotspot VM (written in C++) searches the stack for a function to optimise. Once the function is chosen, Hotspot adds it to a thread-safe compilation queue. The Graal compiler is run in different native threads concurrently to the application native threads. Graal takes functions to optimise from the compilation queue, generates concurrently optimised n-function and hands them over to the hotspot VM for installation. The optimised n-functions handed by Graal to hotspot respect the Graal Java native interface~\cite{Grim13a}. They include deoptimisation metadata that the hot spot VM is able to understand. When dynamic deoptimisation happens, code written in the hotspot VM (in C++) is responsible for the deoptimisation of the stack using the metadata handed at installation time.

In our work, Scorch optimiser is able to optimise its own code according to different constraints. The stack search code can be optimised only if a hot spot is detected while the optimiser is running in background mode. The code responsible for the optimisation of a function can be optimised only indirectly through the background mode. Scorch deoptimiser code cannot be optimised at all. 

\paragraph{Comparison between the architectures.}Table \ref{tbl:comparison} sumarizes the similarities and the differences between the architectures. We call \emph{the base VM} the core elements of the VM excluding the optimising JIT: the interpreter, the baseline JIT and the GC. We then distinguish three parts in the optimising JIT:
\begin{enumerate}
	\item \emph{The stack search:} responsible to find a function to optimise based on a stack with a hot spot.
	\item \emph{The optimisation of a function:} responsible to generate an optimised function based on a non-optimised function and runtime information. This is by far the largest and most complex part of the optimising JIT.
	\item \emph{The deoptimisation of a frame:} responsible to recreate non-optimised stack frames from an optimised frame. 
\end{enumerate}
In both architectures, the base VM is optimised and compiled ahead-of-time (AOT) to executable code. The stack search code is also optimised and compiled ahead-of-time in the case of the Graal-hotspot architecture, while it is running in the same runtime as the application in the case of Scorch. The stack search code can be optimised in Scorch if the hot spot is detected while Scorch is running in background mode. The code responsible to generate the optimised function is in both cases running in the same runtime than the application optimised. In the case of Graal-hotspot, the code can be optimised the same way than the application code. In the case of Sista, the optimisation of the function is postponed to the background compilation queue. Lastly, the code responsible for the deoptimisation of a frame is optimised and compiled ahead-of-time in the case of the Graal-hotspot architecture, while it is running in the same runtime than the application in the case of Sista, but cannot be optimised at runtime.

\begin{table}
  \caption{Comparison between the Sista and the Graal-hotspot architectures}
  \vspace{0.1cm}
  \centering
  \begin{tabular}{c||c|c}
    \toprule
    & Sista & Graal-hotspot \\
    \midrule
    \midrule
	Base & Compiled and & Compiled and \\
	VM & optimised AOT & optimised AOT \\
    \midrule
	Stack & Application runtime, optimised & Compiled and \\
	Search & at runtime if hotspot detected & optimised AOT \\
	 & in background mode & \\
    \midrule
	Optimisation & Application runtime, optimised & Application runtime,\\
	of a & at runtime through & optimised at runtime \\
	Function & the background mode & unconditionally \\
    \midrule
	Deoptimisation & Application runtime, & Compiled and \\
	of a Frame & not optimised at runtime & optimised AOT \\
	\bottomrule
  \end{tabular}
  \label{tbl:comparison}
\end{table}

\paragraph{Notable differences.}Having the stack search and deoptimisation code in Smalltalk, even with constraints, allow us to change part of the design such as the deoptimisation metadata without having to recompile the VM. This is an advantage in our context, as we want Smalltalk developer to be able contribute to the project without having to recompile the VM or look into low-level details. It can however be seen as a draw-back as the constraints, decreases the performance of the deoptimisation of stack frames and the start-up performance of the stack search.

The other difference is how the optimisation of a function is managed. In Sista, Scorch needs to postpone the optimisation to background mode while the Graal-hotspot architecture allows one to optimise the function like any application function. Our constraints comes from the fact that Pharo is currently single-threaded. In Graal-hotspot, the optimisation of a function is done in a concurrent native thread. This allows one to avoid having a critical and a background mode, as well as allowing the optimising JIT to optimise this part of the code without any constraints. The solution of Graal-hotspot has less constraints, but it requires multithreading support.

\subsection{Jikes RVM}

Jikes RVM~\cite{Alp99a,Arn00} optimising runtime compiler is written entirely in Java and can optimise Java code, including its own code. However, it is not currently able to use runtime information to direct its optimisations and does not generate deoptimisation guards\footnote{"The provided AOS [Adaptive Optimisation System] models do not support Feedback-Directed Optimizations" http://www.jikesrvm.org/ProjectStatus/}, so it is not that relevant in our context. 

The runtime compiler uses however an interesting technique~\cite{Arn00} to choose what function to optimise. Instead of profiling counters, Jikes RVM uses an external sampling profiling native thread. Based on the profiling samples, the profiling thread detects what function should be optimised and adds it to a thread-safe compilation queue. The optimising runtime compiler can then start other native threads which take functions to optimise from the compilation queue, optimise and install them. With this technique, the functions to optimise are chosen entirely concurrently. The application is not interrupted, at any time, to search the stack for a function to optimise or to detect a hot spot. This technique therefore allows one to write the hot spot detection in the same runtime than the application optimised, while it can still be optimised the same way than the application run. We did not investigate in this direction because our VM is currently single threaded.

%%%%%%%%%%%%%%%%%%%%%%%%%%%%%%%%%%%%%%%%%%%%%%%%%%%%%%%%%%%%%%%%%%%%%%%%%%%%%%%%%%%%%%%%%%%%%%%%%%%%%%%%%%%%%%%%%%%%%%%%%%%%%

\section*{Conclusion}

In this chapter we discussed the meta-recursion issue. If a hot spot is detected inside the optimiser code, the optimiser may call itself indefinitely to try to optimise itself. The deoptimiser has a similar issue when it needs to deoptimise its own code. The problem exists because the optimiser and the deoptimiser are implemented in Smalltalk and are running in the same runtime and the same native thread than the application they optimise and deoptimise respectively. The main issue is related to meta-recursion.

The optimiser solves this issue by disabling itself when it runs in critical mode (interrupting temporarily the application green thread to perform the optimisation). %The optimiser cannot optimise itself directly while running in critical mode, it can only optimise the  application, which may include libraries used both by the application and the optimiser itself. For functions taking a long time to optimise, the optimiser cannot stop the application for too long or the application becomes unresponsive, hence it postpone the optimisation to a background compilation queue where functions are optimised when the application is in idle. When performing optimisations in the background, the optimiser can optimise itself entirely.
The deoptimiser has to solve the problem differently as it cannot be disabled temporarily or Smalltalk code cannot be executed any more. The deoptimiser avoids the problem by being written using a small number of classes, independent from the rest of the system, that cannot be optimised nor call any external function.

The next chapter explains how the runtime state is persisted across multiple VM start-ups, including the running green threads and the optimised functions.

\ifx\wholebook\relax\else
    \end{document}
\fi
\ifx\wholebook\relax\else

% --------------------------------------------
% Lulu:

    \documentclass[a4paper,12pt,twoside]{../includes/ThesisStyle}

	\usepackage[T1]{fontenc} %%%key to get copy and paste for the code!
%\usepackage[utf8]{inputenc} %%% to support copy and paste with accents for frnehc stuff
\usepackage{times}
\usepackage{ifthen}
\usepackage{xspace}
\usepackage{alltt}
\usepackage{latexsym}
\usepackage{url}            
\usepackage{amssymb}
\usepackage{amsfonts}
\usepackage{amsmath}
\usepackage{stmaryrd}
\usepackage{enumerate}
\usepackage{cite}
%\usepackage[pdftex,colorlinks=true,pdfstartview=FitV,linkcolor=blue,citecolor=blue,urlcolor=blue]{hyperref}
\usepackage{xspace}
%\usepackage{graphicx}
\usepackage{subfigure}
\usepackage[scaled=0.85]{helvet}
        
        
\newcommand{\sepe}{\mbox{>>}}
\newcommand{\pack}[1]{\emph{#1}}
\newcommand{\ozo}{\textsc{oZone}\xspace}
\newcommand\currentissues{\par\smallskip\textbf{Current Issues -- }}

\newboolean{showcomments}
\setboolean{showcomments}{true}
\ifthenelse{\boolean{showcomments}}
  {\newcommand{\bnote}[2]{
	\fbox{\bfseries\sffamily\scriptsize#1}
    {\sf\small$\blacktriangleright$\textit{#2}$\blacktriangleleft$}
    % \marginpar{\fbox{\bfseries\sffamily#1}}
   }
   \newcommand{\cvsversion}{\emph{\scriptsize$-$Id: macros.tex,v 1.1.1.1 2007/02/28 13:43:36 bergel Exp $-$}}
  }
  {\newcommand{\bnote}[2]{}
   \newcommand{\cvsversion}{}
  } 


\newcommand{\here}{\bnote{***}{CONTINUE HERE}}
\newcommand{\nb}[1]{\bnote{NB}{#1}}
\newcommand{\fix}[1]{\bnote{FIX}{#1}}
%%%% add your own macros 

\newcommand{\sd}[1]{\bnote{Stef}{#1}}
\newcommand{\ja}[1]{\bnote{Jannik}{#1}}
\newcommand{\na}[1]{\bnote{Nico}{#1}}
%%% 


\newcommand{\figref}[1]{Figure~\ref{fig:#1}}
\newcommand{\figlabel}[1]{\label{fig:#1}}
\newcommand{\tabref}[1]{Table~\ref{tab:#1}}
\newcommand{\layout}[1]{#1}
\newcommand{\commented}[1]{}
\newcommand{\secref}[1]{Section \ref{sec:#1}}
\newcommand{\seclabel}[1]{\label{sec:#1}}

%\newcommand{\ct}[1]{\textsf{#1}}
\newcommand{\stCode}[1]{\textsf{#1}}
\newcommand{\stMethod}[1]{\textsf{#1}}
\newcommand{\sep}{\texttt{>>}\xspace}
\newcommand{\stAssoc}{\texttt{->}\xspace}

\newcommand{\stBar}{$\mid$}
\newcommand{\stSelector}{$\gg$}
\newcommand{\ret}{\^{}}
\newcommand{\msup}{$>$}
%\newcommand{\ret}{$\uparrow$\xspace}

\newcommand{\myparagraph}[1]{\noindent\textbf{#1.}}
\newcommand{\eg}{\emph{e.g.,}\xspace}
\newcommand{\ie}{\emph{i.e.,}\xspace}
\newcommand{\ct}[1]{{\textsf{#1}}\xspace}


\newenvironment{code}
    {\begin{alltt}\sffamily}
    {\end{alltt}\normalsize}

\newcommand{\defaultScale}{0.55}
\newcommand{\pic}[3]{
   \begin{figure}[h]
   \begin{center}
   \includegraphics[scale=\defaultScale]{#1}
   \caption{#2}
   \label{#3}
   \end{center}
   \end{figure}
}

\newcommand{\twocolumnpic}[3]{
   \begin{figure*}[!ht]
   \begin{center}
   \includegraphics[scale=\defaultScale]{#1}
   \caption{#2}
   \label{#3}
   \end{center}
   \end{figure*}}

\newcommand{\infe}{$<$}
\newcommand{\supe}{$\rightarrow$\xspace}
\newcommand{\di}{$\gg$\xspace}
\newcommand{\adhoc}{\textit{ad-hoc}\xspace}

\usepackage{url}            
\makeatletter
\def\url@leostyle{%
  \@ifundefined{selectfont}{\def\UrlFont{\sf}}{\def\UrlFont{\small\sffamily}}}
\makeatother
% Now actually use the newly defined style.
\urlstyle{leo}



	\input{../includes/formatAndDefs}

	\graphicspath{{.}{../figures/}}
	\begin{document}
\fi

\chapter{Runtime state persistence across start-ups}
\label{chap:persistence}
\minitoc

This chapter describes how the Sista VM persists the runtime state across multiple VM start-ups, including the running green threads and the optimised code. The first Section discusses how snapshots are implemented in the existing Pharo runtime, including how the code and the running green threads are persisted across start-ups and their interactions with Sista. Section \ref{sec:warmup} focuses on the main issue: the start-up performance of many VMs today is significantly worse than the peak performance. Several cases where the start-up performance is a problem are described.  Section \ref{sec:relWork} compares our approach to existing VMs. Few VMs attempts to persist the runtime state across multiple start-ups, but some VMs include solutions to improve start-up performance, solving the same problem.

\section{Snapshots and persistence}

Snapshots are available in multiple object-oriented languages such as Smalltalk \cite{Gold83a} and later Dart \cite{Anna13a}. As discussed in Section \ref{par:snapshot}, in our case we use Pharo which features snapshots. A snapshot is a serialized form of all the objects present at a precise moment in the runtime. Everything is an object in Pharo, including green threads or v-functions. To start-up Pharo, the virtual machine loads all the objects from a snapshot and resumes the execution based on the green thread that was active at snapshot time. This is how Pharo is normally launched.

\paragraph{Pharo development workflow.}A Pharo programmer does not modify source code in files as many other programming languages. For development, Pharo is started using a snapshot which includes development tools, user interface elements and a source code to v-function compiler. When started, the programmer can open the development tools and write or edit the source code of a function. Then, the compiler generates a v-function from the source code (which is implicitely added in the heap). Then, the programmer may take a new snapshot, which includes the changes made. Further start-ups, on the new snapshot, features the changes made by the programmer. We note that in this paragraph we described the normal development flow of a Smalltalk programmer: this is not a workflow the programmer can do but does not normally do, this is how all programmers currently do it.

\paragraph{Application to Sista.}In the context of Sista, optimised v-functions are installed at runtime by Scorch. Those functions effectively modify the current heap of objects. Hence, when a new snapshot is taken, optimised v-functions are persisted. The next start-up of Pharo will use directly the optimised v-functions.

\paragraph{Green threads and snapshots.}To persist running green threads in a platform-independent way, to take a snapshot, the VM reifies each stack frame to a context object as explained in Section \ref{par:frameToContext}. Effectively, this means that only v-frames are persisted: n-frames are converted to v-frames to be part of the snapshot. N-frames cannot be persisted in any case as snapshots are machine-independent.

When the VM starts from a snapshot, all running green threads are executed using the v-function interpreter. However, once a function is called multiple times or a loop is interpreted a certain number of iterations, Cogit generates a n-function for the corresponding v-function (optimised or not), and the runtime resumes with the n-function.

\paragraph{Conclusion.}To conclude, programmers normally work on Pharo by modifying the current heap, for example by adding new v-functions to method dictionaries of classes, and then take a snapshot of the heap to save their code. V-functions are persisted in snapshot but n-functions are not. 

In Sista, the optimising JIT is the combination of Scorch, which generates and installs optimised v-functions, and Cogit, which generates and installs n-functions. 

Optimised v-functions generated by Scorch are, without any additional work, persisted across multiple start-ups as part of the snapshot like unoptimised v-functions. N-functions generated by Cogit are never persisted. 

Most of the compilation time is currently spent in Scorch, hence, if Pharo is started using a snapshot including optimised v-functions, Pharo can reach peak performance very quickly. Green threads using optimised functions are persisted in the snapshot in the form of optimised and unoptimised v-frames (n-frames are converted to v-frames, they cannot be persisted because they refer to n-functions). 

Sista allows Pharo to have very good peak performance thanks to the optimising JIT while avoiding one of the major draw-back of just-in-time compilation: peak performance can be reached quickly after start-up thanks to persisted optimised v-functions.

\section{Warm-up time problem}
\label{sec:warmup}

The most important problem solved by persisting the runtime state across start-up is the warm-up time problem, i.e., the time wasted by the VM at each start-up to reach peak performance. Depending on use-cases, the warm-up time may or may not matter. In long-running applications, the warm-up time required to reach peak performance is negligible compared to the overall uptime of the application. However, when applications are started frequently and are short-lived, warm-up time matters.

We give three examples where the virtual machine start-up time matters. These examples are specific, but they give an idea where the warm-up performance is a serious problem.

\paragraph{Distributed applications.}
Modern large distributed applications run on hundreds, if not thousands, of machines such as the slaves one can rent on Amazon Web Services. Slaves are usually rented per hour, though now some services such as Amazon Lambda allows one to rent a slave for small amount of time down to a hundred milliseconds. Depending on usage, the application automatically rents new slaves or frees used slaves. This way, the application scales up very well as thousands of slaves are rent if needed, while a single slave is rent if the application is used very little. The server cost of the distributed application depends purely on how much computation power is needed: one pays for slaves when they are used and does not pay when they are not used.

The problem is that to reduce the cost to the minimum, the application needs to rent a slave when needed, but frees it at the 10th of second where the slave is not used to avoid paying for an unused slave. Doing so implies having potentially very short-lived slaves when the application usage varies greatly from a 10th of a second to the next 10th of a second. Slaves could have a life expectancy of a couple hundred milliseconds. Now, if the slave does not have enough time to reach peak performance in its short life-time, the money saved by not paying for unused slaves is dominated by the money wasted in computation power used in the optimising JIT to reach peak performance. To have very short-lived slaves worth it, the time between the slave start-up and the peak performance of the application used has to be as small as possible. A good VM for such kind of scenariois a VM where peak performance is reached immediately after start-up.

\paragraph{Mobile application.}
In the case of mobile applications, the start-up performance matters because of battery consumption. During warm-up time, the optimising compiler recompiles frequently used code. All this compilation process requires time and energy, whereas the application is not run. In the example of the Android runtime, the implementation used JIT compilation with the Dalvik VM \cite{Born08a}, then switched to client-side ahead of time compilation (ART) to avoid that energy consumption at start-up, and is now switching back to JIT compilation because of the AOT (Ahead of Time compiler) constraints \cite{Geof15a}. These different attempts show the difficulty to build a system that requires JIT compilation for high performance but can't afford an energy consuming start-up time.

\paragraph{Web pages.}
Web pages sometimes execute just a bit of Javascript code at start-up, or use extensively Javascript in their lifetime (in this latter case, one usually talk about web application). A Javascript virtual machine has to reach peak performance as quickly as possible to perform well on web pages where only a bit of Javascript code is executed at start-up, while it has also to perform well on long running web applications.

\section{Related work}
\label{sec:relWork}

This section discusses other strategies implemented in other VMs and research projects to decrease the warm-up time.

\subsection{Preheating through snapshots}

\paragraph{Dart snapshots.}

The Dart programming languages features snapshots for fast application start-up. In Dart, the programmer can generate different kind of snapshots \cite{Anna13a}. Since that publication, the Dart team have added two new kind of snapshots, specialized for iOS and Android application deployment, which are the most similar to our snapshots.

\subparagraph{Android.} A Dart snapshot for an Android application is a complete representation of the application code and the heap once the application code has been loaded but before the execution of the application. The Android snapshots are taken after a warm-up phase to be able to record call site caches in the snapshot. The call site cache is a regular heap object accessed from machine code, and its presence in the snapshot allows to persist type feedback and call site frequency.

In this case, the code is loaded pre-optimised with inline caches prefilled values. However, optimised functions are not loaded as our architecture allows to do. Only unoptimised code with precomputed runtime information is loaded.

\subparagraph{iOS.} For iOS, the Dart snapshot is slightly different as iOS does not allow JIT compilers. All reachable functions from the iOS application are compiled ahead of time, using only the features of the Dart optimising compiler that don't require dynamic deoptimisation. A shared library is generated, including all the instructions, and a snapshot that includes all the classes, functions, literal pools, call site caches, etc.

This second case is difficult to compare to our architecture: iOS forbids machine code generation, which is currently required by our architecture. A good application of our architecture to iOS is future work.

\paragraph{Cloneable VMs.}

In the work of Kawachiya and all~\cite{Kawa07a}, extensions of a Java VM allows to clone a running VM to quicken the start-up. The heap is cloned, in a similar way to our snapshot, but the n-functions are also cloned. Cloning n-functions improves start-up performance over our approach, but the clone is processor-dependent: there is no way of cloning with their approach a Java runtime from an x86 machine to an ARMv6 machine. Our approach requires slightly more warm-up time to quickly compile our optimised n-functions to n-functions, but our approach is platform-independent.

\subsection{Fast warm-up}

An alternative to snapshots is to improve the JIT compiler so the peak performance can be reached as early as possible. The improvements would consists of decreasing the JIT compilation time by improving the efficiency of the JIT code, or have better heuristic so the JIT can generate optimised code with the correct speculations with little runtime information.

\paragraph{Tiered architecture}
One solution, used the Webkit VM\cite{Webkit15}, is to have a tiered architecture with many tiers. In the version of Webkit in production from March 2015 to February 2016 \cite{Webkit15}, the code is:
\begin{itemize}
\item interpreted by a bytecode interpreter the first 6 executions.
\item compiled to machine code at 7th execution, with a non-optimising compiler, and executed as machine code up to 66 executions.
\item recompiled to more optimised machine code at 67th execution, with an optimising compiler doing some but not all optimisations, up to 666 executions.
\item recompiled to heavily optimised machine code at 667th execution, with an optimising compiler using LLVM as a backend.
\end{itemize}

At each step, the compilation time is greater but the execution time decreases. This tiered approach (4 tiers in the case of Webkit), allows to have good performance from start-up, while reaching high performance for long running code. This kind of approaches has also draw-backs: the VM development team needs to maintain and evolve four different tiers.

\paragraph{Saving runtime information.}

To reach quickly peak performance, an alternative of saving optimised code is to save the runtime information. The Dart snapshot saves already the call site information in its Android snapshots. Other techniques are available.

In Strongtalk \cite{Sun06}, a high-performance Smalltalk that has never reached production, it is possible to save the inlining decision of the optimising compiler in a separate file. The optimising compiler can then reuse this file to make the right inlining decision the first time a hot spot is detected.

In the work of Arnold and all~\cite{Arno05c}, the profiling information of unoptimised runs is persisted in a repository shared by multiple VMs. This allows new VM starting-up to re-use the information of other and previous VM runs to direct compiler optimisations.

Saving runtime information decreases the warm-up time as the optimising JIT can speculate accurately on the program behavior with very few runs. However, on the contrary to our approach, time is still wasted optimising functions.

\paragraph{Saving machine code.}

In the Azul VM Zing \cite{Azul}, available for Java, the official web site claims that "operations teams can save accumulated optimisations from one day or set of market conditions for later reuse" thanks to the technology called \emph{Ready Now!}. In addition, the website precises that the Azul VM provides an API for the developer to help the JIT to make the right optimisation decisions. As Azul is closed source, implementation details are not entirely known. 

However, word has been that the Azul VM reduces the warm-up time by saving machine code across multiple start-ups. If the application is started on another processor, then the saved machine code is simply discarded. We did not go in this direction to persist the optimisation in a platform-independent way (in our architecture, starting the application on x86 instead of ARMv5 does not require the saved optimised code to be discarded), but we have a small overhead due to the bytecode to machine code translation at each start-up. In addition, we believe it's very difficult to persist correctly machine code compared to persisting bytecodes.

Aside from Azul, the work of Reddi and all~\cite{Redd07a} details how they persist the machine code generated by the optimising JIT across multiple start-ups of the VM. JRockit~\cite{JRockit}, an Oracle product, is a production Java VM allowing to persist the machine code generated by the optimising JIT across multiple start-ups.

We did not go in the direction of machine code persistence as we wanted to keep the snapshot platform-independent way: in our architecture, starting the application on x86 instead of ARMv5 does not require the saved optimized code to be discarded, while the other solutions discussed in this paragraph do. However, we have a small overhead due to the bytecode to machine code translation at each start-up. In addition, the added complexity of machine code persistence over bytecode persistence should not be underestimated.

\paragraph{Ahead-of-time analysis.}

In the work of Krintz and Calder, static analysis done ahead of time on Java code generate annotations that are used by the optimising JIT to reduce compilation time (and hence, the warm-up time). As for the persistence of runtime information, on the contrary to our approach, time is still wasted at runtime optimising functions.

\paragraph{Ahead-of-time compilation.}

The last alternative is to pre-optimise the code ahead of time. This can be done by doing static analysis over the code to try to infer types. Applications for the iPhone are a good example where static analysis is used to pre-optimise the Objective-C application. The peak performance is lower than with a JIT compiler if the program uses a lot of virtual calls, as static analysis are not as precised as runtime information on highly dynamic language. 
However, if the program uses few dynamic features (for example most of the calls are not virtual) and is running on top of a high-performance language kernel like the Objective-C kernel, the result can be satisfying.

%%%%%%%%%%%%%%%%%%%%%%%%%%%%%%%%%%%%%%%%%%%%%%%%%%%%%%%%%%%%%%%%%%%%%%%%%%%%%%%%%%%%%%%%%%%%%%%%%%%%%%%%%%%%%%%%%%%%%%%%%%%%%%%%%%%%%%%%%%%%%%%%%%%%%%%%%%%%%%%%%%%%%%%

\section*{Conclusion} This chapter discusses how the runtime state is persisted across multiple start-ups, improving the performance during start-up. The next chapter validates the Sista architecture, mainly through performance evaluation in a set of benchmarks. The validation chapter also evaluates the Sista VM performance when the runtime state is persisted across multiple start-ups.

\ifx\wholebook\relax\else
    \end{document}
\fi
%\include{chapters/ch8-Interface}
\ifx\wholebook\relax\else

% --------------------------------------------
% Lulu:

    \documentclass[a4paper,12pt,twoside]{../includes/ThesisStyle}

	\usepackage[T1]{fontenc} %%%key to get copy and paste for the code!
%\usepackage[utf8]{inputenc} %%% to support copy and paste with accents for frnehc stuff
\usepackage{times}
\usepackage{ifthen}
\usepackage{xspace}
\usepackage{alltt}
\usepackage{latexsym}
\usepackage{url}            
\usepackage{amssymb}
\usepackage{amsfonts}
\usepackage{amsmath}
\usepackage{stmaryrd}
\usepackage{enumerate}
\usepackage{cite}
%\usepackage[pdftex,colorlinks=true,pdfstartview=FitV,linkcolor=blue,citecolor=blue,urlcolor=blue]{hyperref}
\usepackage{xspace}
%\usepackage{graphicx}
\usepackage{subfigure}
\usepackage[scaled=0.85]{helvet}
        
        
\newcommand{\sepe}{\mbox{>>}}
\newcommand{\pack}[1]{\emph{#1}}
\newcommand{\ozo}{\textsc{oZone}\xspace}
\newcommand\currentissues{\par\smallskip\textbf{Current Issues -- }}

\newboolean{showcomments}
\setboolean{showcomments}{true}
\ifthenelse{\boolean{showcomments}}
  {\newcommand{\bnote}[2]{
	\fbox{\bfseries\sffamily\scriptsize#1}
    {\sf\small$\blacktriangleright$\textit{#2}$\blacktriangleleft$}
    % \marginpar{\fbox{\bfseries\sffamily#1}}
   }
   \newcommand{\cvsversion}{\emph{\scriptsize$-$Id: macros.tex,v 1.1.1.1 2007/02/28 13:43:36 bergel Exp $-$}}
  }
  {\newcommand{\bnote}[2]{}
   \newcommand{\cvsversion}{}
  } 


\newcommand{\here}{\bnote{***}{CONTINUE HERE}}
\newcommand{\nb}[1]{\bnote{NB}{#1}}
\newcommand{\fix}[1]{\bnote{FIX}{#1}}
%%%% add your own macros 

\newcommand{\sd}[1]{\bnote{Stef}{#1}}
\newcommand{\ja}[1]{\bnote{Jannik}{#1}}
\newcommand{\na}[1]{\bnote{Nico}{#1}}
%%% 


\newcommand{\figref}[1]{Figure~\ref{fig:#1}}
\newcommand{\figlabel}[1]{\label{fig:#1}}
\newcommand{\tabref}[1]{Table~\ref{tab:#1}}
\newcommand{\layout}[1]{#1}
\newcommand{\commented}[1]{}
\newcommand{\secref}[1]{Section \ref{sec:#1}}
\newcommand{\seclabel}[1]{\label{sec:#1}}

%\newcommand{\ct}[1]{\textsf{#1}}
\newcommand{\stCode}[1]{\textsf{#1}}
\newcommand{\stMethod}[1]{\textsf{#1}}
\newcommand{\sep}{\texttt{>>}\xspace}
\newcommand{\stAssoc}{\texttt{->}\xspace}

\newcommand{\stBar}{$\mid$}
\newcommand{\stSelector}{$\gg$}
\newcommand{\ret}{\^{}}
\newcommand{\msup}{$>$}
%\newcommand{\ret}{$\uparrow$\xspace}

\newcommand{\myparagraph}[1]{\noindent\textbf{#1.}}
\newcommand{\eg}{\emph{e.g.,}\xspace}
\newcommand{\ie}{\emph{i.e.,}\xspace}
\newcommand{\ct}[1]{{\textsf{#1}}\xspace}


\newenvironment{code}
    {\begin{alltt}\sffamily}
    {\end{alltt}\normalsize}

\newcommand{\defaultScale}{0.55}
\newcommand{\pic}[3]{
   \begin{figure}[h]
   \begin{center}
   \includegraphics[scale=\defaultScale]{#1}
   \caption{#2}
   \label{#3}
   \end{center}
   \end{figure}
}

\newcommand{\twocolumnpic}[3]{
   \begin{figure*}[!ht]
   \begin{center}
   \includegraphics[scale=\defaultScale]{#1}
   \caption{#2}
   \label{#3}
   \end{center}
   \end{figure*}}

\newcommand{\infe}{$<$}
\newcommand{\supe}{$\rightarrow$\xspace}
\newcommand{\di}{$\gg$\xspace}
\newcommand{\adhoc}{\textit{ad-hoc}\xspace}

\usepackage{url}            
\makeatletter
\def\url@leostyle{%
  \@ifundefined{selectfont}{\def\UrlFont{\sf}}{\def\UrlFont{\small\sffamily}}}
\makeatother
% Now actually use the newly defined style.
\urlstyle{leo}



	\input{../includes/formatAndDefs}

	\graphicspath{{.}{../figures/}}
	\begin{document}
\fi

\chapter{Validation}
\label{chap:validation}
\minitoc

\section{Counter overhead}

\section{Performance gain}

\section{Efficient start-ups}

Doc: from sista arch paper + do better benchmarks

we need to discuss the maturity of the project and why start-up time is not that good any way. we dont want to spend time here as persistent.

We have both performance result and start-up time result.


\ifx\wholebook\relax\else
    \end{document}
\fi
\ifx\wholebook\relax\else

% --------------------------------------------
% Lulu:

    \documentclass[a4paper,12pt,twoside]{../includes/ThesisStyle}

	\usepackage[T1]{fontenc} %%%key to get copy and paste for the code!
%\usepackage[utf8]{inputenc} %%% to support copy and paste with accents for frnehc stuff
\usepackage{times}
\usepackage{ifthen}
\usepackage{xspace}
\usepackage{alltt}
\usepackage{latexsym}
\usepackage{url}            
\usepackage{amssymb}
\usepackage{amsfonts}
\usepackage{amsmath}
\usepackage{stmaryrd}
\usepackage{enumerate}
\usepackage{cite}
%\usepackage[pdftex,colorlinks=true,pdfstartview=FitV,linkcolor=blue,citecolor=blue,urlcolor=blue]{hyperref}
\usepackage{xspace}
%\usepackage{graphicx}
\usepackage{subfigure}
\usepackage[scaled=0.85]{helvet}
        
        
\newcommand{\sepe}{\mbox{>>}}
\newcommand{\pack}[1]{\emph{#1}}
\newcommand{\ozo}{\textsc{oZone}\xspace}
\newcommand\currentissues{\par\smallskip\textbf{Current Issues -- }}

\newboolean{showcomments}
\setboolean{showcomments}{true}
\ifthenelse{\boolean{showcomments}}
  {\newcommand{\bnote}[2]{
	\fbox{\bfseries\sffamily\scriptsize#1}
    {\sf\small$\blacktriangleright$\textit{#2}$\blacktriangleleft$}
    % \marginpar{\fbox{\bfseries\sffamily#1}}
   }
   \newcommand{\cvsversion}{\emph{\scriptsize$-$Id: macros.tex,v 1.1.1.1 2007/02/28 13:43:36 bergel Exp $-$}}
  }
  {\newcommand{\bnote}[2]{}
   \newcommand{\cvsversion}{}
  } 


\newcommand{\here}{\bnote{***}{CONTINUE HERE}}
\newcommand{\nb}[1]{\bnote{NB}{#1}}
\newcommand{\fix}[1]{\bnote{FIX}{#1}}
%%%% add your own macros 

\newcommand{\sd}[1]{\bnote{Stef}{#1}}
\newcommand{\ja}[1]{\bnote{Jannik}{#1}}
\newcommand{\na}[1]{\bnote{Nico}{#1}}
%%% 


\newcommand{\figref}[1]{Figure~\ref{fig:#1}}
\newcommand{\figlabel}[1]{\label{fig:#1}}
\newcommand{\tabref}[1]{Table~\ref{tab:#1}}
\newcommand{\layout}[1]{#1}
\newcommand{\commented}[1]{}
\newcommand{\secref}[1]{Section \ref{sec:#1}}
\newcommand{\seclabel}[1]{\label{sec:#1}}

%\newcommand{\ct}[1]{\textsf{#1}}
\newcommand{\stCode}[1]{\textsf{#1}}
\newcommand{\stMethod}[1]{\textsf{#1}}
\newcommand{\sep}{\texttt{>>}\xspace}
\newcommand{\stAssoc}{\texttt{->}\xspace}

\newcommand{\stBar}{$\mid$}
\newcommand{\stSelector}{$\gg$}
\newcommand{\ret}{\^{}}
\newcommand{\msup}{$>$}
%\newcommand{\ret}{$\uparrow$\xspace}

\newcommand{\myparagraph}[1]{\noindent\textbf{#1.}}
\newcommand{\eg}{\emph{e.g.,}\xspace}
\newcommand{\ie}{\emph{i.e.,}\xspace}
\newcommand{\ct}[1]{{\textsf{#1}}\xspace}


\newenvironment{code}
    {\begin{alltt}\sffamily}
    {\end{alltt}\normalsize}

\newcommand{\defaultScale}{0.55}
\newcommand{\pic}[3]{
   \begin{figure}[h]
   \begin{center}
   \includegraphics[scale=\defaultScale]{#1}
   \caption{#2}
   \label{#3}
   \end{center}
   \end{figure}
}

\newcommand{\twocolumnpic}[3]{
   \begin{figure*}[!ht]
   \begin{center}
   \includegraphics[scale=\defaultScale]{#1}
   \caption{#2}
   \label{#3}
   \end{center}
   \end{figure*}}

\newcommand{\infe}{$<$}
\newcommand{\supe}{$\rightarrow$\xspace}
\newcommand{\di}{$\gg$\xspace}
\newcommand{\adhoc}{\textit{ad-hoc}\xspace}

\usepackage{url}            
\makeatletter
\def\url@leostyle{%
  \@ifundefined{selectfont}{\def\UrlFont{\sf}}{\def\UrlFont{\small\sffamily}}}
\makeatother
% Now actually use the newly defined style.
\urlstyle{leo}



	\input{../includes/formatAndDefs}

	\graphicspath{{.}{../figures/}}
	\begin{document}
\fi

\chapter{Future work}
\label{chap:futureWork}
\minitoc

This Chapter discusses future work related to Sista. The focus is on research-oriented future work, but the first two sections also mention engineering-oriented future work (for instance, moving the architecture to production).

Section \ref{sec:archEvo} describes the evolution planned or considered for the overall architecture. Section \ref{sec:newOpt} discusses specifically the optimisations that could be implemented in Scorch and Cogit. Section \ref{sec:useCase} details how the application of Sista on a real-world application requiring quick start-ups would be very valuable to validate further the architecture. Lastly, Section \ref{sec:energy} discusses briefly the potentially low energy consumption of the Sista VM and how it could be valuable in specific use-cases.

\section{Architecture evolution}
\label{sec:archEvo}

This section details the evolution of the Sista architecture that are worth investigating. Section \ref{ss:FWInterface} describes the potential evolution of the interface VM-language: Sista is currently using an extended bytecode set to communicate between Scorch and Cogit, another representation may be better. Section \ref{ss:FWIDE} explains the on-going work to integrate Sista with the development tools. The evaluation of the memory footprint used by the Sista runtime is precised in Section \ref{ss:FWMemFootprint}. Section \ref{ss:FWPlatDep} discusses the current platform-dependencies of the persistence of the runtime state across start-ups and how these dependencies could be avoided. Section \ref{ss:FWProduct} briefly states the on-going work to move Sista to production.

\subsection{Interface VM-language}
\label{ss:FWInterface}

\paragraph{Stack-based or register-based IR.}The v-functions discussed in the thesis are currently encoded in a stack-based bytecode Intermediate Representation (IR). Stack-based IRs are usually considered as difficult to deal with in optimising compilers, so this can be seen as a problem. For this reason, Scorch decompiles the v-function to a register-based IR (similar to TAC\footnote{Three Address Code.}, but some operations such as virtual calls may have more than three parameters), performs the optimisations and translates it back to the stack-based bytecode. Cogit then takes the stack-based bytecode as input, and translates it to a register-based IR to generate the n-function. 

The stack-based extended bytecode sets has two main issues:
\begin{itemize}
	\item \textbf{Loss of information:} Scorch IR has significant information about the instructions (liveness, uses) than the stack-based bytecode. This information is lost during the translation to the stack-based bytecode, while it may be valuable for Cogit to perform efficiently low-level optimisations.
	\item \textbf{redundant conversion:} When compiling using Sista's optimising JIT, unoptimised v-functions are compiled by Scorch to optimised v-functions and then by Cogit to optimised n-functions. It feels a bit redundant to take a stack-based IR (bytecode of the v-function), translate it to a register-based IR (Scorch IR), then translate it back to a stack-based IR (extended bytecode of the optimised v-function) and lastly translate it to a register-based IR (Cogit's IR), as shown in the left part of Figure \ref{fig:FutureWorkredundant}. One could consider encoding optimised v-functions in a register-based IR, to avoid two translations.
\end{itemize}

\begin{figure}[h!]
    \begin{center}
        \includegraphics[width=0.7\linewidth]{FutureWorkredondant}
        \caption{redundant conversion}
        \label{fig:FutureWorkredundant}
    \end{center}
\end{figure}

However, the extended bytecode set was designed stack-based for two relevant reasons:
\begin{itemize}
	\item \textbf{Compatibility:} The existing bytecode set is stack-based, and having the extended bytecode set stack-based allows us to generate optimised functions using existing instructions and not only new ones. This was very convenient to have quickly a working version of Sista (only the new unsafe operations used needed to work to have the architecture working).
	\item \textbf{Machine-independent:} A stack-based representation allows to abstract away from low-level details such as the exact number of registers. To abstract away from the exact number of registers, both a stack-based IR and a register-based IR with an infinite number of registers are possible. Although at first look it seems the register-based solution is easier to deal with, experts such as the ones which designed WebAssembly~\cite{WebAssembly} chose to use a stack-based IR over a register-based IR. It is not clear which solution is better.
\end{itemize}	 

One future work is to design another extended bytecode set, register-based, and to evaluate the complexity of the representation in the back-end of Scorch and the front-end of Cogit, as shown on the right part of Figure \ref{fig:FutureWorkredundant}.

\paragraph{Lower-level representation.}The optimised v-functions are encoded in a quite high-level representation. For example, instructions like array accesses generate multiple native instructions. The extended bytecode set is quite high-level because it abstracts away from:
\begin{itemize}
	\item the memory representation of objects.
	\item the processor used.
\end{itemize}

We believe abstracting away from the processor used was a good idea as it allows to use snapshot to persist the optimised state across multiple start-ups. However, snapshots are already dependent on the memory representation of objects used, so one could implement a new representation of optimised v-functions with a lower-level representation, object representation dependent but still processor independent. This would allow Scorch to perform additional optimisations, such as better constant propagation (Some constants are currently hidden in high-level instructions), which are currently hard to support in Cogit.

Currently, only the VM code is aware of the memory representation of object. Hence, using such a lower-level representation would require to duplicate the knowledge about the memory representation of objects from the VM code-base to Pharo, so Scorch could be aware of it. 

\subparagraph{Summary.}To summarize, one could change Sista so optimised v-functions would be encoded in a lower-level representation instead of the extended bytecode set. Such changes would make Scorch dependent of the memory representation of objects, while keeping it independent from the processor used. Some code would need to be duplicated from the VM code-base to Pharo to make Scorch aware of the internal memory representation of objects. Scorch would be able to produce more optimised code, performing optimisations that are currently difficult to implement in Cogit, yielding hopefully better performance.

\subparagraph{Implementation.}To implement such a solution, one could build a back-end for Scorch targeting the abstract assembly instructions set featured by the Cog VM, called Lowcode~\cite{Salg16a}, similar to WebAssembly~\cite{WebAssembly}. Lowcode features low-level instructions, compiled almost in most processors one-to-one to machine instructions. The future work is to implement such a back-end for Scorch and evaluate the complexity and the performance.

\subsection{Development tools integration}
\label{ss:FWIDE}

In Sista, Scorch is an unoptimised v-functions to optimised v-functions compiler, running in the same runtime as the application. Although this design has several advantages, there is a major draw-back: when the programmer accesses the reification of a stack frame, depending on the optimisation state, an optimised frame might be shown. 

In some cases, for example when the programmer is working on Scorch itself, it is relevant to show in the development tools the optimised frames. In other cases, for example when the programmer is working in an end-user application on top of Pharo, development tools should show only unoptimised frames by transparently deoptimising stack frames.

We are now adapting the debugging tools to request deoptimise stack frames when needed. To do so, we are adding a development tool setting: one may or may not want to see the stack internals, depending on what one wants to implement.

\subsection{Memory footprint evaluation}
\label{ss:FWMemFootprint}

When implementing an optimising JIT, it is relevant to evaluate the memory footprint of the optimised code and the deoptimisation metadata. In our context, such an evaluation would be very interesting as:
\begin{itemize}
	\item \textbf{Split architecture:} Due to the split between Scorch and Cogit, optimised functions are present both as v-functions and n-functions, and each of them has different deoptimisation metadata, potentially increasing the memory footprint. 
	\item \textbf{Persistence:} As optimised v-functions are persisted across start-ups, it is interesting to know the size of the optimised v-functions and the corresponding deoptimisation metadata that is persisted.
\end{itemize}

Two future works are planned in this direction:
\begin{itemize}
	\item Does the split in the optimising JIT induce memory overhead compared to a classical function-based optimising JIT, and how big is the overhead ?
	\item What is the size of the optimised v-functions and the corresponding deoptimisation metadata that is persisted across start-up ?
\end{itemize}

We did not evaluate the memory footprint because currently the deoptimisation metadata of Scorch is kept uncompressed. As Scorch deoptimiser requires to read deoptimisation metadata, compressing the metadata requires to write a decompressor which is, as all the deoptimiser code, independent from the rest of the system. This is possible but requires a certain amount of engineering work, which is the reason why we postponed it.

\subsection{Platform-dependency}
\label{ss:FWPlatDep}

In Pharo, snapshots are independent of the processor and the OS used. It is proven as the same snapshot can be deployed on x86, ARMv6 and MIPSEL, as well as on Windows, Mac OS X, iOS, Linux, Android or RISC OS. However, the snapshots are dependent on the machine word size: 32 bit or 64 bit snapshots are not compatible. They are not compatible because the size of managed pointers is different, but also because the representation of specific objects, such as floating numbers, is different. It is however possible to convert offline a 32 bit snapshot to 64 bit and vice-versa. 

As some optimisations related to number arithmetics, such as overflow checks elimination, depends on the number representations, Scorch is currently dependent on the machine word size. A fully portable solution would either need not to do optimisations on machine word specific number representations or deoptimise the affected code on startup.

\subsection{Productisation}
\label{ss:FWProduct}

With the current version, the Sista VM is able to run all our benchmark suite. We are now able to do part of our development with the development tools, written in Pharo, running on the Sista VM. We still have work to do in the integration with the development tools, especially the debugger, but it seems that the biggest part of the work has been done. 

There are still some edge cases where the Sista VM is unstable, which still need to be fixed, but most code can be run on top of the Sista VM as it would be run on the production VM. We have started to integrate the dependencies of Sista in Pharo, such as the new implementation of closures, the new bytecode set or read-only objects. We are now looking forward to integrate Scorch in Pharo.

\section{New optimisations}
\label{sec:newOpt}

Another direction for future work is the implementation of new optimisations in Scorch or Cogit. 

\subsection{State-of-the-art optimisations}

To compare Sista against other optimising JITs efficiently, the next thing to do is to implement all the state-of-the-art compiler optimisation in Scorch and Cogit. Scorch lacks multiple common optimisation passes, including important ones for performance such as floating-pointer related optimisations or advanced escape analysis. Cogit features a naive register allocator to set registers in its dynamic templates, but we have plan to build a more advanced one and evaluate the performance difference. 

\subsection{New optimisations}

Aside from existing optimisations, one could implement new optimisations that are not present in other compilers. 

One way to do so is to have new ideas on how to optimise code, to design and implement new algorithms. This is far from trivial as many experts have worked in compiler optimisations in other compilers, but, it is theoretically possible.

Alternatively, one could describe how traditional optimisations are implemented in the context of the split architecture present in Sista. For example, one could explain which optimisation should be implemented in Scorch, which one should be implemented in Cogit, which one should be implemented partly with both and which annotations are required in the optimised v-functions to communicate extra information from Scorch to Cogit. 

Lastly, and most interestingly, one could work on Smalltalk-specific optimisations that are not possible or not relevant in other programming languages because they do not usually support the unconventional features present in Smalltalk.

Indeed, Smalltalk provides some unconventional operations that are not usually available in other object-oriented languages. These operations are problematic for the optimising JIT. The main operations we are talking about are become, described in Section \label{par:become} and heavy stack manipulation APIs on reified stack frames detailed in Section \ref{par:frameToContext}. 

In each case, the feature has implications in the context of an optimising JIT as at any interrupt point, there any temporary variable of the optimised stack frame, or worst, any internal state (sender frame, program counter, etc.) could be edited to any object in the heap. This would invalidate all assumptions taken at compilation time by Scorch.

Fortunately, all these operations are uncommon in a normal Smalltalk program at runtime. They are usually used for implementing the debugging functionalities of Pharo. Currently, profiling production applications does not show we could earn noticeable performance if we would optimise such cases. The current solution is therefore to always deoptimise the stack frames involved when such unconventional operations happen. In the case of become, if a temporary variable in a stack frame executing an optimised method is edited, the frame is deoptimised. In the case of the stack manipulation, if the reification of the stack is mutated from the language, we deoptimise the corresponding mutated stack frames.

However, in specific libraries or workflow, such operations may be common enough to have some impact on performance. Especially, continuations and exceptions are built in Pharo in the language on top of the stack manipulation features, and a few libraries use them extensively. It could be possible to have Scorch aware of these features and to handle them specifically. Scorch would for example mark some temporary variables as being accessed frequently from the outside of the function: such temporaries would not be optimised and the deoptimisation metadata would include information on how to access those temporaries of inlined functions directly in the optimised frame. Optimising such features would allow to have efficient continuations and exceptions, in a similar way to the optimisation of exceptions described in \cite{Ogas01a}. 

\section{Application of Sista for quick start-ups}
\label{sec:useCase}

One point that is a bit unclear is how to use the Sista for quick start-up in a real-world application. It is possible to persist optimised functions across start-ups. However, how are the optimised functions generated in the first place ? Are they generated from warm-up runs using a test suite or are they generated the first day the application is running ?

A good example on how to use persisted optimisations across start-up is the success story of Azul~\cite{Azul}: a trading bank claims that they are able to use the optimised functions generated the day N-1 to improve the start-up performance of the day N. Another good example is the user-base of the cloneable Java VM~\cite{Kawa07a}.

It seems that depending on the use-case where start-up performance matters (distributed application with short-lived slaves, such as the ones on Amazon lambda, Android application or Web pages), different frameworks and solutions require to be set up to improve start-up performance using persisted optimisations. 

It would be very valuable to focus on one of those use-cases and build a solid framework showing how to use the persistence of optimisations to reduce warm-up time and evaluate what is the cost for the application programmer. Does the programmer have to do something specific to persist the optimisations as part of the deployment (warm-up runs, etc.) or is it done automatically as part of a framework ?

In the case of distributed application on Amazon lambda, slaves may live down to a couple seconds in case of a high-variant demand, so that there are very few slaves rent when the application is unused and a lot of slaves when the application is heavily used. Is it possible to build a fraework that automatically learn from the life of previous slaves what optimised functions are worth keeping, so when a new short-lived slaved is instantiated, it can reach peak performance very quickly ?

In the case of web pages, is it possible to build a global cache so all frequently used web pages would have pre-optimised code available from the runs of the previous users ? How would such a design work with modern security requirements ?

All these applications of the persistence of optimisations across start-ups of Sista are very interesting and could be analysed in future work.

\section{Energy consumption evaluation}
\label{sec:energy}

Another interesting aspect of Sista is the energy saved at start-up by re-using persisted optimised functions instead of wasting cpu cycles re-generating them. 

One of the most relevant use-case is Android application. With the Dalvik VM or the Android Runtime, Google's team on the VM for Android application have switched their VM design from JIT compilation to AOT compilation then back to JIT compilation. The main problem is that JIT compilation yields better peak performance, but requires warm-up time for each Android application start-up which wastes many cpu cycles, which corresponds in practice to an important part of the smart phone battery.

Sista would be relevant in this context as the application could be shipped unoptimised, but each time the user would use the application, the optimised functions generated from previous runs would be persisted so further uses won't waste battery. This way, theoretically, the application could have very good peak performance due to the JIT while not wasting too many battery at each Android application start-up.

In addition, power drain is becoming an important factor not just in mobile computing but also computers embedded in things (Internet of Things) and even on servers. Excessive power leads to increased cooling. In flash memory, excessive writes can not only shorten its life but also lead to increased power (writes draw more power than reads).

To work in this direction, one would need to evaluate the energy consumed by the VM to reach peak-performance. We did not work in this direction because we did not have expertise in energy consumption measurement, but this is definitely a relevant future work.

\section*{Conclusion}

This Chapter discussed the future works that may be done based on this thesis. The next Chapter summarises the contents of the thesis and concludes.

\ifx\wholebook\relax\else
    \end{document}
\fi
\ifx\wholebook\relax\else

% --------------------------------------------
% Lulu:

    \documentclass[a4paper,12pt,twoside]{../includes/ThesisStyle}

	\usepackage[T1]{fontenc} %%%key to get copy and paste for the code!
%\usepackage[utf8]{inputenc} %%% to support copy and paste with accents for frnehc stuff
\usepackage{times}
\usepackage{ifthen}
\usepackage{xspace}
\usepackage{alltt}
\usepackage{latexsym}
\usepackage{url}            
\usepackage{amssymb}
\usepackage{amsfonts}
\usepackage{amsmath}
\usepackage{stmaryrd}
\usepackage{enumerate}
\usepackage{cite}
%\usepackage[pdftex,colorlinks=true,pdfstartview=FitV,linkcolor=blue,citecolor=blue,urlcolor=blue]{hyperref}
\usepackage{xspace}
%\usepackage{graphicx}
\usepackage{subfigure}
\usepackage[scaled=0.85]{helvet}
        
        
\newcommand{\sepe}{\mbox{>>}}
\newcommand{\pack}[1]{\emph{#1}}
\newcommand{\ozo}{\textsc{oZone}\xspace}
\newcommand\currentissues{\par\smallskip\textbf{Current Issues -- }}

\newboolean{showcomments}
\setboolean{showcomments}{true}
\ifthenelse{\boolean{showcomments}}
  {\newcommand{\bnote}[2]{
	\fbox{\bfseries\sffamily\scriptsize#1}
    {\sf\small$\blacktriangleright$\textit{#2}$\blacktriangleleft$}
    % \marginpar{\fbox{\bfseries\sffamily#1}}
   }
   \newcommand{\cvsversion}{\emph{\scriptsize$-$Id: macros.tex,v 1.1.1.1 2007/02/28 13:43:36 bergel Exp $-$}}
  }
  {\newcommand{\bnote}[2]{}
   \newcommand{\cvsversion}{}
  } 


\newcommand{\here}{\bnote{***}{CONTINUE HERE}}
\newcommand{\nb}[1]{\bnote{NB}{#1}}
\newcommand{\fix}[1]{\bnote{FIX}{#1}}
%%%% add your own macros 

\newcommand{\sd}[1]{\bnote{Stef}{#1}}
\newcommand{\ja}[1]{\bnote{Jannik}{#1}}
\newcommand{\na}[1]{\bnote{Nico}{#1}}
%%% 


\newcommand{\figref}[1]{Figure~\ref{fig:#1}}
\newcommand{\figlabel}[1]{\label{fig:#1}}
\newcommand{\tabref}[1]{Table~\ref{tab:#1}}
\newcommand{\layout}[1]{#1}
\newcommand{\commented}[1]{}
\newcommand{\secref}[1]{Section \ref{sec:#1}}
\newcommand{\seclabel}[1]{\label{sec:#1}}

%\newcommand{\ct}[1]{\textsf{#1}}
\newcommand{\stCode}[1]{\textsf{#1}}
\newcommand{\stMethod}[1]{\textsf{#1}}
\newcommand{\sep}{\texttt{>>}\xspace}
\newcommand{\stAssoc}{\texttt{->}\xspace}

\newcommand{\stBar}{$\mid$}
\newcommand{\stSelector}{$\gg$}
\newcommand{\ret}{\^{}}
\newcommand{\msup}{$>$}
%\newcommand{\ret}{$\uparrow$\xspace}

\newcommand{\myparagraph}[1]{\noindent\textbf{#1.}}
\newcommand{\eg}{\emph{e.g.,}\xspace}
\newcommand{\ie}{\emph{i.e.,}\xspace}
\newcommand{\ct}[1]{{\textsf{#1}}\xspace}


\newenvironment{code}
    {\begin{alltt}\sffamily}
    {\end{alltt}\normalsize}

\newcommand{\defaultScale}{0.55}
\newcommand{\pic}[3]{
   \begin{figure}[h]
   \begin{center}
   \includegraphics[scale=\defaultScale]{#1}
   \caption{#2}
   \label{#3}
   \end{center}
   \end{figure}
}

\newcommand{\twocolumnpic}[3]{
   \begin{figure*}[!ht]
   \begin{center}
   \includegraphics[scale=\defaultScale]{#1}
   \caption{#2}
   \label{#3}
   \end{center}
   \end{figure*}}

\newcommand{\infe}{$<$}
\newcommand{\supe}{$\rightarrow$\xspace}
\newcommand{\di}{$\gg$\xspace}
\newcommand{\adhoc}{\textit{ad-hoc}\xspace}

\usepackage{url}            
\makeatletter
\def\url@leostyle{%
  \@ifundefined{selectfont}{\def\UrlFont{\sf}}{\def\UrlFont{\small\sffamily}}}
\makeatother
% Now actually use the newly defined style.
\urlstyle{leo}



	\input{../includes/formatAndDefs}

	\graphicspath{{.}{../figures/}}
	\begin{document}
\fi

\chapter{Conclusion}
\label{chap:conclusion}
\minitoc

In this Chapter we summarize this dissertation. We list the contributions, the published papers and their impact.

\section{Summary}

\paragraph{Chapter \ref{chap:stateOfTheArt}} defined the terminology used in the Ph.D. The Chapter then described the two most common architectures for optimising JITs, with examples of VM using one architecture or the other. Then, related work to metacircular VMs and the persistence of the runtime state across start-ups were listed.

\paragraph{Chapter \ref{chap:existing}} detailled the existing Pharo runtime. The focus was on the features not present in most other VMs or relevant in the context of the dissertation. Sista is built on top of the existing runtime described in this Chapter.

\paragraph{Chapter \ref{chap:architecture}} explained Sista, including the inner details of the architecture. Especially, the split between Scorch and Cogit in the optimising JIT architecture was described. The Chapter compared Sista against existing architectures for optimising JITs.

\paragraph{Chapter \ref{chap:runtimeEvolution}} discussed the evolutions required to the existing Pharo runtime to implement the Sista architecture on top of the existing runtime. This included all the core aspects of the system which needed to be changed to support Sista, such as the implementation of closures or the ability to mark an object as read-only.

\paragraph{Chapter \ref{chap:metacircular}} was articulated around a problem happening in metacircular optimising JIT such as Scorch: Scorch may end up in an infinite recursion, calling itself repeatedly and slowing down code execution drastically. The problem happens both during function optimisation and frame deoptimisation. The problem is solved by disabling the optimiser in specific circumstances and by making the deoptimisation code completely independent from the rest of the system.

\paragraph{Chapter \ref{chap:persistence}} detailled how the runtime state is persisted across multiple start-ups in the context of Sista, including optimised v-functions, allowing to decrease warm-up time. 

\paragraph{Chapter \ref{chap:validation}} showed benchmarks run on the existing Pharo runtime and the Sista runtime. The Sista VM is 1.5x to 5x faster at peak performance than the current production runtime and peak performance is reached almost immediately after start-up if optimised v-functions are persisted as part of the snapshot.

\paragraph{Chapter \ref{chap:futureWork}} discussed future work that may be interesting based on the contents of this thesis.

\section{Contributions}

The main contributions of this thesis are:
\begin{itemize}
	\item The implementation of Sista, which yields 1.5x to 5x speed-up in execution time compared to the existing Pharo runtime and allow to reach peak performance very quickly if optimised v-functions are persisted from previous runs of the VM.
	\item An alternative bytecode set solving multiple existing encoding limitations.
	\item A language extension: each object can now be marked as read-only.
	\item An alternative implementation of closures, both allowing simplifications in existing code and enabling new optimisation possibilities.
\end{itemize}

\section{Published papers}

Here is a complete list of my publications in chronological order:

\begin{enumerate}
	\item Cl\'ement B\'era and Markus Denker. Towards a flexible Pharo Compiler. In International Workshop on Smalltalk Technologies 2013, IWST '13, 2013.
	\item Cl\'ement B\'era and Eliot Miranda. A bytecode set for adaptive optimizations. In International Workshop on Smalltalk Technologies 2014, IWST '14, 2014.
	\item Eliot Miranda and Cl\'ement B\'era. A Partial Read Barrier for Efficient Support of Live Object-oriented Programming. In International Symposium on Memory Management, ISMM '15, 2015.
	\item Cl\'ement B\'era, Eliot Miranda, Marcus Denker and St\'ephane Ducasse. Practical Validation of Bytecode to Bytecode JIT Compiler Dynamic Deoptimization. Journal of Object Technology, 2016.
	\item Nevena Milojkovi\'c, Cl\'ement B\'era, Mohammad Ghafari and Oscar Nierstrasz. Inferring Types by Mining Class Usage Frequency from Inline Caches. In International Workshop on Smalltalk Technologies IWST'16, 2016.
	\item Cl\'ement B\'era. A low Overhead Per Object Write Barrier for the Cog VM. In International Workshop on Smalltalk Technologies IWST'16, 2016.
\end{enumerate}

Here is the list of my publications waiting for approval:

\begin{enumerate}
	\item Nevena Milojkovi\'c, Cl\'ement B\'era, Mohammad Ghafari and Oscar Nierstrasz. Mining Inline Cache Data to Order Inferred Types in Dynamic Languages. Accepted with minor revisions in Science of Computer programming, SCP'17, 2017.
	\item Cl\'ement B\'era, Eliot Miranda, Tim Felgentreff, Marcus Denker and St\'ephane Ducasse. Sista: Saving Optimized Code in Snapshots for Fast Start-Up. Waiting to be resubmitted after last rejection, ECOOP'17, 2017.
\end{enumerate}

\section{Impact of the thesis}

Multiple contributions of the thesis (the alternative bytecode set, the read-only objects and the alternative implementation of closures) are integrated and interested the user-base of Pharo. The main impact is Sista, which is currently being integrated in Pharo and will hopefully enable new research and industrial work to be done with Pharo.

\ifx\wholebook\relax\else
    \end{document}
\fi

\appendix
\include{appendix/def}

\bibliographystyle{includes/ThesisStyle}
\bibliography{bib/sista}
%bib/rmod, bib/others


\end{document}

%%% Local Variables: 
%%% coding: utf-8
%%% mode: latex
%%% TeX-master: "main"
%%% TeX-PDF-mode: t
%%% End:
